\documentclass[12pt,oneside,openany]{book}
\usepackage[vietnamese]{babel}
\addto\captionsvietnamese{% Replace "english" with the language you use
    \renewcommand{\contentsname}{Contents}
    \renewcommand{\chaptername}{Part}
}

%%-----------------------------------------
\title{Probability and Statistic}
\author{Xuan Thuy}
\date{Apr 2023}
\let\cleardoublepage=\clearpage

%%-----------------------------------------
\begin{document}

\maketitle
\tableofcontents
\chapter{Probability}
\section{Random variables}

Một ví dụ thực tế thường được dùng để giải thích các khái niệm trong xác suất là ví dụ 
tung xúc xắc, ở đây ta sẽ sử dụng ví dụ này để minh họa các khái niệm. 
\\ \\
Biến ngẫu nhiên (random variable) là đại lượng được dùng để đại diện cho những 
giá trị ngẫu nhiên không xác định. Trong ví dụ về tung xúc xắc, biến ngẫu nhiên là 
số chấm thu được ở mặt trên của xúc xắc. 
\\ \\
Không gian mẫu (sample space) là tập hợp tất cả các giá trị mà một biến ngẫu nhiên có thể nhận.
Trong ví dụ về tung xúc xắc, không gian mẫu là tập hợp tất cả các khả năng mà mặt trên của xúc xắc
có thể nhận, cụ thể là một chấm, hai chấm, ba chấm, bốn chấm, năm chấm và sáu chấm. 
\\ \\
Biến cố (event) là một sự kiện xảy ra giúp ta nhận được một kết quả của biến ngẫu nhiên.
Trong ví dụ về tung xúc xắc, biến cố là sự kiện ta tung xúc xắc. 
\\ \\
Kết quả (outcome) là giá trị mà biến ngẫu nhiên nhận được sau biến cố xảy ra. 
Trong ví dụ về tung xúc xắc, sau khi tung xúc xắc, ta thu được kết quả là một chấm chẳng hạn. 
\\ \\
Khi ta thực hiện lặp đi lặp lại nhiều lần, ta có thể thu hoạch được nhiều kết quả khác nhau đối với
cùng biến ngẫu nhiên, sẽ có những kết quả xuất hiện nhiều lần hơn các kết quả khác, sẽ có những Kết
quả ít xuất hiện hơn các kết quả khác, sẽ có những kết quả có số lần xuất hiện xấp xỉ nhau.
Thông tin về đầu ra được đo bởi phân phối xác suất (probability distribution) của biến ngẫu nhiên.
\begin{equation}
    p(x)
\end{equation}

\section{Biến ngẫu nhiên dời rạc và biến ngẫu nhiên liên tục}
Một biến ngẫu nhiên có thể rời rạc (discrete) hoặc liên tục (continuous).
\\ \\
Trong ví dụ trên về việc tung xúc xắc 

\section{Joint probability}
\section{Marginalization}
\section{Conditional probability}
\section{Bayes' theorem}
\section{Independence}
\section{Expectation}

\chapter{Distribution}
\section{Bernoulli distribution}


\end{document}
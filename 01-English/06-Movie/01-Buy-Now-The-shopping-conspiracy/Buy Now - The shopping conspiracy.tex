\documentclass[a4paper]{article}
\usepackage{hyperref}
\usepackage{bilingual}
\usepackage[T5]{fontenc}
\usepackage[utf8]{inputenc}
\usepackage[vietnamese, english]{babel}


\title{Buy Now - The shopping conspiracy}
\author{Xuan Thuy}

\begin{document}
	\section*{Buy Now - The shopping conspiracy}
	
	\textit{\href{https://www.netflix.com/vn-en/title/81554996}{Link movie}}\\
	
	[pulsing] tiếng xung
	
	[Sasha] Welcome. [echoing]
	
	[video game beeping]\\
	
	Your presentation will begin in five, four, three, two, one.
	
	\begin{vietnamese-v2}
		Bài thuyết trình của bạn sẽ bắt đầu trong năm, bốn, ba, hai, một.	
	\end{vietnamese-v2}

	[assistant] ABC, common mark.
	
	\begin{vietnamese-v2}
		Common mark trong phim
	\end{vietnamese-v2}
	
	[Maren] Um... Yeah, Amazon actually had a patent on the phrase "one-click buying.".
	 \footnote{
		A patent protects inventions and technical innovations. Patent có nghĩa là bằng sáng chế chỉ bảo hộ các phát minh, quy trình kỹ thuật, thiết kế mới.
		Để bảo vệ một cụm từ (như slogan, tên thương hiệu), người ta đăng kí \textbf{trademark (nhãn hiệu thương mại)} chứ không phải patent.
		Nếu ai đó nói họ có \textbf{a patent on the phrase} thì thực tế họ muốn nói họ có trademark cho cụm từ đó, tức là quyền độc quyền sử dụng cụm từ đó trong kinh doanh để phân biệt sản phẩm hoặc dịch vụ của mình với người khác.
		}
	And Jeff... You know, this was Jeff's idea.
	
	\begin{vietnamese-v2}
		Ừ... Thực ra, Amazon đã có bằng sáng chế cho cụm từ 'mua hàng một cú nhấp chuột'. Và Jeff... Bạn biết đấy, đây là ý tưởng của Jeff.
	\end{vietnamese-v2}
	
	[mysterious music plays]
	I am a \textit{principal user experience designer}
	 \footnote{
		"Principal User Experience Designer" là vị trí cấp cao trong lĩnh vực thiết kế trải nghiệm người dùng (UX). Người giữ vai trò này chịu trách nhiệm lãnh đạo và định hướng chiến lược thiết kế trải nghiệm người dùng, làm việc với nhiều nhóm chức năng để giải quyết các thách thức kinh doanh phức tạp, đồng thời tạo ra các giải pháp thiết kế đáp ứng nhu cầu người dùng và mục tiêu doanh nghiệp. Họ thường có kinh nghiệm sâu rộng, dẫn dắt nhóm thiết kế, phối hợp với các bộ phận khác như sản phẩm, kỹ thuật, nghiên cứu để đảm bảo trải nghiệm người dùng nhất quán và hiệu quả.
		
		
		
		Tóm lại, \textbf{Principal UX Designer} là chuyên gia thiết kế trải nghiệm người dùng ở cấp cao, vừa làm công việc thiết kế, vừa lãnh đạo và định hướng chiến lược UX cho sản phẩm hoặc dịch vụ.
		
		Principal (adjective) means first in order of importance; main.
		
		Principal (noun) means the person with the highest authority or most important position in an organization, institution, or group..}.
	
	\begin{vietnamese-v2}
		Tôi là một nhà thiết kế trải nghiệm người dùng cấp cao.
	\end{vietnamese-v2}
	
	I worked at Amazon for 15 years.
	I worked on the product detail pages, on launching different categories in different countries.
	
	\begin{vietnamese-v2}
		Tôi đã làm cho Amazon trong 15 năm.
		Tôi đã làm việc trên các trang chi tiết sản phẩm, cũng như triển khai ra mắt các danh mục khác nhau ở các quốc gia khác nhau.
	\end{vietnamese-v2}
	
	You know, I feel like there wasn't a page on that site I didn't touch.
	
	\begin{vietnamese-v2}
		Bạn biết đấy, tôi cảm thấy như không có trang nào trên trang web đó mà tôi không chạm vào.
	\end{vietnamese-v2}
	\pagebreak
	
	[mouse clicking]
	The system was really being built and optimized to help you buy everything that you ever need, and more of it than you ever thought you needed.
	
	\begin{vietnamese-v2}
		Hệ thống thực sự đang được xây dựng và tối ưu hóa để giúp bạn mua mọi thứ bạn cần, và nhiều hơn nữa những thứ bạn từng nghĩ là mình cần.
	\end{vietnamese-v2}
	
	
	[mouse clicking] [man] Oh my gosh.
	
	[producer] Was there any thought while you were there about what happens after the stuff is sold?
	
	No.
	
	\begin{vietnamese-v2}
		Khi bạn ở đó, có ai nghĩ về chuyện sẽ xảy ra sau khi món hàng được bán không?
		
		Không.
	\end{vietnamese-v2}
	
	[soft electronic music plays]
	
	We got some packages.
	
	[Mara] Buying new stuff feels great, right?
	
	- This thing's dope. - Oh!
	
	\begin{vietnamese-v2}
		[nhạc điện tử nhẹ nhàng vang lên]
		
		Chúng ta có vài gói hàng.
		
		[Mara] Mua đồ mới thật tuyệt, phải không?
		
		Cái này thật chất. - Ồ!
	\end{vietnamese-v2}
	
	[Mara] The problem is that every year. We're consuming more, producing more  \footnote{
		Consume nghĩa là tiêu thụ. Tiêu thụ năng lượng, thời gian, nhiên liệu. Produce là sản xuất.
		
		Tiêu dùng nhiều hơn thường có nghĩa là gia tăng lượng hàng hóa, dịch vụ hoặc tài nguyên được sử dụng. Nó có thể đề cập đến việc mua nhiều sản phẩm hơn, sử dụng nhiều năng lượng hơn hoặc thậm chí tiếp nhận nhiều thông tin hơn. Về mặt kinh tế, mức tiêu dùng cao hơn thúc đẩy nhu cầu, từ đó có thể dẫn đến sự gia tăng sản xuất.
		
		Sản xuất nhiều hơn có nghĩa là tạo ra thêm hàng hóa, dịch vụ hoặc sản lượng. Các doanh nghiệp, ví dụ, có xu hướng sản xuất nhiều hơn khi nhu cầu tăng, trong khi cá nhân có thể tăng năng suất trong công việc hoặc trong các hoạt động sáng tạo.
	}, and there's a flip side \footnote{
		"Flip side" có nghĩa là mặt đối lập hoặc khía cạnh khác của một vấn đề, tình huống hoặc sự việc. Nó thường được dùng để chỉ góc nhìn hoặc hậu quả khác biệt so với điều đang được đề cập.
		
	} to that that no one wants you to see.
	
	\begin{vietnamese-v2}
		[Mara] Vấn đề là mỗi năm, chúng ta tiêu thụ nhiều hơn, sản xuất nhiều hơn, và có một mặt trái mà không ai muốn bạn thấy
	\end{vietnamese-v2}
	
	[Sasha] Designed to stand out  \footnote{
		"Stand out" có nghĩa là nổi bật, khác biệt hoặc thu hút sự chú ý so với những thứ xung quanh. Nó có thể áp dụng cho người, vật, ý tưởng hoặc bất kỳ điều gì có sự khác biệt đáng chú ý.
	} for a generation that doesn't stop.
	
	\begin{vietnamese-v2}
		[Sasha] Được thiết kế để nổi bật với một thế hệ không ngừng phát triển.
	\end{vietnamese-v2}

	\pagebreak
	
	[Maren] You're being 100\% played, and it's... it's a science. It's an intentional complex, highly-refined science  \footnote{
		\textbf{Intentional -->} Có chủ ý, được thực hiện một cách có ý thức và có mục đích rõ ràng.
		
		\textbf{Complex -->} Phức tạp, tinh vi
		
		\textbf{Highly-refined -->} Được tinh chỉnh cao, chính xác
		
		\textbf{Science -->} Khoa học
		
		Cụm từ "intentional complex, highly-refined science" có thể hiểu là "khoa học phức tạp có chủ đích, được tinh chỉnh ở mức độ cao". Nó gợi ý về một lĩnh vực nghiên cứu hoặc phương pháp tiếp cận có tính toán, tinh vi và mang tính phân tích sâu sắc—có thể là khoa học lý thuyết tiên tiến, nghiên cứu công nghệ cắt cạnh, hoặc các phương pháp khoa học cực kỳ chính xác.
	
	} 
	
	[chuckles] to get you to buy stuff.
	
	\begin{vietnamese-v2}
		[Maren] Bạn đang bị lừa 100\%, và đó... đó là một khoa học. Đó là một khoa học phức tạp, tinh vi và có chủ ý để khiến bạn mua đồ.
	\end{vietnamese-v2}
	
	It's not just Amazon.
	
	Most of the big corporations  \footnote{
		Corporations (công ty cổ phần hoặc tập đoàn) là một loại hình doanh nghiệp có tư cách pháp nhân độc lập, nghĩa là nó tồn tại riêng biệt với chủ sở hữu của nó. Một số đặc điểm chính của corporations:
		
		\textbf{Tư cách pháp nhân:} Một corporation có thể ký hợp đồng, sở hữu tài sản, và chịu trách nhiệm pháp lý giống như một cá nhân.
		
		\textbf{Cổ đông:} Một corporation được sở hữu bởi các cổ đông, những người nắm giữ cổ phần (shares) của công ty.
		
		\textbf{Trách nhiệm hữu hạn:} Các cổ đông không chịu trách nhiệm cá nhân đối với các khoản nợ hoặc nghĩa vụ của công ty—họ chỉ mất số tiền đã đầu tư.
		
		\textbf{Quản lý bởi hội đồng quản trị:} Corporation thường có ban giám đốc (Board of Directors) để đưa ra các quyết định chiến lược.
		
		\textbf{Dễ mở rộng:} Do có thể phát hành cổ phiếu, corporations có khả năng huy động vốn dễ dàng hơn so với doanh nghiệp cá nhân hay hợp tác xã.
		
		Corporations có thể là tư nhân hoặc công khai (public company), tức là cổ phiếu của họ được giao dịch trên thị trường chứng khoán. Một số tập đoàn nổi tiếng trên thế giới bao gồm Microsoft, Apple, và Samsung.
	
	
	} are doing it, and every trick in the book is being used to hide what's really going on.
	
	\begin{vietnamese-v2}
		Không chỉ Amazon.
		
		Hầu hết các tập đoàn lớn đều đang làm điều đó, và mọi thủ thuật trong sách đều được sử dụng để che giấu những gì thực sự đang diễn ra.
	\end{vietnamese-v2}
	
	[woman] Holy shh...
	Dude, they destroyed that.
	I was the president of... of... of the Adidas brand.
	
	\begin{vietnamese-v2}
		[Người phụ nữ] Trời ơi...
		
		Anh bạn, họ đã phá hủy nó.
		
		Tôi là chủ tịch của... của... của thương hiệu Adidas.
	\end{vietnamese-v2}
	
	\pagebreak
	
	I think I definitely, um, feel like I've got some sins \footnote{
		\textbf{Sins (tội lỗi)} thường được hiểu theo hai cách chính:
		
		Quan điểm tôn giáo: Trong nhiều tôn giáo, "sin" là hành vi đi ngược lại các nguyên tắc đạo đức hoặc luật lệ thiêng liêng. Nó có thể bao gồm những hành động gây hại cho người khác, vi phạm niềm tin đạo đức, hoặc đi ngược lại ý muốn của một đấng tối cao. Ví dụ, trong Kitô giáo, có "bảy tội lỗi chết người" (Seven Deadly Sins) như tham lam, đố kỵ, kiêu ngạo...
		
		Quan điểm xã hội \& cá nhân: Ngoài phạm vi tôn giáo, "sins" đôi khi được dùng để chỉ hành động sai trái về mặt đạo đức hoặc pháp lý—những điều mà xã hội coi là xấu hoặc đáng lên án.
		
		Từ "sins" còn xuất hiện nhiều trong văn hóa, nghệ thuật và triết học, phản ánh sự đấu tranh giữa điều thiện và điều ác trong con người.
	} to make up for \footnote{Cụm động từ \textbf{"make up for"} có nghĩa là \textbf{bù đắp, đền bù,} hoặc \textbf{bồi thường} cho một điều gì đó bị mất mát hoặc thiếu hụt}.
	
	\begin{vietnamese-v2}
		Tôi nghĩ chắc chắn, ừm, tôi cảm thấy như mình có một số lỗi lầm cần chuộc lại.
	\end{vietnamese-v2}
	
	
	There's definitely things I participated in that I feel like I could've, um, and should've done better.
	
	\begin{vietnamese-v2}
		Có những điều tôi đã tham gia mà tôi cảm thấy lẽ ra tôi có thể, ừm, và nên làm tốt hơn.
	\end{vietnamese-v2}
	
	
	I just bought 500 iPhones!
	
	Oh my God!

	I started at Apple, joined the founding team of Oculus.

	The voices of, "We're going to protect the business" went out over the little voice in your head saying,

	"Should we really be doing this?"  \footnote{
		Câu \textbf{"Should we really be doing this?"} có cấu trúc ngữ pháp như sau:
	
			\textbf{Should} → Động từ khiếm khuyết (modal verb), dùng để diễn đạt sự cần thiết, lời khuyên hoặc nghi vấn về tính hợp lý của hành động.
			
			\textbf{we} → Đại từ chủ ngữ (subject pronoun), chỉ người nói và ít nhất một người khác.
			
			\textbf{really} → Trạng từ (adverb), dùng để nhấn mạnh mức độ nghi vấn hoặc sự chân thật của câu.
			
			\textbf{be doing} → Dạng tiếp diễn của động từ "do":
			
			\textbf{be} → Động từ "to be" trong dạng nguyên mẫu (infinitive) sau "should".
			
			\textbf{doing} → Động từ chính ở dạng hiện tại tiếp diễn (present continuous), cho thấy hành động đang diễn ra hoặc có kế hoạch được thực hiện.
			
			\textbf{this} → Đại từ chỉ định (demonstrative pronoun), thay thế cho một hành động hoặc sự việc đã được đề cập hoặc ngầm hiểu.
		
	}
	
	\begin{vietnamese-v2}
		Tôi vừa mua 500 chiếc iPhone!
		
		Ôi trời ơi!
		
		Tôi bắt đầu tại Apple, tham gia vào đội ngũ sáng lập của Oculus.
		
		Những giọng nói vang lên, "Chúng ta sẽ bảo vệ công việc kinh doanh," lấn át tiếng nói nhỏ trong đầu bạn đang tự hỏi,
		
		"Chúng ta có thực sự nên làm điều này không?"
	\end{vietnamese-v2}
	
	[crack]
	Whatever the brands are doing right now, it's hurting a lot of people.
	
	\begin{vietnamese-v2}
		[Bể nứt]
		
		Dù các thương hiệu đang làm gì vào lúc này, nó đang gây tổn hại cho rất nhiều người.
	\end{vietnamese-v2}
	
	\pagebreak

	[Jim] These things are not just litter. These things are hazardous waste \footnote{
		\textbf{"Hazardous waste"} có nghĩa là \textbf{chất thải nguy hại}, tức là các loại chất thải có thể gây nguy hiểm cho sức khỏe con người hoặc môi trường
		
		\textbf{Hazardous} → Tính từ (adjective), có nghĩa là nguy hiểm.
	
	}. We're drinking and breathing them, and they're poison.

	\begin{vietnamese-v2}
		[Jim] Những thứ này không chỉ là rác thải. Chúng là chất thải nguy hại. Chúng ta đang uống và hít thở chúng, và chúng là chất độc.
	\end{vietnamese-v2}
	
	[Mara] It's like that film Wall-E. The amount of stuff we're being encouraged to consume \footnote{
		Cụm từ \textbf{"encouraged to consume"} có nghĩa là được khuyến khích tiêu thụ một thứ gì đó, thường là thực phẩm, sản phẩm hoặc tài nguyên
	
		Từ \textbf{"encouraged"} trong tiếng Anh là dạng quá khứ hoặc phân từ của động từ \textbf{"encourage"}.
		
		\textbf{Encourage} có nghĩa là khuyến khích, động viên, cổ vũ ai đó làm điều gì đó.
		
		\textbf{Encouraged} = được khuyến khích, được động viên, cảm thấy có động lực.
	}, the waste from this is getting everywhere, and it's affecting everyone on the planet.
	
	\begin{vietnamese-v2}
		[Mara] Nó giống như bộ phim Wall-E vậy. Lượng đồ dùng mà chúng ta được khuyến khích tiêu thụ, lượng rác thải từ đó đang tràn ngập khắp nơi, và nó đang ảnh hưởng đến tất cả mọi người trên hành tinh.
	\end{vietnamese-v2}
	
	[mysterious electronic music]
	[Maren] I went to say customers are going to be pissed off \footnote{
		\textbf{"Pissed off"} là một cụm từ tiếng Anh thông dụng, mang tính không trang trọng (informal/slang).
	
		Nó có nghĩa là \textbf{rất tức giận, bực mình, khó chịu}.
	} when they realize that they have been contributing \footnote{
		\textbf{"Contributing"} là dạng hiện tại phân từ của động từ \textbf{"contribute"}, mang nghĩa là \textbf{tham gia hoặc đóng góp vào một quá trình, hoạt động hoặc kết quả nào đó}.
		
		\textbf{Contributing} có nghĩa là hành động \textbf{đóng góp, góp phần} (có thể là tiền bạc, công sức, ý tưởng, thời gian...) để giúp đỡ hoặc tạo ra một kết quả chung.
		
		Từ này thường dùng khi nói về việc ai đó hoặc cái gì đó góp phần vào sự thành công, phát triển hoặc ảnh hưởng đến một sự việc.
	
	} to their own demise \footnote{
		Từ \textbf{"demise"} trong tiếng Anh có nhiều nghĩa, tùy theo ngữ cảnh:
	
			\textbf{Sự chết, sự qua đời, sự băng hà} (thường dùng cho người, đặc biệt là vua chúa hoặc những người quan trọng).
			
			\textbf{Sự sụp đổ, sự suy vong, sự kết thúc} của một tổ chức, hệ thống, doanh nghiệp hoặc quyền lực nào đó.
	
	}, but they didn't wanna hear it. I looked around \footnote{
		Câu \textbf{"I looked around"} trong tiếng Anh có nghĩa là:
		
			\textbf{Tôi nhìn quanh}
			
			\textbf{Tôi quan sát xung quanh}
	
	}, and I was like, "How did this happen?" 
	
	\begin{vietnamese-v2}
		[Maren] Tôi định nói rằng khách hàng sẽ tức giận khi nhận ra rằng họ đã góp phần vào sự sụp đổ của chính mình, nhưng họ không muốn nghe điều đó. Tôi nhìn xung quanh và tự hỏi, "Chuyện này đã xảy ra như thế nào?"
	\end{vietnamese-v2}
	
	\pagebreak{}
	
	"And... And what do we do now?"
	
	[clicking]
	
	[Sasha] Play title.
	
	[gurgling]
	
	[silence]
	
	[music chord grows]
	
	[Sasha] Hello.
	
	I'm Sasha. A personal assistant created to offer unfiltered insights \footnote{
		Ý nghĩa của cụm từ \textbf{"unfiltered insights"}
	
	
	} on how to succeed in business.
	
	\begin{vietnamese-v2}
		"Và... Và bây giờ chúng ta làm gì?"
		
		[tiếng nhấp chuột]
		
		[Sasha] Phát tiêu đề.
		
		[tiếng ục ục]
		
		[sự im lặng]
		
		[hợp âm nhạc vang lên]
		
		[Sasha] Xin chào.
		
		Tôi là Sasha, một trợ lý cá nhân được tạo ra để cung cấp những hiểu biết thẳng thắn về cách thành công trong kinh doanh.
	\end{vietnamese-v2}
	
	
	
	[tinkling]
	- I can even create my own bespoke imagery. \footnote{
		\textbf{"Bespoke"} là một từ tiếng Anh bắt nguồn từ cụm "be spoken for," nghĩa là đã được đặt hàng riêng biệt. Bespoke mô tả các sản phẩm được thiết kế và chế tác thủ công theo yêu cầu cá nhân, với đặc điểm nổi bật là sự cá nhân hóa, độc nhất vô nhị, chất lượng cao và thể hiện phong cách riêng biệt của người sở hữu.
		
		\textbf{"Imagery"} là danh từ chỉ hình ảnh hoặc các biểu tượng thị giác được sử dụng để truyền đạt ý tưởng, cảm xúc hoặc thông tin.
		
		\textbf{Bespoke imagery} là hình ảnh được tạo ra hoặc thiết kế riêng biệt, mang tính cá nhân hóa cao, không đại trà mà độc nhất, phản ánh chính xác phong cách, sở thích và cá tính của người đặt hàng hoặc chủ sở hữu
	}
	- [dolphin clicks]
	
	\begin{vietnamese-v2}
		[tiếng leng keng]
		
		Tôi thậm chí có thể tạo ra hình ảnh tùy chỉnh của riêng mình.
		
		[tiếng lách cách của cá heo]
	\end{vietnamese-v2}
	
	[birds chirping]
	Cut to time-lapse footage  \footnote{
		\textbf{"Time-lapse"} là kỹ thuật quay phim hoặc chụp ảnh với tốc độ khung hình thấp hơn bình thường, sau đó phát lại với tốc độ chuẩn, làm cho các chuyển động chậm như sự phát triển của cây cối, mây trôi, hoặc các hiện tượng thiên nhiên khác xuất hiện nhanh hơn nhiều so với thực tế.
		
		\textbf{"Footage"} là đoạn phim hoặc các cảnh quay đã được ghi lại bằng máy quay phim hoặc camera.
	
	} of growth in the natural world.
	
	
	[birds continue chirping]
	Over the course of this unique and entertaining \footnote{
	 	\textbf{Entertaining} là tính từ, mang nghĩa "vui nhộn, thú vị, làm cho người khác cảm thấy thích thú hoặc được giải trí".
	 	
	} interaction \footnote{
		\textbf{Interaction} là danh từ, chỉ sự tương tác, giao tiếp hay hành động qua lại giữa hai hoặc nhiều người hoặc giữa người và vật.
		
		\textbf{Interaction} bao gồm việc nói chuyện, nhìn nhau, chia sẻ hoặc tham gia vào bất kỳ hành động nào có sự tham gia của các bên liên quan.
	
	}, you will receive the five most important lessons in profit maximization.
	
	\begin{vietnamese-v2}
		[tiếng chim hót]
		
		Chuyển cảnh sang cảnh tua nhanh thời gian về sự phát triển trong thế giới tự nhiên.
	\end{vietnamese-v2}
	
	
	[choir harmonizes]
	There will also be a surprise for those who stay engaged \footnote{
		\textbf{"stay engaged"} nghĩa là duy trì sự tập trung, không rời bỏ hoặc không mất hứng trong suốt quá trình.
	
	} until the end.
	
	\begin{vietnamese-v2}
		[tiếng hợp xướng hòa âm]
		
		Sẽ có một điều bất ngờ dành cho những ai theo dõi đến cuối cùng.
	\end{vietnamese-v2}
	
	[beeps]
	Rule one.
	Sell more. [echoing]
	[beeping]
	Selling is key to success. But consumers will need constant motivation to increase the amounts they purchase.
	Thankfully, the fashion industry offers a textbook example \footnote{
		\textbf{a textbook example:} ví dụ điển hình, mẫu mực, thường được dùng để minh họa cho một khái niệm hay hiện tượng.
	
	} of rapid \footnote{
		\textbf{rapid:} nhanh chóng, tốc độ cao
	
	}, self-driven growth.
	
	\begin{vietnamese-v2}
		[tiếng bíp]
		
		Quy tắc đầu tiên. Bán nhiều hơn. [vọng lại]
			
		[tiếng bíp]
		
		Bán hàng là chìa khóa để thành công. Nhưng người tiêu dùng sẽ cần động lực liên tục để tăng số lượng hàng họ mua.
		
		May mắn thay, ngành công nghiệp thời trang cung cấp một ví dụ điển hình về sự tăng trưởng nhanh chóng và tự phát.
		
		Ngành thời trang là một ví dụ điển hình về sự tăng trưởng nhanh và tự nhiên, phát triển mạnh mẽ nhờ các động lực bên trong ngành, như nhu cầu tiêu dùng liên tục, đổi mới sáng tạo, và khả năng thích ứng nhanh với xu hướng thị trường
	\end{vietnamese-v2}
	
	
	[Eric] No one needs a new hoodie. No one needs a new T-shirt. No one needs a new pair of shoes. There's so many out there, and most of us have more than enough. What you need is a compelling  \footnote{
		\textbf{Compelling} là tính từ, mang nghĩa "thuyết phục, hấp dẫn, khiến người khác tin tưởng hoặc chấp nhận điều gì đó".
	
		\textbf{"Compelling reason"} là một lý do hoặc nguyên nhân đủ mạnh, thuyết phục để khiến người khác tin tưởng, chấp nhận hoặc hành động theo.
	} reason to buy that product.
	
	\begin{vietnamese-v2}
		[Eric] Không ai thực sự cần một chiếc áo hoodie mới. Không ai cần một chiếc áo phông mới. Không ai cần một đôi giày mới. Có quá nhiều ngoài kia, và hầu hết chúng ta đã có đủ. Điều bạn cần là một lý do hấp dẫn để mua sản phẩm đó.
	\end{vietnamese-v2}
	
	As a board member, it was one of the most important things I had on my to-do list.
	You know, what are the sales we're generating?
	And what are the profits we're generating?
	If we weren't generating those two things, we weren't doing it.
	
	\begin{vietnamese-v2}
		Là một thành viên hội đồng quản trị, đó là một trong những điều quan trọng nhất trong danh sách công việc của tôi.
		
		Bạn biết đấy, doanh số bán hàng mà chúng ta tạo ra là gì? Và lợi nhuận chúng ta thu được là bao nhiêu?
		
		Nếu chúng ta không tạo ra hai yếu tố đó, nghĩa là chúng ta không làm đúng nhiệm vụ của mình.
	\end{vietnamese-v2}
	
	
	Sneakers are big business in America. 
	Nearly half a billion pairs were sold last year alone. 
	This week Adidas unveiled  \footnote{
		\textbf{unveiled:} động từ quá khứ của "unveil", nghĩa là "ra mắt", "giới thiệu công khai" một sản phẩm hoặc dự án lần đầu tiên.
		
	} the so-called \footnote{
	
	\textbf{the so-called:} cụm từ dùng để giới thiệu hoặc nhấn mạnh một tên gọi, có thể mang sắc thái nghi vấn hoặc nhấn mạnh rằng đây là tên gọi phổ biến hoặc được biết đến, nghĩa là "cái gọi là".
	
	} smart shoe.
	
	\begin{vietnamese-v2}
		Giày thể thao là một ngành kinh doanh lớn ở Mỹ.  
		
		Gần nửa tỷ đôi đã được bán chỉ trong năm ngoái.  
		
		Tuần này, Adidas đã ra mắt sản phẩm được gọi là giày thông minh.
	\end{vietnamese-v2}
	
	
	[Eric] We talk about, hey, we need to differentiate ourselves from Nike, or from Puma, or somebody else. It's not about that.
	It's how you're relevant  \footnote{
		\textbf{Relevant:} Tính từ, nghĩa là "liên quan", "phù hợp", "đáng chú ý" hoặc "có ý nghĩa".
		
		Ở đây, "relevant" mang nghĩa bạn có sự liên kết hoặc phù hợp với đối tượng nào đó.
	} to the consumer. I was responsible for all product, all communications, all digital efforts, um, and all strategy.
	
	\begin{vietnamese-v2}
		[Eric] Chúng tôi nói về việc cần phải khác biệt so với Nike, Puma hoặc một thương hiệu khác. Nhưng vấn đề không phải ở đó.
		
		Điều quan trọng là cách bạn có ý nghĩa đối với người tiêu dùng.
		
		Tôi chịu trách nhiệm về tất cả sản phẩm, tất cả hoạt động truyền thông, mọi nỗ lực kỹ thuật số, ừm, và toàn bộ chiến lược.
	\end{vietnamese-v2}
	
	2014, Adidas was bleeding out \footnote{
		\textbf{"Bleeding out"} là cụm từ ẩn dụ, nghĩa đen là "chảy máu nhiều", ở đây dùng để mô tả tình trạng nghiêm trọng, công ty đang chịu tổn thất lớn, mất mát tài chính hoặc gặp khủng hoảng nghiêm trọng.
	
	}, we... we had hit the floor \footnote{
		\textbf{"Hit the floor"} nghĩa bóng là "chạm đáy", "rơi xuống mức thấp nhất".
		
	}, 
	everything was going south \footnote{
		\textbf{"Going south"} là thành ngữ tiếng Anh, nghĩa là mọi thứ đang đi xuống, trở nên tồi tệ, suy giảm hoặc thất bại.
		
	}.
	It was the greatest turnaround in the history of the company, if not the history of the industry.

	\begin{vietnamese-v2}
		Năm 2014, Adidas đang gặp khủng hoảng nghiêm trọng, chúng tôi... chúng tôi đã chạm đáy, mọi thứ đang đi xuống.  
		
		Đó là sự hồi sinh vĩ đại nhất trong lịch sử của công ty, nếu không muốn nói là trong cả ngành công nghiệp.
	\end{vietnamese-v2}

	[reporter 1] Shares of Adidas \footnote{
		\textbf{Shares of Adidas:} cổ phiếu của Adidas, tức là phần vốn đại diện cho quyền sở hữu trong công ty Adidas được giao dịch trên thị trường chứng khoán.
	
	} enjoying huge momentum. \footnote{
		\textbf{huge momentum:} động lực lớn, sức bật mạnh mẽ, xu hướng tăng trưởng hoặc chuyển động tích cực đáng kể.
	
	}
	[reporter 2] Shares of Adidas, look at this, they're rising overnight, up about 7.5%.
	
	\begin{vietnamese-v2}
		[phóng viên 1] Cổ phiếu của Adidas đang có đà tăng trưởng mạnh mẽ.
		
		[phóng viên 2] Cổ phiếu của Adidas, hãy nhìn vào đây, chúng đang tăng lên trong đêm, tăng khoảng 7,5%.
	\end{vietnamese-v2}
	
	\pagebreak
	
	[Eric] How'd we grow so big? It's a question I've always asked myself. I think it's a combination of a few things \footnote{
		\textbf{"A combination of a few things"} nghĩa là sự kết hợp của một vài yếu tố hoặc thành phần khác nhau.
	
	}.
	First of all, it's about a story. 
	
	\begin{vietnamese-v2}
		[Eric] Chúng tôi đã phát triển lớn mạnh như thế nào? Đó là câu hỏi mà tôi luôn tự hỏi. Tôi nghĩ đó là sự kết hợp của một vài yếu tố.
		
		Trước hết, đó là về một câu chuyện.
	\end{vietnamese-v2}
	
	Think about an English football team.
	[crowd cheers]
	How many jerseys \footnote{
		\textbf{Danh từ số nhiều của "jersey"}, nghĩa là áo thể thao, áo đấu của các đội bóng hoặc vận động viên.
	
	} do you think they have a year?
	Not just one to play in. There's the home jersey, the away jersey, the third jersey.
	
	\begin{vietnamese-v2}
		Hãy nghĩ về một đội bóng đá Anh.
		
		[tiếng cổ vũ của đám đông]
		
		Bạn nghĩ họ có bao nhiêu mẫu áo đấu mỗi năm?
		
		Không chỉ có một chiếc để thi đấu. Có áo sân nhà, áo sân khách, và cả áo thứ ba.
	\end{vietnamese-v2}
	
	[crowd cheers]
	There's the different jerseys you have to celebrate different occasions.
	
	\begin{vietnamese-v2}
		[tiếng cổ vũ của đám đông]
		
		Có những mẫu áo khác nhau để kỷ niệm những dịp đặc biệt.
	\end{vietnamese-v2}
	
	[referee whistles]
	Why do you think they do that?
	To create a new story, and to create a new buying opportunity.
	
	\begin{vietnamese-v2}
		[tiếng còi trọng tài vang lên]
		
		Bạn nghĩ tại sao họ làm điều đó?
		
		Để tạo ra một câu chuyện mới và mở ra một cơ hội mua sắm mới.
	\end{vietnamese-v2}
	
	[rap music]
	♪ Ah, if you catch me on a whale ♪
	
	\begin{vietnamese-v2}
		[nhạc rap]
		
		♪ Ah, nếu bạn bắt gặp tôi trên một con cá voi ♪
	\end{vietnamese-v2}
	
	
	[Eric] At Adidas, we recognize that the culture of sport doesn't stop when you're done with the game when the whistle blows \footnote{
		\textbf{When the whistle blows:} cụm này nghĩa đen là "khi tiếng còi vang lên", thường dùng để chỉ thời điểm báo hiệu bắt đầu hoặc kết thúc một sự kiện, hoạt động hoặc hành động.
	
	}.
	
	\begin{vietnamese-v2}
		[Eric] Tại Adidas, chúng tôi nhận ra rằng văn hóa thể thao không kết thúc khi trận đấu kết thúc, khi tiếng còi vang lên.
	\end{vietnamese-v2}
	
	\pagebreak
	
	♪ There's money over that side ♪
	
	[Eric] It goes into \footnote{
		\textbf{goes into:} động từ cụm, nghĩa là "đi vào", "lan tỏa vào" hoặc "xuất hiện trong" một nơi nào đó.
	
	} the streets, and into the music venues \footnote{
		\textbf{"Music venues"} là các địa điểm tổ chức biểu diễn âm nhạc, như phòng hòa nhạc, quán bar, sân khấu.
	
	}, into the... into the hallways of schools.
	
	So we started to do things with musicians and artists.
	
	Beyoncé Knowles.
	
	Pharrell Williams.
	
	And one of the most, you know, controversial, um, artists in the history of mankind named Kanye West.

	The storytelling is so critical in this industry to really... to really drive consumption.
	
	\begin{vietnamese-v2}
		♪ Có tiền ở phía đó ♪
		
		[Eric] Nó lan tỏa vào đường phố, vào các địa điểm âm nhạc, vào... vào hành lang của các trường học.
		
		Vì vậy, chúng tôi bắt đầu hợp tác với các nhạc sĩ và nghệ sĩ.
		
		Beyoncé Knowles. Pharrell Williams. Và một trong những nghệ sĩ gây tranh cãi nhất trong lịch sử nhân loại, Kanye West.
		
		Việc kể chuyện đóng vai trò vô cùng quan trọng trong ngành này để thực sự... thúc đẩy tiêu dùng.
		
		Beyoncé đã hợp tác với Adidas thông qua thương hiệu Ivy Park, với những bộ sưu tập mang phong cách mạnh mẽ và sáng tạo. Pharrell Williams cũng có một mối quan hệ lâu dài với Adidas, với dòng sản phẩm Humanrace mang đậm dấu ấn cá nhân của anh. Kanye West, với thương hiệu Yeezy, đã tạo ra một trong những cú hích lớn nhất trong lịch sử Adidas trước khi mối quan hệ hợp tác này kết thúc vào năm 2022.
	\end{vietnamese-v2}
	
	[crowd chatters]
	And that's what the role of Adidas is \footnote{
		\textbf{the role of Adidas is:} vai trò của Adidas là...
	
	}, and every other fashion brand is, is how do you create "objects of desire"? \footnote{
		\textbf{objects of desire:} "những vật thể khao khát" hay "đối tượng được mong muốn", nghĩa là những sản phẩm mà người tiêu dùng rất muốn sở hữu, có sức hấp dẫn mãnh liệt.
	
	}
	This is where the fashion industry's really good. They know you.
	[chuckles] Like, we know you.
	
	\begin{vietnamese-v2}
		[tiếng đám đông trò chuyện]
		
		Và đó là vai trò của Adidas cũng như mọi thương hiệu thời trang khác—làm thế nào để tạo ra những "đối tượng khao khát"?
		
		Đây chính là điểm mạnh của ngành công nghiệp thời trang. Họ hiểu bạn.
		
		[chuckles] Như cách chúng tôi hiểu bạn.
	\end{vietnamese-v2}
	
	We spend a lot of time on consumer research to understand the different consumers.
	And we know the consumers to approach with different messages.
	
	\begin{vietnamese-v2}
		Chúng tôi dành rất nhiều thời gian để nghiên cứu người tiêu dùng nhằm hiểu rõ từng nhóm khách hàng khác nhau.  
		
		Và chúng tôi biết cách tiếp cận người tiêu dùng với những thông điệp phù hợp.
	\end{vietnamese-v2}
	
	\pagebreak
	
	[female voice over] Adidas by Gucci.
	Listen, I participated in... in my career with creating more at a faster pace \footnote{
		\textbf{Pace:} Danh từ, nghĩa là nhịp độ, tốc độ của một hoạt động, quá trình hoặc sự kiện.
	
		\textbf{"Faster pace"} nghĩa là một nhịp độ nhanh hơn, tốc độ diễn ra nhanh hơn
	}.
	I don't wanna point just at Adidas. This is happening with all fashion brands.
	
	\begin{vietnamese-v2}
		[giọng nữ] **Adidas by Gucci.**  
		
		Nghe này, tôi đã tham gia vào... trong sự nghiệp của mình với việc tạo ra nhiều sản phẩm hơn với tốc độ nhanh hơn.  
		
		Tôi không muốn chỉ ra riêng Adidas. Điều này đang diễn ra với tất cả các thương hiệu thời trang.  
		
		Adidas và Gucci đã hợp tác để tạo ra một bộ sưu tập kết hợp giữa phong cách thể thao và thời trang cao cấp. Bộ sưu tập này bao gồm giày thể thao, quần áo và phụ kiện mang dấu ấn của cả hai thương hiệu, với thiết kế độc đáo và chất liệu cao cấp.
	\end{vietnamese-v2}
	
	[whoosh]
	[Roger] I've been in the industry for almost 20 years.
	Probably one out of six \footnote{
		\textbf{one out of six:} Đây là cách diễn đạt tỷ lệ hoặc phần trăm, nghĩa là "một trong sáu".
	
	} dress shirts sold in the US is made by us.
	
	\begin{vietnamese-v2}
		[tiếng vút qua]  
		
		[Roger] Tôi đã làm việc trong ngành này gần 20 năm.  
		
		Có lẽ cứ sáu chiếc áo sơ mi được bán ở Mỹ thì có một chiếc do chúng tôi sản xuất.  
	\end{vietnamese-v2}
	
	[intriguing music plays]
	
	The type of customers we service could be anywhere from your high street brands to global designer brands that sell worldwide.
	
	When I first joined, you had two seasons a year, so you'd make something that'd sell for six months.
	
	\begin{vietnamese-v2}
		[nhạc đầy cuốn hút vang lên]  
		
		Loại khách hàng mà chúng tôi phục vụ có thể bao gồm từ các thương hiệu phổ biến trên phố đến các thương hiệu thiết kế toàn cầu bán sản phẩm trên khắp thế giới.  
		
		Khi tôi mới gia nhập, mỗi năm chỉ có hai mùa thời trang, vì vậy bạn sẽ thiết kế một sản phẩm và nó sẽ được bán trong sáu tháng.  
	\end{vietnamese-v2}
	
	Nowadays, with the introduction of fast fashion, it's forced other brands to rethink about having newness every month.
	
	So, no one knows any official numbers, but if you Google it, it will show that Gap produces around 12,000 new items a year.
	
	H\&M's like 25,000.
	
	Zara's like 36,000.
	
	And SHEIN is somewhere around 1.3 million new items a year.
	
	\begin{vietnamese-v2}
		Với sự xuất hiện của thời trang nhanh, các thương hiệu khác buộc phải suy nghĩ lại về việc liên tục tung ra sản phẩm mới mỗi tháng.  
		
		Không có con số chính thức, nhưng nếu bạn tìm kiếm trên Google, bạn sẽ thấy rằng:  
		
		Gap sản xuất khoảng 12.000 mặt hàng mới mỗi năm.  
		
		H\&M khoảng 25.000.  
		
		Zara khoảng 36.000.  
		
		SHEIN lên đến 1,3 triệu mặt hàng mới mỗi năm.  
	\end{vietnamese-v2}
	
	
	[playful music continues]
	I actually think that the numbers published online might be even low. \footnote{
		\textbf{might be even low:} "có thể còn thấp hơn nữa" – diễn tả sự nghi ngờ rằng các con số hiện có có thể chưa phản ánh đầy đủ hoặc chưa chính xác, thực tế có thể cao hơn.
	}
	
	\begin{vietnamese-v2}
		[nhạc vui nhộn tiếp tục]
		
		Tôi thực sự nghĩ rằng những con số được công bố trực tuyến có thể còn thấp hơn thực tế.
	\end{vietnamese-v2}

	[chatter]
	Actually just placed the order last week.
	
	- Probably once a month. 
	
	- At least once a month.
	
	I order sometimes twice a month.
	
	\begin{vietnamese-v2}
		[tiếng trò chuyện]
		
		Thực ra tôi vừa đặt hàng tuần trước.
		
		Có lẽ khoảng một lần mỗi tháng.
		
		Ít nhất một lần mỗi tháng.
		
		Đôi khi tôi đặt hàng hai lần một tháng.
	\end{vietnamese-v2}
	
	Where do I start?
	
	[Roger] We produce 100 billion pieces of garment \footnote{
		\textbf{100 billion pieces of garment:} "100 tỷ sản phẩm may mặc" – số lượng rất lớn các mặt hàng quần áo được sản xuất mỗi năm.
	
		\textbf{Garment} là danh từ chỉ bất kỳ loại trang phục, quần áo nào mà con người mặc lên người, bao gồm áo, quần, váy, áo khoác, v.v.
	} a year as an industry.
	
	A hundred billion.
	
	This is absolutely stunning.
	
	How many people are in the world?
	
	How many pieces per person? People don't need that much clothes.
	
	\begin{vietnamese-v2}
		Đâu là điểm khởi đầu? 
		[Roger] Chúng ta sản xuất 100 tỷ bộ quần áo mỗi năm trong ngành công nghiệp này. 
		Một trăm tỷ. 
		Thật sự kinh ngạc. 
		Có bao nhiêu người trên thế giới? Bao nhiêu bộ quần áo cho mỗi người? Con người không cần nhiều quần áo đến vậy.
	\end{vietnamese-v2}

	[Eric] I was ambitious,	competitive, uh, disciplined, hardworking, um, loud.
	
	\begin{vietnamese-v2}
		[Eric] Tôi từng tham vọng, cạnh tranh, ừm, kỷ luật, chăm chỉ, ừm, ồn ào.
	\end{vietnamese-v2}

	[violin plays single note]
	It was career success in... in, um, acceleration. So when I started to learn the impact I was having at Adidas, you can't unsee and unhear that.
	Y-y-you... You can only sing yourself to sleep so often.
	
	\begin{vietnamese-v2}
		[violin chơi một nốt đơn] 
		
		Đó là thành công trong sự nghiệp... trong, ừm, sự tăng tốc. 
		
		Vì vậy, khi tôi bắt đầu nhận ra tác động mà tôi đã có tại Adidas, bạn không thể không thấy và không nghe điều đó.
		
		B-b-bạn... Bạn chỉ có thể hát ru mình để ngủ bao nhiêu lần đây.
	\end{vietnamese-v2}
	
	\pagebreak
	
	[melodic bleeping]
	[Sasha] Hello... again.
	Many individuals will become disillusioned \footnote{
		\textbf{disillusioned:} Sự mất ảo tưởng, sự vỡ mộng, cảm giác thất vọng khi nhận ra sự thật không như mong đợi.
	
	} on the way to the top. Do not become one of them.
	
	\begin{vietnamese-v2}
		[tiếng bíp du dương] [Sasha] Xin chào... lại nhé. 
		Nhiều người sẽ trở nên vỡ mộng trên con đường lên đến đỉnh cao. Đừng trở thành một trong số họ.
	\end{vietnamese-v2}

	[playful music plays]
	Please now enjoy 19.2 seconds of adorable \footnote{
		\textbf{"Adorable"} là một tính từ trong tiếng Anh, có nghĩa là đáng yêu, dễ thương, thường dùng để mô tả người, động vật hoặc vật gì đó khiến người khác cảm thấy thích thú, yêu mến vì vẻ ngoài hoặc tính cách dễ mến.
	
	} cat and duck videos.
	
	\begin{vietnamese-v2}
		[nhạc vui nhộn vang lên]
		Hãy tận hưởng 19.2 giây video đáng yêu về mèo và vịt nào!
	\end{vietnamese-v2}
	
	[man] Oh my gosh!
	[Sasha] Remember, learning the art of distraction \footnote{
		\textbf{the art of distraction:} "nghệ thuật gây xao nhãng" hoặc "kỹ năng phân tâm".
		
		\textbf{"Distraction"} thường có nghĩa là sự xao nhãng, mất tập trung, nhưng trong ngữ cảnh này có thể hiểu là khả năng kiểm soát hoặc sử dụng xao nhãng một cách có chủ đích.
	
	} will be crucial \footnote{
		\textbf{Crucial:} rất quan trọng, thiết yếu, quyết định.
	
	} to achieving success \footnote{
		\textbf{To achieving success:} để đạt được thành công.
	
	}.
	
	\begin{vietnamese-v2}
		[người đàn ông] Ôi trời ơi! 
		[Sasha] Hãy nhớ rằng, học cách đánh lạc hướng sẽ là chìa khóa để đạt được thành công.
	\end{vietnamese-v2}
	
	
	[street bustle]
	Cut to New York. Reveal \footnote{
		\textbf{Reveal:} tiết lộ, phơi bày, làm rõ.
	
	} core messages \footnote{
		\textbf{Core messages:} thông điệp cốt lõi, những ý chính, giá trị trung tâm mà thương hiệu muốn truyền tải.
	
	} behind adverts. \footnote{
		\textbf{Behind adverts:} đằng sau các quảng cáo, tức là những thông điệp sâu sắc, ý nghĩa thực sự nằm ẩn trong các chiến dịch quảng cáo.
	
	}
	
	\begin{vietnamese-v2}
		[tiếng phố xá nhộn nhịp] 
		Chuyển cảnh đến New York. Tiết lộ thông điệp cốt lõi đằng sau các quảng cáo.
	\end{vietnamese-v2}
	
	[pensive music plays]
	Clear communication with consumers is an essential part of selling more \footnote{
		\textbf{an essential part of selling more} là một phần thiết yếu để tăng doanh số bán hàng.
		
		\textbf{"Essential"} là một tính từ trong tiếng Anh, mang nghĩa cần thiết, thiết yếu, cốt yếu, chủ yếu 
	}. But, to achieve profit maximization, you need to ensure that once they desire \footnote{
		\textbf{"Desire"} là một từ tiếng Anh có thể là danh từ hoặc động từ, mang nghĩa liên quan đến mong muốn, khao khát.
	
	} products, they can acquire \footnote{
		\textbf{"Acquire"} là một động từ trong tiếng Anh, có nghĩa là đạt được, giành được, thu được, tiếp nhận một thứ gì đó thông qua nỗ lực, mua bán, học hỏi hoặc trải nghiệm.
	
	} them as quickly as possible.
	
	Every second counts.
	
	Never limit your vision.
	
	If you can imagine it, you can create it.
	
	\begin{vietnamese-v2}
		[nhạc trầm tư vang lên] Giao tiếp rõ ràng với người tiêu dùng là một phần thiết yếu để bán được nhiều hơn. Nhưng, để đạt được tối đa hóa lợi nhuận, bạn cần đảm bảo rằng một khi họ khao khát sản phẩm, họ có thể sở hữu nó nhanh nhất có thể. 
		
		Mỗi giây đều quan trọng. 
	
		Đừng giới hạn tầm nhìn của bạn. 
	
		Nếu bạn có thể tưởng tượng nó, bạn có thể tạo ra nó.
	\end{vietnamese-v2}
	
	
	♪ I didn't have to wrap your gift I didn't leave the house ♪
	
	♪ In fact I never saw your gift I did it all by mouse ♪
	
	♪ I didn't venture \footnote{
		\textbf{Venture:} Liều làm, mạo hiểm, đánh bạo làm hoặc nói điều gì đó có thể mang rủi ro hoặc táo bạo.
	
	
	} to the mall I didn't waste a day ♪
	
	♪ Thanks to Amazon dot com ♪
	
	♪ Amazon dot com ♪
	
	♪ Amazon dot com... ♪
	
	\begin{vietnamese-v2}
		♪ Tôi không phải gói quà cho bạn Tôi không ra khỏi nhà ♪
		
		♪ Thực ra tôi chưa bao giờ nhìn thấy quà của bạn Tôi đã làm tất cả bằng chuột ♪
		
		♪ Tôi không mạo hiểm đến trung tâm thương mại Tôi không lãng phí một ngày nào ♪
		
		♪ Cảm ơn Amazon dot com ♪
		
		♪ Amazon dot com ♪
		
		♪ Amazon dot com... ♪
	\end{vietnamese-v2}
	
	I started there so early when it felt impossible to get people to buy anything online, but it seemed like a fun challenge to even think that you would ever buy one pair of jeans online. [laughs]		
	
	\begin{vietnamese-v2}
		Tôi đã bắt đầu từ rất sớm, khi việc khiến mọi người mua hàng trực tuyến gần như là điều không tưởng. 
		Nhưng đó lại là một thử thách thú vị—thậm chí chỉ nghĩ đến việc ai đó sẽ mua một chiếc quần jean trực tuyến cũng đã đủ hấp dẫn rồi. [cười]
	\end{vietnamese-v2}
	
	[beeps, clicks]

	I thought, "I'll go for two years and learn everything and get out." But it was surprisingly exciting.\footnote{
		\textbf{Surprisingly} là trạng từ, có nghĩa là "một cách ngạc nhiên", dùng để nhấn mạnh rằng điều gì đó xảy ra vượt ngoài sự mong đợi hoặc gây bất ngờ.
	
		\textbf{Exciting} là tính từ mang hậu tố “-ing”, dùng để mô tả một sự vật, sự việc hoặc tình huống gây ra cảm giác hứng thú, phấn khích cho người khác. Ví dụ: "The movie is exciting" (Bộ phim rất thú vị)
	}

	It... It was fast-paced and high tolerance for risk \footnote{
		\textbf{Fast-paced} mô tả một môi trường hoặc hoạt động diễn ra với tốc độ nhanh, nhịp độ cao, đòi hỏi sự phản ứng nhanh và thích nghi kịp thời với các thay đổi liên tục. 
		
		\textbf{High tolerance for risk} (độ chịu rủi ro cao) chỉ mức độ sẵn sàng chấp nhận rủi ro lớn. 
	}. It's like, "Yeah, let's do this."

	\begin{vietnamese-v2}
		[tiếng bíp, tiếng nhấp chuột] 
		
		Tôi đã nghĩ: "Mình sẽ dành hai năm để học mọi thứ rồi rời đi." 
		
		Nhưng hóa ra, nó lại đầy hứng khởi. 
		
		Nó... Nó diễn ra rất nhanh, và có sự chấp nhận rủi ro cao. Kiểu như, "Ừ, làm thôi!"
	\end{vietnamese-v2}

	You have an idea, it's like, "Here's a team, here's some money, go do it."

	Like, you were inventing. I have patents, you know, it's... like, it was fun.

	I would be in meetings with Jeff Bezos with me and four other people.

	Being able to actually talk to him and disagree with him.

	\begin{vietnamese-v2}
		Bạn có một ý tưởng, kiểu như: "Đây là một nhóm, đây là một khoản tiền, hãy thực hiện nó." 
		
		Kiểu như, bạn đang sáng tạo. 
		
		Tôi có bằng sáng chế, bạn biết đấy, nó... thật sự rất thú vị. 
		
		Tôi đã tham gia các cuộc họp với Jeff Bezos, chỉ có tôi và bốn người khác. Được thực sự trò chuyện với ông ấy và thậm chí bất đồng quan điểm.
	\end{vietnamese-v2}

	[Jeff] This may look like an ordinary cowboy hat, but it's actually an OSHA approved hard hat. \footnote{
		\textbf{Hard hat} là mũ bảo hộ cứng được làm từ vật liệu chắc chắn, dùng để bảo vệ đầu người lao động khỏi các chấn thương do vật rơi hoặc va đập, thường được sử dụng trong các công trường xây dựng và các ngành nghề liên quan đến an toàn lao động
	
	}
	We have the world's best hand drill.
	
	\begin{vietnamese-v2}
		[Jeff] Trông có vẻ như một chiếc mũ cao bồi bình thường, nhưng thực ra nó là một mũ bảo hộ đạt chuẩn OSHA. Chúng tôi có chiếc máy khoan cầm tay tốt nhất thế giới.	
	\end{vietnamese-v2}
	
	[drills whir]
	[Maren] Jeff has no patience for anyone who isn't as smart as he is.
	And I would see him just cut somebody off at the knees, like, "You are not smart. You should not be speaking."
	
	\begin{vietnamese-v2}
		[tiếng khoan rít lên] [Maren] Jeff không có kiên nhẫn với bất kỳ ai không thông minh như ông ấy. 
		Và tôi đã thấy ông ấy thẳng thừng ngắt lời ai đó, kiểu như: "Bạn không thông minh. Bạn không nên nói."
	\end{vietnamese-v2}
	
	[intriguing music plays]
	I think Jeff saw, initially \footnote{
		\textbf{"Initially"} là một trạng từ trong tiếng Anh, có nghĩa là "ban đầu", "lúc đầu" hoặc "đầu tiên"
	
	}, that people would buy books and music and movies and DVDs.
	
	But that the real money would come if you could sell apparel \footnote{
		\textbf{apparel:} danh từ, nghĩa là "quần áo" hoặc "trang phục".
	
	} and food online.
	
	And I was the designer that launched the beauty store, launched the jewelry store \footnote{
		\textbf{jewelry store} (tiếng Việt: cửa hàng trang sức) là một cơ sở kinh doanh bán lẻ chuyên về các sản phẩm trang sức như nhẫn, vòng cổ, hoa tai, vàng bạc, đá quý và đồng hồ đeo tay
	
	}, launched the apparel store, and it was a totally different way to shop.
	
	\begin{vietnamese-v2}
		[nhạc gây tò mò vang lên] Tôi nghĩ ban đầu Jeff thấy rằng mọi người sẽ mua sách, nhạc, phim và DVD trực tuyến. 
		
		Nhưng số tiền thực sự sẽ đến nếu bạn có thể bán quần áo và thực phẩm online. 
		
		Và tôi là nhà thiết kế đã ra mắt cửa hàng mỹ phẩm, cửa hàng trang sức, cửa hàng thời trang—đó là một cách mua sắm hoàn toàn khác.
	\end{vietnamese-v2}
	
	[clicks]
	[upbeat music plays]
	[click]
	
	We would actually say things like, "We want to be there when you have your next shoppable moment."
	
	Like, you're sitting in bed and you think, "I should buy a new carrot peeler."
	
	"I need a new pillowcase."
	
	"This one is scratchy." You know? Whatever.
	
	\begin{vietnamese-v2}
		[tiếng nhấp chuột] 
		
		[nhạc sôi động vang lên] 
		
		[nhấp chuột] 
		
		Chúng tôi thực sự đã nói những điều như: "Chúng tôi muốn có mặt khi bạn có khoảnh khắc mua sắm tiếp theo." 
		
		Kiểu như, bạn đang ngồi trên giường và nghĩ: "Mình nên mua một cái dao bào cà rốt mới." 
		
		"Mình cần một vỏ gối mới." 
		
		"Cái này gây khó chịu." Bạn biết đấy? Đại loại vậy.
	\end{vietnamese-v2}
	
	Wherever you are, whatever you're doing, if a thought occurs to you that you need something, we wanted Amazon to be the thing that occurred to you.
	
	We would say, "If the system was magic, what would the system do?".
	
	If the system was magic, what would it do? It would be like, there's just a conveyor belt that goes straight from wherever the item is to your door as quickly and frictionlessly as possible.
	
	
	\begin{vietnamese-v2}
		Bất kể bạn đang ở đâu, đang làm gì, nếu bạn chợt nghĩ rằng mình cần thứ gì đó, chúng tôi muốn Amazon là điều đầu tiên xuất hiện trong tâm trí bạn. 
		
		Chúng tôi sẽ nói: "Nếu hệ thống là phép thuật, nó sẽ làm gì?" 
		
		Nếu hệ thống là phép thuật, nó sẽ hoạt động như thế nào? Nó sẽ giống như có một băng chuyền chạy thẳng từ nơi món hàng đang nằm đến tận cửa nhà bạn, nhanh chóng và trơn tru nhất có thể.
	\end{vietnamese-v2}
	
	[muffled groans]
	It's too easy. And that was... That was the point, was to... reduce your time to think a little bit more critically\footnote{
		Từ "critically" là trạng từ trong tiếng Anh có nghĩa là một cách nghiêm trọng, nguy kịch hoặc một cách quan trọng, quyết định.
	
	} about a purchase that you thought you wanted to make.
	
	I mean, it's really a science.
	
	We were constantly developing new ways to get you to buy.
	
	
	\begin{vietnamese-v2}
		[tiếng rên rỉ nhỏ] 
		
		Nó quá dễ dàng. Và đó... Đó chính là mục tiêu, là... giảm thời gian để bạn suy nghĩ kỹ hơn về một món hàng mà bạn nghĩ mình muốn mua. 
		
		Ý tôi là, đó thực sự là một khoa học. 
		
		Chúng tôi liên tục phát triển những cách mới để khiến bạn mua hàng.
	\end{vietnamese-v2}
	
	[woman] This is everything I bought at 2 a.m. last night.
	
	\begin{vietnamese-v2}
		[người phụ nữ] Đây là tất cả những gì tôi đã mua lúc 2 giờ sáng hôm qua.
	\end{vietnamese-v2}
	
	[Maren] Influencing \footnote{
		Influencing là động từ (đang ở dạng V-ing, dùng như động từ hoặc danh động từ), mang ý nghĩa ảnh hưởng, tác động đến ai đó hoặc điều gì đó.
	
	} your behavior \footnote{
		behavior là danh từ mang ý nghĩa hành vi, cách cư xử, cách ứng xử của một người hoặc sinh vật.
	
	} in subtle \footnote{
		subtle là tính từ mang ý nghĩa tinh tế, khéo léo, khó nhận ra hoặc không rõ ràng.
	
	} ways that you'd never even realize. We could have, you know, a certain sentence \footnote{
		\textbf{"Certain sentence"} thường có nghĩa là \textbf{"một câu cụ thể"} hoặc \textbf{"một bản án nhất định"}, tùy thuộc vào ngữ cảnh sử dụng. 
	
	} that says, "Free shipping if you purchase \$25 or more." In one case you make the "\$25" orange.

	\begin{vietnamese-v2}
		[Maren] Tác động đến hành vi của bạn theo những cách tinh tế mà bạn thậm chí không nhận ra. Chúng tôi có thể có, bạn biết đấy, một câu nói như: "Miễn phí vận chuyển nếu bạn mua từ 25 đô la trở lên." Trong một trường hợp, chúng tôi làm cho "25 đô la" có màu cam.
	\end{vietnamese-v2}

	[chimes]
	In another one you make it green. And you can have nine different things that you're testing against each other.
	
	\begin{vietnamese-v2}
		[tiếng chuông vang lên] 
		Trong một trường hợp khác, bạn làm cho nó màu xanh lá cây. Và bạn có thể có chín yếu tố khác nhau để thử nghiệm so sánh với nhau.
	\end{vietnamese-v2}
	
	[chiming]
	
	And there's enough traffic to the site that you could get statistically relevant data \footnote{
		\textbf{statistically relevant data} (dữ liệu có ý nghĩa thống kê) — tức là dữ liệu đủ lớn, đủ chính xác để phân tích có giá trị về mặt thống kê.
	
	} on which version of that sentence makes the most money.
	
	Every pixel on that page was tested and optimized and optimized and optimized.
	
	\begin{vietnamese-v2}
		[tiếng chuông vang lên] 
		Có đủ lượng truy cập vào trang web để thu thập dữ liệu thống kê về phiên bản nào của câu đó mang lại doanh thu cao nhất. 
		
		Mỗi pixel trên trang đó đều đã được thử nghiệm, tối ưu hóa, rồi lại tối ưu hóa, và tối ưu hóa thêm nữa.
	\end{vietnamese-v2}
	
	[laughs] And Amazon had this whole system designed around tailoring \footnote{
		\textbf{Tailoring} là danh động từ của "tailor", nghĩa là "tùy chỉnh", "điều chỉnh phù hợp"
	
	} the site to be exactly the right combination of elements and colors and products that would most likely make you buy something.
	
	\begin{vietnamese-v2}
		[cười] Và Amazon có cả một hệ thống được thiết kế để điều chỉnh trang web sao cho có sự kết hợp chính xác giữa các yếu tố, màu sắc và sản phẩm—những thứ có khả năng cao nhất khiến bạn muốn mua hàng.
	\end{vietnamese-v2}
	
	
	If you think of just the things that, you know, one person buys, and then multiply that by... [chuckles] by how many people are now shopping on places like Amazon.
	
	\begin{vietnamese-v2}
		Nếu bạn nghĩ về những thứ mà một người mua, rồi nhân lên... [cười khúc khích] với số lượng người hiện đang mua sắm trên các nền tảng như Amazon.
	\end{vietnamese-v2}
	
	[orchestral crescendo]
	[conveyor belts rumble]
	You know, I felt like I was making shopping better and making it easier to find a delightful \footnote{
		\textbf{"Delightful"} là một tính từ trong tiếng Anh, có nghĩa là: vui thích, làm cho vui sướng, làm hài lòng, dễ chịu, thú vị.
	
	} item.
	I wasn't thinking about the consequences \footnote{
		\textbf{"Consequences"} là danh từ số nhiều của từ "consequence" trong tiếng Anh, có nghĩa là: \textbf{Hậu quả, kết quả,} hệ quả của một hành động, sự kiện hoặc quyết định nào đó.
	
	} of that as it becomes a more and more efficient \footnote{
		\textbf{"Efficient"} là tính từ trong tiếng Anh, mang nghĩa "có năng suất cao", "hiệu quả", tức là làm việc hoặc hoạt động một cách nhanh chóng, có tổ chức, đạt được kết quả tối đa với mức tiêu hao thời gian, công sức, chi phí tối thiểu
	
	} engine.
	
	I don't think we were ever thinking about where does all this stuff go.
	
	\begin{vietnamese-v2}
		[dàn nhạc cao trào] 
		[băng chuyền ầm ầm chuyển động] 
		
		Bạn biết đấy, tôi cảm thấy như mình đang làm cho việc mua sắm trở nên tốt hơn và giúp việc tìm thấy một món đồ thú vị dễ dàng hơn. 
		
		Tôi đã không nghĩ về hậu quả của điều đó khi hệ thống ngày càng trở thành một cỗ máy hiệu quả hơn. 
		
		Tôi không nghĩ rằng chúng tôi đã từng tự hỏi tất cả những thứ này sẽ đi đâu.
	\end{vietnamese-v2}
	
	[wind whooshes]
	[Sasha] Cut to plastic bag blowing in wind.
	Remember, to sell more, you must produce more.
	Lots more.
	I thought it would be fun at this point for you to see some specifics.
	
	\begin{vietnamese-v2}
		[gió rít qua] 
		[Sasha] Chuyển cảnh sang một chiếc túi nhựa bay trong gió. 
		Hãy nhớ rằng, để bán được nhiều hơn, bạn phải sản xuất nhiều hơn. 
		Rất nhiều hơn. 
		Tôi nghĩ sẽ thú vị vào lúc này nếu bạn thấy một số chi tiết cụ thể.
	\end{vietnamese-v2}
	
	[rumbling]
	Visualize total global production of goods needed to meet increased online sales.
	Visualize 2.5 million shoes produced each hour.
	
	\begin{vietnamese-v2}
		[tiếng ầm ầm] 
		Hình dung tổng sản lượng hàng hóa toàn cầu cần thiết để đáp ứng sự gia tăng của doanh số bán hàng trực tuyến. 
		Hình dung 2,5 triệu đôi giày được sản xuất mỗi giờ.
	\end{vietnamese-v2}
	
	[Mara] There used to be a stopgap\footnote{
		\textbf{Stopgap} là một danh từ tiếng Anh chỉ giải pháp tạm thời hoặc biện pháp lấp chỗ trống được sử dụng để giải quyết một vấn đề hoặc đáp ứng một nhu cầu cấp bách trong một khoảng thời gian ngắn cho đến khi có giải pháp lâu dài hơn.
	}.  I mean, think about it.
	If you have to get up, get in your car, drive to the store, look on the shelves, find the product.
	Then go home.
	That's a lot of work.
	But if all you have to do is push a button and it appears on your doorstep, of course you're gonna buy more products.
	
	\begin{vietnamese-v2}
		[Mara] Đã từng có một rào cản. Ý tôi là, hãy nghĩ về nó. 
		Nếu bạn phải đứng dậy, vào xe, lái đến cửa hàng, tìm kiếm trên kệ, chọn sản phẩm. 
		Rồi lại lái xe về nhà. Đó là rất nhiều công đoạn. 
		Nhưng nếu tất cả những gì bạn cần làm chỉ là nhấn một nút và sản phẩm xuất hiện ngay trước cửa nhà bạn, thì tất nhiên bạn sẽ mua nhiều hơn rồi.
	\end{vietnamese-v2}
	
	[upbeat music playing]
	[Sasha] Visualize 68,733 phones produced each hour.
	
	\begin{vietnamese-v2}
		[nhạc sôi động vang lên] 
		[Sasha] Hình dung 68.733 chiếc điện thoại được sản xuất mỗi giờ.
	\end{vietnamese-v2}
	
	[upbeat music continues]
	Visualize 190,000 garments produced each minute.
	Amazing. [echoing]
	
	\begin{vietnamese-v2}
		[nhạc sôi động tiếp tục vang lên] 
		Hình dung 190.000 bộ quần áo được sản xuất mỗi phút. 
		Thật đáng kinh ngạc. [vang vọng]
	\end{vietnamese-v2}
	
	Visualize 12 tons of plastic produced each second. I ran one of the biggest multinationals in the world called Unilever.
	Unilever is one of the biggest producers of household goods.
	
	\begin{vietnamese-v2}
		Hình dung 12 tấn nhựa được sản xuất mỗi giây. Tôi đã điều hành một trong những tập đoàn đa quốc gia lớn nhất thế giới, gọi là Unilever. 
		Unilever là một trong những nhà sản xuất hàng tiêu dùng lớn nhất.
	\end{vietnamese-v2}
	
	\pagebreak
	
	Think about shampoos, detergents \footnote{
		\textbf{Detergents} là gì? Detergent trong tiếng Anh có nghĩa là "chất tẩy rửa". Đây là các hóa chất hoặc dung dịch dùng để làm sạch, loại bỏ bụi bẩn, dầu mỡ và các vết bẩn khác trên quần áo, đồ vải và các vật dụng khác. Trong ngữ cảnh máy giặt, detergent thường là các sản phẩm tẩy rửa giặt giũ như bột giặt, nước giặt, nước xả vải... Đặc điểm của detergent là có thể là dạng bột hoặc lỏng và được đặt trong các ngăn riêng biệt trên máy giặt để phân phối đúng cách trong quá trình giặt.
	
	}. Reaching about three and a half billion consumers a day.
	
	\begin{vietnamese-v2}
		Hãy nghĩ về dầu gội, chất tẩy rửa. Tiếp cận khoảng ba tỷ rưỡi người tiêu dùng mỗi ngày.
	\end{vietnamese-v2}
	
	[upbeat music continues]
	I don't think the consumer is actually here the culprit.
	Of course, they consume, but why do they consume?
	Because they're encouraged to to a great extent.
	
	\begin{vietnamese-v2}
		[nhạc sôi động tiếp tục vang lên] Tôi không nghĩ rằng người tiêu dùng thực sự là thủ phạm trong trường hợp này. 
		Tất nhiên, họ tiêu dùng, nhưng tại sao họ lại tiêu dùng? 
		Bởi vì họ được khuyến khích ở mức độ rất lớn.
	\end{vietnamese-v2}
	
	When we throw it away, we actually don't throw it away.
	"Away" doesn't exist.
	It ends up somewhere else on this planet Earth.
	And it increasingly has consequences. \footnote{
		Từ \textbf{"consequences"} trong tiếng Việt có nghĩa là "hậu quả" hoặc "kết quả", thường dùng để chỉ kết quả hoặc tác động phát sinh từ một hành động hoặc sự kiện nào đó, nhiều khi mang nghĩa tiêu cực hoặc không mong muốn. 
	}
	
	\begin{vietnamese-v2}
		Khi chúng ta vứt bỏ một thứ gì đó, thực ra chúng ta không thực sự "vứt nó đi". 
		"Bỏ đi" không thực sự tồn tại. 
		Nó chỉ kết thúc ở một nơi khác trên hành tinh này. 
		Và hậu quả của điều đó ngày càng gia tăng.
	\end{vietnamese-v2}
	
	[playful harmony]
	[Sasha] Many congratulations on the completion of rule one.
	
	\begin{vietnamese-v2}
		[giai điệu vui tươi] 
		[Sasha] Xin chúc mừng vì đã hoàn thành quy tắc đầu tiên.
	\end{vietnamese-v2}
	
	[pulsing]
	[rings]
	You're doing really, really well and are now ready to take profit maximization to the next level.
	
	\begin{vietnamese-v2}
		[nhịp đập] 
		[tiếng chuông vang lên] 
		Bạn đang làm rất tốt và bây giờ đã sẵn sàng để đưa việc tối đa hóa lợi nhuận lên một tầm cao mới.
	\end{vietnamese-v2}
	
	Coming up, I will teach you how to control products after they are purchased.
	
	\begin{vietnamese-v2}
		Sắp tới, tôi sẽ hướng dẫn bạn cách kiểm soát sản phẩm sau khi chúng được mua.
	\end{vietnamese-v2}
	
	[playful harmony]
	And how to dictate when consumers replace old with new.
	
	\begin{vietnamese-v2}
		[giai điệu vui tươi] 
		Và cách định hướng thời điểm người tiêu dùng thay thế cái cũ bằng cái mới.
	\end{vietnamese-v2}
	
	[playful harmony]
	You will also be one step closer to unlocking your surprise.
	
	\begin{vietnamese-v2}
		[giai điệu vui tươi] 
		Bạn cũng sẽ tiến thêm một bước để mở khóa điều bất ngờ của mình.
	\end{vietnamese-v2}
	
	
	[flatly] Yay.
	Rule two.
	Waste more. [echoing]
	
	\begin{vietnamese-v2}
		[giọng đều đều] Yay. 
		Quy tắc thứ hai. 
		Lãng phí nhiều hơn. [vang vọng]
	\end{vietnamese-v2}
	
	[optimistic violin strings play]
	For a master class in creating waste for profit, you may find it beneficial to view additional material at this point.
	Imagine a light bulb on Broadway speaking with an old-timey voice.
	
	\begin{vietnamese-v2}
		[tiếng đàn violin lạc quan vang lên] 
		Để có một bài giảng chuyên sâu về việc tạo ra chất thải để thu lợi nhuận, bạn có thể thấy hữu ích khi xem thêm tài liệu vào thời điểm này. 
		Hãy tưởng tượng một bóng đèn trên Broadway đang nói chuyện với giọng cổ điển.
	\end{vietnamese-v2}
	
	[Lightbulb] Oh, hey there.
	They tell me I'm part of a very exclusive club, because I'm over 100 years old and still shining.
	There's not many of my kind still around. That's for sure.
	
	\begin{vietnamese-v2}
		[Bóng đèn] Ồ, chào bạn. 
		Họ nói rằng tôi thuộc một câu lạc bộ rất đặc biệt, vì tôi đã hơn 100 tuổi mà vẫn sáng rực rỡ. 
		Không còn nhiều người giống tôi nữa đâu. Đó là điều chắc chắn.
	\end{vietnamese-v2}
	
	Guess you could say we were just too good.
	Our world was turned upside down on January 15th, 1925.
	That was the fateful \footnote{
		Từ \textbf{"fateful"} trong tiếng Việt có nghĩa là "định mệnh" hoặc "đầy tai họa". Nó thường được dùng để mô tả một sự kiện hoặc quyết định có ảnh hưởng rất lớn, quyết định số phận hoặc mang lại hậu quả nghiêm trọng, không thể thay đổi.
	} day that the Phoebus cartel came into being.
	
	\begin{vietnamese-v2}
		Bạn có thể nói rằng chúng tôi đã quá tốt. 
		Thế giới của chúng tôi đã bị đảo lộn vào ngày 15 tháng 1 năm 1925. 
		Đó là ngày định mệnh mà tập đoàn Phoebus ra đời.
	\end{vietnamese-v2}
	
	[sci-fi whirring]
	[Sasha, echoing] Rewinding time.
	Going back...
	[voice rewinding]
	[electricity buzzing]
	[Lightbulb] This group of senior executives \footnote{
		\textbf{Senior executives (giám đốc điều hành cấp cao)} là những người quản lý cấp cao trong doanh nghiệp, đảm nhiệm vai trò quan trọng trong hệ thống lãnh đạo tổ chức. Họ thường chịu trách nhiệm quản lý các hoạt động chiến lược của công ty, đưa ra các quyết định quan trọng và định hướng phát triển cho toàn bộ tổ chức.
	
	} from the major light bulb manufacturers conspired \footnote{
		Từ "conspired" là dạng quá khứ đơn và quá khứ phân từ của động từ "conspire". Nghĩa của "conspire" trong tiếng Việt là "âm mưu," tức là hợp tác hoặc phối hợp một cách bí mật với người khác để thực hiện một hành động bất hợp pháp hoặc có tính chất gian lận.
	
	
	} together to cut short our lives and maximize their profits.
	
	\begin{vietnamese-v2}
		[tiếng rít khoa học viễn tưởng] [Sasha, vang vọng] Quay ngược thời gian. 
		Trở lại... [giọng nói tua ngược] 
		[tiếng điện kêu rè rè] [Bóng đèn] Nhóm các giám đốc điều hành cấp cao từ những nhà sản xuất bóng đèn lớn đã cùng nhau âm mưu rút ngắn tuổi thọ của chúng tôi để tối đa hóa lợi nhuận của họ.
	\end{vietnamese-v2}
	
	[foreboding violin notes play]
	To encourage frequent replacement, every bulb was reduced from 2,500 hours to just 1,000 hours of light.
	Creating mountains of unnecessary waste.
	
	\begin{vietnamese-v2}
		[tiếng đàn violin đầy dự cảm vang lên] Để thúc đẩy việc thay thế thường xuyên, mỗi bóng đèn đã bị giảm tuổi thọ từ 2.500 giờ xuống chỉ còn 1.000 giờ ánh sáng. Tạo ra những núi rác thải không cần thiết.
	\end{vietnamese-v2}
	
	[buzz, crack]
	[Sasha] Glorious waste. [echoing]
	
	\begin{vietnamese-v2}
		[tiếng rít, tiếng nứt vỡ] [Sasha] Chất thải vinh quang. [vang vọng]
	\end{vietnamese-v2}
	
	[Lightbulb] They called it "planned obsolescence."
	Now it's not so much of a conspiracy.
	Making products designed to break or be rapidly discarded has become commonplace in almost every industry.
	Everything, from clothing that lasts a few washes, to electric toothbrushes with batteries you can't replace.
	To printers that intentionally stop working even when there's ink in them.
	
	\begin{vietnamese-v2}
		[Bóng đèn] Họ gọi đó là "sự lỗi thời có kế hoạch." 
		Bây giờ nó không còn đơn thuần là một âm mưu nữa. 
		Việc chế tạo sản phẩm với mục đích hỏng hóc hoặc bị loại bỏ nhanh chóng đã trở thành điều phổ biến trong hầu hết mọi ngành công nghiệp. 
		Từ quần áo chỉ dùng được vài lần giặt, đến bàn chải điện với pin không thể thay thế. 
		Đến cả máy in được thiết kế để ngừng hoạt động dù vẫn còn mực.
	\end{vietnamese-v2}
	
	[dog barks]
	[woman] Everything I print, it doesn't print correctly.
	I have to get a new printer. That one's at the end of its lifespan.
	But it's only three years old!
	
	\begin{vietnamese-v2}
		[tiếng chó sủa] 
		[người phụ nữ] Mọi thứ tôi in ra đều không đúng. 
		Tôi phải mua một chiếc máy in mới. Chiếc này đã đến cuối vòng đời của nó. 
		Nhưng nó mới chỉ ba năm tuổi thôi!
	\end{vietnamese-v2}
	
	[Lightbulb] Today, planned obsolescence has become a cornerstone of successful business the world over.
	
	\begin{vietnamese-v2}
		[Bóng đèn] Ngày nay, sự lỗi thời có kế hoạch đã trở thành nền tảng của những doanh nghiệp thành công trên toàn thế giới.
	\end{vietnamese-v2}
	
	[soft music plays]
	[male voice over] We have created a product that is the most deliberate evolution of our original founding design.
	
	\begin{vietnamese-v2}
		[nhạc nhẹ nhàng vang lên] 
		[giọng nam thuyết minh] Chúng tôi đã tạo ra một sản phẩm mang tính tiến hóa có chủ đích nhất từ thiết kế sáng lập ban đầu của mình.
	\end{vietnamese-v2}
	
	[Nirav] Apple had this idea that our products are perfect, they're perfect from day one.
	
	\begin{vietnamese-v2}
		[Nirav] Apple đã có ý tưởng rằng sản phẩm của họ là hoàn hảo, hoàn hảo ngay từ ngày đầu tiên.
	\end{vietnamese-v2}
	
	[male voice over] Removing imperfections, establishing a seamlessness between materials, and producing a pristine, mirrorlike surface.

	\begin{vietnamese-v2}
		[giọng nam thuyết minh] Loại bỏ những điểm không hoàn hảo, thiết lập sự liền mạch giữa các vật liệu, và tạo ra một bề mặt nguyên sơ, phản chiếu như gương.
	\end{vietnamese-v2}

	The pinnacle and the ideal is to have this seamless, sealed up object.
	We don't want you to use that Apple product from a few years ago that you bought, we want you to pick up the new, perfect object and use that one.

	\begin{vietnamese-v2}
		Đỉnh cao và lý tưởng là có một vật thể liền mạch, được niêm phong hoàn hảo. 
		Chúng tôi không muốn bạn tiếp tục sử dụng sản phẩm Apple từ vài năm trước mà bạn đã mua, chúng tôi muốn bạn chọn lấy đối tượng mới, hoàn hảo và sử dụng nó.
	\end{vietnamese-v2}

	One in, one out.
	I have been in consumer electronics for a bit over a decade now.
	I started at Apple, joined the founding team of Oculus, led the hardware organization there.
	
	\begin{vietnamese-v2}
		Một vào, một ra. 
		Tôi đã làm việc trong lĩnh vực điện tử tiêu dùng hơn một thập kỷ. 
		Tôi bắt đầu tại Apple, tham gia đội ngũ sáng lập của Oculus, và lãnh đạo tổ chức phần cứng tại đó.
	\end{vietnamese-v2}
	
	
	[upbeat electronic music plays]
	So, joining Apple, this was back in 2009.
	After living at my parents' basement for a couple of months, I actually heard back from Apple, and I joined in and started writing software for what eventually became FaceTime.
	
	\begin{vietnamese-v2}
		[nhạc điện tử sôi động vang lên] 
		Vậy là khi gia nhập Apple, điều đó diễn ra vào năm 2009. 
		Sau vài tháng sống trong tầng hầm nhà bố mẹ, tôi nhận được phản hồi từ Apple, gia nhập công ty và bắt đầu viết phần mềm cho thứ cuối cùng đã trở thành FaceTime.
	\end{vietnamese-v2}
	
	[man] It's going to change the way we communicate forever.
	
	\begin{vietnamese-v2}
		[người đàn ông] Điều này sẽ thay đổi cách chúng ta giao tiếp mãi mãi.
	\end{vietnamese-v2}
	
	[Nirav] You saw kind of hints of the magic even buried away as, you know, a little, you know, straight out of school new grad in an engineering team, but there was really genuinely, and still is, I think, a focus on delivering a product that is a great user experience.
	Even if some of the side effects of that sometimes end up being pretty negative.
	
	\begin{vietnamese-v2}
		[Nirav] Bạn có thể cảm nhận được phần nào điều kỳ diệu, ngay cả khi nó bị giấu kín, giống như một tân cử nhân mới tốt nghiệp, bước vào một đội ngũ kỹ sư. Nhưng thực sự có—và vẫn còn—một sự tập trung vào việc mang lại một trải nghiệm sản phẩm tuyệt vời. Ngay cả khi một số tác dụng phụ của điều đó đôi khi lại khá tiêu cực.
	\end{vietnamese-v2}
	
	
	[crowd cheers]
	[Steve] And we are calling it iPhone.
	[loud cheers]
	Today...
	today Apple is going to reinvent the phone.
	It's kind of crazy, but in these companies the focus is on the launch, ultimately, it's on this moment.
	
	\begin{vietnamese-v2}
		[tiếng reo hò của đám đông]  
		[Steve] Và chúng tôi gọi nó là iPhone.  
		[tiếng hoan hô lớn]  
		Hôm nay...  
		hôm nay Apple sẽ tái định nghĩa chiếc điện thoại.  
		Thật điên rồ, nhưng tại những công ty này, trọng tâm thực sự là ở sự ra mắt, cuối cùng, là ở khoảnh khắc này.
	\end{vietnamese-v2}
	
	You've seen it over and over again with Apple with that grand keynote.
	We're gonna take it to the next level.
	And today we're introducing the iPhone 3G.
	
	\begin{vietnamese-v2}
		Bạn đã thấy điều này lặp đi lặp lại với Apple qua những bài phát biểu hoành tráng.  
		Chúng tôi sẽ đưa nó lên một tầm cao mới.  
		Và hôm nay, chúng tôi giới thiệu iPhone 3G.
	\end{vietnamese-v2}

	[cheers]
	We need to sell iPhone in more countries. And for every iPhone that followed, we've built on the vision of the original iPhone.
	
	\begin{vietnamese-v2}
		[tiếng hoan hô]  
		Chúng ta cần bán iPhone ở nhiều quốc gia hơn. Và với mỗi chiếc iPhone tiếp theo, chúng tôi đã xây dựng dựa trên tầm nhìn của chiếc iPhone nguyên bản.
	\end{vietnamese-v2}
	
	[Nirav] And because Apple has been so, so successful with this, every other consumer electronics company in the world semulates that.
	The Galaxy Note 8.
	
	\begin{vietnamese-v2}
		[Nirav] Và vì Apple đã quá thành công với điều này, mọi công ty điện tử tiêu dùng khác trên thế giới đều bắt chước theo.  
		Galaxy Note 8.
	\end{vietnamese-v2}
	
	
	[cheers]
	[Nirav] If you're making notebooks or smartphones, where essentially all consumers already have one, your business model depends on those consumers needing to replace the ones that they already have.
	
	\begin{vietnamese-v2}
		[tiếng hoan hô] [Nirav] Nếu bạn sản xuất máy tính xách tay hoặc điện thoại thông minh—những thứ mà hầu hết người tiêu dùng đã sở hữu—mô hình kinh doanh của bạn phụ thuộc vào việc khiến họ phải thay thế những sản phẩm họ đã có.
	\end{vietnamese-v2}
	
	[crowd] Four, three, two, one!
	[cheering]
	[Nirav] It's really a pipeline. You design a product, you manufacture a product, you sell the product, you launch the product, it gets used, and then it turns into waste.
	
	\begin{vietnamese-v2}
		[đám đông] Bốn, ba, hai, một! [tiếng hoan hô] [Nirav] Đó thực sự là một quy trình. Bạn thiết kế sản phẩm, sản xuất sản phẩm, bán sản phẩm, tung ra sản phẩm, sản phẩm được sử dụng, và rồi nó trở thành rác thải.
	\end{vietnamese-v2}
	
	[clock ticks]
	There's something like 13 million phones thrown out every day, which is a crazy, mind-boggling number. And if you think about that in aggregate, it's that we're all replacing these things every two or three years.
	
	\begin{vietnamese-v2}
		[tiếng đồng hồ tích tắc] Có khoảng 13 triệu chiếc điện thoại bị vứt bỏ mỗi ngày—một con số điên rồ, khó tin. Và nếu bạn nhìn nhận tổng thể, thực tế là tất cả chúng ta đang thay thế những thiết bị này cứ mỗi hai hoặc ba năm.
	\end{vietnamese-v2}
	
	Even though they are incredibly advanced and expensive, and in some ways almost the pinnacle of our industrial capability as a civilization, they are basically throwaway objects.
	And so, I look at that and I think, that's really broken, that's broken across every possible way that you can look at it.
	
	\begin{vietnamese-v2}
		Dù chúng vô cùng tiên tiến và đắt đỏ, và theo một số cách nào đó gần như là đỉnh cao của khả năng công nghiệp của nền văn minh chúng ta, nhưng chúng lại trở thành những vật phẩm bị vứt bỏ. Và vì vậy, khi nhìn vào điều đó, tôi nghĩ rằng hệ thống này thực sự đã hỏng—hỏng theo mọi khía cạnh có thể nhìn nhận.
	\end{vietnamese-v2}
	
	[whooshing]
	[Sasha] Remember, releasing a continual stream of new products will encourage consumers to discard old ones.
	This is fundamental to growth. Unfortunately, some individuals will feel the desire to maintain or repair items you consider ready for replacement.
	
	\begin{vietnamese-v2}
		[tiếng vút qua] [Sasha] Nhớ rằng, việc liên tục tung ra những sản phẩm mới sẽ khuyến khích người tiêu dùng loại bỏ những sản phẩm cũ. Đây là yếu tố cơ bản của sự tăng trưởng. Đáng tiếc, một số cá nhân vẫn sẽ muốn giữ lại hoặc sửa chữa những món đồ mà bạn cho là đã sẵn sàng để thay thế.
	\end{vietnamese-v2}
	
	
	[foreboding music plays]
	These people should be actively discouraged wherever possible.
	
	\begin{vietnamese-v2}
		[nhạc đầy dự cảm vang lên] Những người này nên bị ngăn cản một cách tích cực ở bất cứ đâu có thể.
	\end{vietnamese-v2}
	
	[male reporter] Kyle Wiens is the founder and CEO of iFixit, a company that offers tools, parts, and repair manuals for thousands of gadgets.
	
	\begin{vietnamese-v2}
		[male reporter] Kyle Wiens là người sáng lập và CEO của iFixit, một công ty cung cấp dụng cụ, linh kiện và hướng dẫn sửa chữa cho hàng nghìn thiết bị.
	\end{vietnamese-v2}
	
	[Kyle] It used to be the case that repair was the default.
	You could get repair manuals, companies sold parts, and over the last 30 years that has systematically disappeared from our lives.
	
	\begin{vietnamese-v2}
		[Kyle] Trước đây, việc sửa chữa từng là điều mặc định. Bạn có thể tìm được hướng dẫn sửa chữa, các công ty bán linh kiện, nhưng trong 30 năm qua, điều đó đã dần biến mất khỏi cuộc sống của chúng ta.
	\end{vietnamese-v2}
	
	This is a disposable product.
	Use it for two years, throw it away, buy a new laptop.
	
	\begin{vietnamese-v2}
		Đây là một sản phẩm dùng một lần.
		Sử dụng trong hai năm, rồi vứt đi, mua một chiếc laptop mới.
	\end{vietnamese-v2}
	
	I don't like that.
	I had always assumed we're not fixing things because it's not possible.
	
	\begin{vietnamese-v2}
		Tôi không thích điều đó. Tôi đã luôn cho rằng chúng ta không sửa chữa đồ dùng vì điều đó là không thể.
	\end{vietnamese-v2}
	
	It's not economically viable.
	Actually, we're not fixing things because lawyers are going out of their way to censor that knowledge from the world.
	
	\begin{vietnamese-v2}
		Nó không khả thi về mặt kinh tế.
		Thực ra, chúng ta không sửa chữa mọi thứ vì các luật sư đang cố gắng kiểm duyệt kiến thức đó khỏi thế giới.
	\end{vietnamese-v2}
	
	We still get copyright takedown notices from companies trying to censor information from iFixit.
	I got one the other day.
	We told them to fuck off.
	It's not just that companies are taking away information about how to fix your stuff.
	Across-the-board, they're making the products themselves much more difficult to repair.
	
	\begin{vietnamese-v2}
		Chúng tôi vẫn nhận được thông báo gỡ bỏ bản quyền từ các công ty cố gắng kiểm duyệt thông tin từ iFixit.
		Tôi đã nhận được một thông báo như vậy vào ngày hôm kia.
		Chúng tôi đã bảo họ cút đi.
		Không chỉ các công ty đang lấy đi thông tin về cách sửa chữa đồ đạc của bạn.
		Trên mọi phương diện, họ đang khiến bản thân các sản phẩm trở nên khó sửa chữa hơn nhiều.
	\end{vietnamese-v2}
	
	[reporter] Apple doesn't want its customers to fix the new iPhone.
	They're making sure that back cover is tamperproof swapping out the more common Phillips screws, with so-called pentalobe screws.
	Why change the screws?
	
	\begin{vietnamese-v2}
		[phóng viên] Apple không muốn khách hàng của mình sửa iPhone mới.
		Họ đang đảm bảo rằng nắp lưng không bị giả mạo bằng cách thay thế các vít Phillips phổ biến hơn bằng các vít pentalobe.
		Tại sao phải thay vít?
	\end{vietnamese-v2}
	
	[Kyle] Apple would prefer that we not have access to our own hardware, which means that Apple's going to be selling more machines because people have to replace them frequently. It's their product, and you can't really begrudge them quality control.
	
	\begin{vietnamese-v2}
		[Kyle] Apple muốn chúng ta không được tiếp cận phần cứng của riêng mình, điều đó có nghĩa là Apple sẽ bán được nhiều máy hơn vì mọi người phải thay thế chúng thường xuyên. Đó là sản phẩm của họ và bạn không thể thực sự phàn nàn về việc kiểm soát chất lượng của họ.
	\end{vietnamese-v2}
	
	[Kyle] Honestly, it's getting worse. We're seeing more and more companies that are now actually gluing phones, tablets and laptops together, making repairs really difficult.
	
	\begin{vietnamese-v2}
		[Kyle] Thành thật mà nói, tình hình đang trở nên tệ hơn. Chúng ta thấy ngày càng nhiều công ty thực sự dán điện thoại, máy tính bảng và máy tính xách tay lại với nhau, khiến việc sửa chữa trở nên thực sự khó khăn.
	\end{vietnamese-v2}
	
	There's one category of product that we're very frustrated with right now, that unfortunately is massively popular, and that is AirPods and wireless earbuds in general.
	
	\begin{vietnamese-v2}
		Có một loại sản phẩm mà chúng tôi rất thất vọng hiện nay, nhưng thật không may là nó lại cực kỳ phổ biến, đó là AirPods và tai nghe không dây nói chung.
	\end{vietnamese-v2}
	
	Each earbud has a battery and then the charging case has a battery.
	There is no way, uh, to open this up and swap out the battery in any of them.
	The batteries will degrade after 18 months, a couple years, and then you're stuck and you have to go and buy new ones.
	
	\begin{vietnamese-v2}
		Mỗi tai nghe đều có pin và hộp sạc cũng có pin.
		Không có cách nào, uh, để mở hộp ra và thay pin trong bất kỳ hộp nào.
		Pin sẽ bị hỏng sau 18 tháng, vài năm, và sau đó bạn sẽ bị kẹt và phải đi mua pin mới.
	\end{vietnamese-v2}
	
	
	This is evil.
	I... I am a reasonable person that has been radicalized by an earbud, which is outrageous.
	
	\begin{vietnamese-v2}
		Điều này thật độc ác.
		Tôi... Tôi là một người bình thường nhưng đã bị cực đoan hóa bởi một chiếc tai nghe, điều này thật vô lý.
	\end{vietnamese-v2}
	
	
	[beeping]
	Apple pioneered these and... and mainstreamed it, and... and removed the headphone jack from their phone, so they forced everyone to buy these.
	And... And Apple has special, I think, responsibility because where they go the rest of the industry follow.
	The lies by omission in the products that we consume are just incredible.
	
	\begin{vietnamese-v2}
		[bíp]
		Apple là người tiên phong trong những sản phẩm này và... và đưa chúng vào sản phẩm chính thống, và... và loại bỏ giắc cắm tai nghe khỏi điện thoại của họ, vì vậy họ buộc mọi người phải mua những sản phẩm này.
		Và... Và tôi nghĩ Apple có trách nhiệm đặc biệt, vì họ đi đâu thì phần còn lại của ngành công nghiệp cũng đi theo.
		Những lời nói dối bị bỏ sót trong các sản phẩm mà chúng ta tiêu thụ thật không thể tin được.
	\end{vietnamese-v2}
	
	[soft music plays]
	We have had the wool pulled over our eyes by marketers at these tech companies.
	
	\begin{vietnamese-v2}
		[nhạc nhẹ phát]
		Chúng ta đã bị các nhà tiếp thị tại các công ty công nghệ này lừa gạt.
	\end{vietnamese-v2}
	
	[crunching]
	Just as we are on this treadmill of consumerism, they're on a treadmill of having to extract more and more profits from us.
	
	\begin{vietnamese-v2}
		[tiếng kêu răng rắc]
		Giống như chúng ta đang ở trên cỗ máy tiêu dùng này, họ cũng đang ở trên cỗ máy phải khai thác ngày càng nhiều lợi nhuận từ chúng ta.
	\end{vietnamese-v2}
	
	[cheering]
	[man] Apple today became the first company to trade with a one-trillion dollar market capitalization.
	
	\begin{vietnamese-v2}
		[hoan hô]
		[người đàn ông] Hôm nay, Apple đã trở thành công ty đầu tiên giao dịch với vốn hóa thị trường một nghìn tỷ đô la.
	\end{vietnamese-v2}
	
	
	[loud cheers]
	[Nirav] As soon as your business model starts to revolve around that replacement cycle, the object being replaced in whole instead of being something that can last longer, it becomes extremely difficult to then reverse and go back.
	
	\begin{vietnamese-v2}
		[tiếng reo hò lớn]
		[Nirav] Ngay khi mô hình kinh doanh của bạn bắt đầu xoay quanh chu kỳ thay thế đó, tức là vật thể được thay thế hoàn toàn thay vì là thứ có thể tồn tại lâu hơn, thì việc đảo ngược và quay lại trở nên cực kỳ khó khăn.
	\end{vietnamese-v2}
	
	
	[cheers]
	If you're that CEO, if you're that executive and you go to the board and say, "We're going to take our 50 billion in revenue we make every year, and turn it into 25 billion," they're going to show you the door and someone else is gonna take your seat.
	
	
	\begin{vietnamese-v2}
		[hoan hô]
		Nếu bạn là CEO, nếu bạn là giám đốc điều hành và bạn đến hội đồng quản trị và nói, "Chúng ta sẽ lấy 50 tỷ doanh thu kiếm được hàng năm và biến nó thành 25 tỷ", họ sẽ đuổi bạn ra khỏi công ty và người khác sẽ thay thế bạn.
	\end{vietnamese-v2}
	
	[explosion, crackling]
	[Sasha] Thank you for your continued engagement.
	[pulsing]
	I like you and your diligent approach to this interaction.
	
	\begin{vietnamese-v2}
		[bùng nổ, nổ lách tách]
		[Sasha] Cảm ơn bạn đã tiếp tục tham gia.
		[xung nhịp]
		Tôi thích bạn và cách tiếp cận cần cù của bạn đối với tương tác này.
	\end{vietnamese-v2}
	
	[foghorn blares]
	Managed correctly, consumers' waste can equal profit for your business.
	[foghorn blows]
	But, if you need to dispose of products directly, discretion will be key.
	
	\begin{vietnamese-v2}
		[foghorn blares]
		Được quản lý đúng cách, chất thải của người tiêu dùng có thể mang lại lợi nhuận cho doanh nghiệp của bạn.
		[foghorn blows]
		Nhưng nếu bạn cần xử lý sản phẩm trực tiếp, sự thận trọng sẽ là chìa khóa.
	\end{vietnamese-v2}
	
	
	[foghorn blows]
	[echoing] You do not want consumers to see what you're discarding. Ever.
	
	
	\begin{vietnamese-v2}
		[tiếng còi báo hiệu]
		[tiếng vọng] Bạn không muốn người tiêu dùng nhìn thấy những gì bạn đang vứt bỏ. Không bao giờ.
	\end{vietnamese-v2}
	
	[Anna] Look at this bag.
	Look, look, look.
	Bring in the camera to see.
	Oh my God. Oh.
	For the most part, waste is on the curb, and its public, and so you're able to see what exactly corporations are tossing.
	
	\begin{vietnamese-v2}
		[Anna] Nhìn cái túi này này.
		Nhìn này, nhìn này, nhìn này.
		Mang máy ảnh vào để xem nào.
		Ôi trời. Ồ.
		Phần lớn rác thải đều ở lề đường, và ở nơi công cộng, vì vậy bạn có thể thấy chính xác những gì các tập đoàn đang vứt đi.
	\end{vietnamese-v2}
	
	This is all chocolate.
	This is all chocolate.
	We see corporations throw out perfectly usable unsold products pretty much every day.
	
	\begin{vietnamese-v2}
		Đây toàn là sô cô la.
		Đây toàn là sô cô la.
		Chúng ta thấy các tập đoàn vứt bỏ những sản phẩm hoàn toàn có thể sử dụng được nhưng không bán được hầu như mỗi ngày.
	\end{vietnamese-v2}
	
	[playful music plays]
	But there's certain times a year when that really spikes.
	After every single holiday, there is a huge amount of corporate waste of the unsold merchandise.
	
	\begin{vietnamese-v2}
		[nhạc vui tươi phát]
		Nhưng có những thời điểm nhất định trong năm mà điều đó thực sự tăng đột biến.
		Sau mỗi kỳ nghỉ, có một lượng lớn hàng hóa không bán được bị lãng phí của công ty.
	\end{vietnamese-v2}
	
	[newscaster] It's a rush for Black Friday bargains over the weekend, as over 226 million people shopped in stores and online.
	
	\begin{vietnamese-v2}
		[Người dẫn chương trình] Cuối tuần qua, mọi người đổ xô đi mua sắm trong ngày Black Friday khi có tới hơn 226 triệu người mua sắm tại các cửa hàng và trực tuyến.
	\end{vietnamese-v2}
	
	[Anna] You'll have Halloween.
	Then you'll have Thanksgiving stuff.
	Then you need a quick turnaround because you have Christmas.
	Happy holidays.
	- Hanukkah. - [tinkle]
	
	\begin{vietnamese-v2}
		[Anna] Bạn sẽ có Halloween.
		Sau đó bạn sẽ có Lễ Tạ ơn.
		Sau đó bạn cần một sự thay đổi nhanh chóng vì bạn có Giáng sinh.
		Kỳ nghỉ vui vẻ.
		- Hanukkah. - [lè lưỡi]
	\end{vietnamese-v2}
	
	[male voice, echoing] Buy more toys.
	Did you hear that?
	Then you have Valentine's Day.
	This bag is filled with Valentine's Day stuff...
	Then you have Easter.
	Nice Easter shoes, lady!
	Then you have July 4th, and then Memorial Day, Labor Day...
	And then it starts all over again.
	
	\begin{vietnamese-v2}
		[giọng nam, vọng lại] Mua thêm đồ chơi đi.
		Bạn có nghe thấy không?
		Sau đó là Ngày lễ tình nhân.
		Chiếc túi này chứa đầy đồ Ngày lễ tình nhân...
		Sau đó là Lễ Phục sinh.
		Giày Phục sinh đẹp đấy, cô gái!
		Sau đó là Ngày 4 tháng 7, rồi đến Ngày tưởng niệm, Ngày lao động...
		Và rồi mọi thứ lại bắt đầu lại.	
	\end{vietnamese-v2}
	
	[music ends]
	I worked in an investment bank for a couple of years.
	There were a lot of things actually that I enjoyed about it, but your first priority needs to be to the corporation.
	I would wake up, walk a couple blocks, take the subway to work.
	A lot of times around like 9:00 p.m., 10:00 p.m., take a cab home.
	
	\begin{vietnamese-v2}
		[nhạc kết thúc]
		Tôi đã làm việc tại một ngân hàng đầu tư trong một vài năm.
		Thực ra có rất nhiều điều tôi thích ở đó, nhưng ưu tiên hàng đầu của bạn phải là công ty.
		Tôi sẽ thức dậy, đi bộ vài dãy nhà, đi tàu điện ngầm đến nơi làm việc.
		Nhiều lần vào khoảng 9:00 tối, 10:00 tối, bắt taxi về nhà.
	\end{vietnamese-v2}
	
	And then do the same thing the next day.
	I... I left because I wanted to find something that felt more inherently meaningful.
	I do think what I see is the tip of the iceberg.
	Usable items that get discarded are depressing to see.
	
	\begin{vietnamese-v2}
		Và sau đó làm điều tương tự vào ngày hôm sau.
		Tôi... Tôi rời đi vì tôi muốn tìm thứ gì đó có ý nghĩa hơn.
		Tôi nghĩ những gì tôi thấy chỉ là phần nổi của tảng băng chìm.
		Thật đáng buồn khi nhìn thấy những vật dụng có thể sử dụng được nhưng bị vứt bỏ.
	\end{vietnamese-v2}
	
	[man] It's a yoga wheel.
	[Anna] But every now and then, you get glimpses of something even darker.
	[woman] Ahh! It's so cute!
	This is crazy. This is what they do with unwanted merchandise.
	
	\begin{vietnamese-v2}
		[người đàn ông] Đó là một bánh xe yoga.
		[Anna] Nhưng thỉnh thoảng, bạn lại thoáng thấy một thứ gì đó đen tối hơn.
		[người phụ nữ] Ồ! Dễ thương quá!
		Thật điên rồ. Đây là những gì họ làm với hàng hóa không mong muốn.
	\end{vietnamese-v2}
	
	They order an employee to deliberately slash it, so no one can use it.
	Often this happens to prevent products from being sold off at a discount, which companies feel can cheapen the brand image.
	But there are all kinds of reasons, almost always to do with maximizing profit.
	
	\begin{vietnamese-v2}
		Họ ra lệnh cho một nhân viên cố tình cắt nó đi để không ai có thể sử dụng nó.
		Thường thì điều này xảy ra để ngăn chặn các sản phẩm được bán với giá chiết khấu, điều mà các công ty cảm thấy có thể làm giảm giá trị hình ảnh thương hiệu.
		Nhưng có đủ loại lý do, hầu như luôn liên quan đến việc tối đa hóa lợi nhuận.
	\end{vietnamese-v2}
	
	I put out a video asking, "Have you ever worked for a retail corporation that made you deliberately destroy usable items?"
	And I said, "Share your story, use the hashtag 'RetailMadeMe.'"
	
	\begin{vietnamese-v2}
		Tôi đã đăng một video hỏi rằng, "Bạn đã bao giờ làm việc cho một công ty bán lẻ mà bắt bạn phải cố tình phá hủy những vật dụng còn sử dụng được chưa?"
		Và tôi đã nói, "Hãy chia sẻ câu chuyện của bạn, sử dụng hashtag 'RetailMadeMe.'"
	\end{vietnamese-v2}
	
	[overlapping voices]
	[Anna] And it felt like in that moment, the floodgates were opening a little bit.
	
	\begin{vietnamese-v2}
		[giọng nói chồng chéo]
		[Anna] Và cảm giác như vào khoảnh khắc đó, cánh cổng lũ đang mở ra một chút.
	\end{vietnamese-v2}
	
	[man] Yes, I used to work in a Barnes \& Noble café.
	We had to open bakery items, put them in a trash can, and then pour wet coffee grounds on top of them.
	
	\begin{vietnamese-v2}
		[người đàn ông] Vâng, tôi từng làm việc tại quán cà phê Barnes \& Noble.
		Chúng tôi phải mở các sản phẩm bánh, bỏ chúng vào thùng rác, rồi đổ bã cà phê ướt lên trên.
	\end{vietnamese-v2}
	
	We had to throw them out the day before they expired.
	At the end of every shift at Panda Express, you mix all of the food together so that no one wants to eat it, you weigh it and throw it away, so they can keep track of losses.
	
	\begin{vietnamese-v2}
		Chúng tôi phải vứt chúng đi vào ngày trước khi chúng hết hạn.
		Vào cuối mỗi ca làm việc tại Panda Express, bạn trộn tất cả thức ăn lại với nhau để không ai muốn ăn, bạn cân chúng và vứt đi, để họ có thể theo dõi lượng thức ăn bị mất.
	\end{vietnamese-v2}
	
	[orchestra crescendo playing]
	GameStop made us slash the back of discs or just take whole bundles of accessories out to the dumpster.
	They make you destroy everything, all while there is a camera watching the dumpsters.
	Why can't we help our local shelters? This is bullshit.
	
	\begin{vietnamese-v2}
		[đàn nhạc crescendo đang chơi]
		GameStop bắt chúng tôi cắt mặt sau của đĩa hoặc chỉ cần mang toàn bộ các bó phụ kiện ra thùng rác.
		Họ bắt bạn phá hủy mọi thứ, trong khi có một chiếc camera theo dõi thùng rác.
		Tại sao chúng ta không thể giúp đỡ các nơi trú ẩn địa phương của mình? Đây là chuyện nhảm nhí.
	\end{vietnamese-v2}
	
	[woman] I used to work at Bath \& Body Works.
	We had an issue with homeless people dumpster diving, and then taking our product, um, from the dumpster after we'd thrown it away.
	So my manager started having us squeeze out the product into the trash, and she said, "Well, we don't want to be the brand that homeless people use."
	So I quit.
	
	\begin{vietnamese-v2}
		[phụ nữ] Tôi từng làm việc tại Bath \& Body Works.
		Chúng tôi gặp vấn đề với những người vô gia cư lục thùng rác, rồi lấy sản phẩm của chúng tôi, ừm, từ thùng rác sau khi chúng tôi vứt nó đi.
		Vì vậy, quản lý của tôi bắt chúng tôi đổ sản phẩm vào thùng rác, và bà ấy nói, "Ồ, chúng tôi không muốn trở thành thương hiệu mà những người vô gia cư sử dụng."
		Vì vậy, tôi nghỉ việc.
	\end{vietnamese-v2}
	
	[male reporter 1] This is one of Amazon's biggest UK warehouses, and from inside, millions of perfectly good products each year are sent to be destroyed. One former employee, who wishes to remain anonymous, reveals the scale of what they're asked to do.
	
	\begin{vietnamese-v2}
		[phóng viên nam 1] Đây là một trong những nhà kho lớn nhất của Amazon tại Anh, và bên trong, hàng triệu sản phẩm hoàn toàn tốt được gửi đi để tiêu hủy mỗi năm. Một cựu nhân viên, người muốn giấu tên, tiết lộ quy mô những gì họ được yêu cầu làm.
	\end{vietnamese-v2}
	
	[male employee] From a Friday to a Friday, our target was approximately 130,000 items a week.
	There's no rhyme or reason to what gets destroyed.
	
	\begin{vietnamese-v2}
		[nhân viên nam] Từ thứ sáu đến thứ sáu, mục tiêu của chúng tôi là khoảng 130.000 mặt hàng mỗi tuần.
		Không có vần điệu hay lý do nào cho việc những thứ bị phá hủy.
	\end{vietnamese-v2}
	
	[Maren] Amazon was dumping into landfills toys.
	[reporter, in French] We have confirmation that Amazon destroys products.
	This Lego set's worth 128 euros.
	And for Amazon, it was just cheaper to dump them than it was to try to redistribute them.
	
	\begin{vietnamese-v2}
		[Maren] Amazon đã đổ đồ chơi vào bãi rác.
		[phóng viên, bằng tiếng Pháp] Chúng tôi có xác nhận rằng Amazon đã tiêu hủy sản phẩm.
		Bộ Lego này có giá trị 128 euro.
		Và đối với Amazon, việc đổ chúng đi rẻ hơn nhiều so với việc cố gắng phân phối lại chúng.
	\end{vietnamese-v2}
	
	[male reporter 2] One recent estimate suggests returns, including those from Amazon, accounted for five billion pounds of landfilled waste in the US alone.
	
	\begin{vietnamese-v2}
		[phóng viên nam 2] Một ước tính gần đây cho thấy lượng hàng trả lại, bao gồm cả hàng trả lại từ Amazon, chiếm tới năm tỷ pound chất thải chôn lấp chỉ tính riêng tại Hoa Kỳ.
	\end{vietnamese-v2}
	
	[Maren] And when you make products, it actually generates a lot of planet-warming emissions. We know now that this is contributing to climate change.
	
	\begin{vietnamese-v2}
		[Maren] Và khi bạn tạo ra sản phẩm, nó thực sự tạo ra rất nhiều khí thải làm nóng hành tinh. Chúng ta hiện biết rằng điều này đang góp phần vào biến đổi khí hậu.
	\end{vietnamese-v2}
	
	[soft electronic music plays]
	So if you destroy stuff before it's even used once, that is just insane.
	What math are they doing?
	But you know that... You know that they've calculated it out, and somehow it equals profit.
	
	\begin{vietnamese-v2}
		[nhạc điện tử nhẹ nhàng phát]
		Vậy nếu bạn phá hủy đồ đạc trước khi nó được sử dụng lần nào, thì điều đó thật điên rồ.
		Họ đang tính toán cái gì vậy?
		Nhưng bạn biết rằng... Bạn biết rằng họ đã tính toán ra rồi, và bằng cách nào đó nó bằng lợi nhuận.
	\end{vietnamese-v2}
	
	Everybody wants to believe that the company that they're working for is not evil.
	Back then, I didn't have time to think about anything else besides just trying to keep my job and, you know, raise my kids.
	
	\begin{vietnamese-v2}
		Mọi người đều muốn tin rằng công ty mà họ đang làm việc không phải là xấu xa.
		Lúc đó, tôi không có thời gian để nghĩ về bất cứ điều gì khác ngoài việc cố gắng giữ công việc của mình và, bạn biết đấy, nuôi dạy con cái.
	\end{vietnamese-v2}
	
	I had been drinking the Kool-Aid. I was drinking the Kool-Aid.
	But I was really starting to know that... that Amazon was not... [chuckles] not a net good in the world.
	
	\begin{vietnamese-v2}
		Tôi đã uống Kool-Aid. Tôi đã uống Kool-Aid.
		Nhưng tôi thực sự bắt đầu biết rằng... rằng Amazon không phải... [cười khúc khích] không phải là một thứ tốt đẹp trên thế giới.
	\end{vietnamese-v2}
	
	[soft string music plays]
	Yeah, you know, there's no free lunch, you know. Yeah.
	If it feels too good to be true, there's probably some... some consequence, or cost, that you're not thinking about that you're paying. And it will, you know, it'll come home to roost at some point.
	
	\begin{vietnamese-v2}
		[nhạc dây nhẹ nhàng vang lên]
		Vâng, bạn biết đấy, không có bữa trưa nào miễn phí, bạn biết đấy. Vâng.
		Nếu cảm thấy quá tốt để trở thành sự thật, có lẽ có một số... một số hậu quả, hoặc chi phí, mà bạn không nghĩ đến mà bạn đang phải trả. Và nó sẽ, bạn biết đấy, nó sẽ trở về nhà để ngủ vào một lúc nào đó.
	\end{vietnamese-v2}
	
	[music continues]
	[chatter]
	Um, I remember being at a happy hour with some friends.
	And a friend of mine was just saying, like, "Maren, you know how they treat their workers, and you know how bad it is for the planet."
	"Like, how can you work there? How can you work there?"
	I kind of needed somebody to just hammer that into me.
	That was the moment I actually... I actually remember I cried.
	
	\begin{vietnamese-v2}
		[nhạc tiếp tục]
		[nói chuyện phiếm]
		Ừm, tôi nhớ là mình đã đi uống rượu với một số người bạn.
		Và một người bạn của tôi chỉ nói rằng, "Maren, bạn biết họ đối xử với công nhân của họ như thế nào, và bạn biết điều đó tệ hại thế nào đối với hành tinh này."
		"Kiểu như, làm sao bạn có thể làm việc ở đó? Làm sao bạn có thể làm việc ở đó?"
		Tôi cần ai đó chỉ cho tôi biết điều đó.
		Đó là khoảnh khắc tôi thực sự... Tôi thực sự nhớ là tôi đã khóc.
	\end{vietnamese-v2}
	
	[soft sniffing]
	And I was like, "Okay."
	"I know."
	"I... You're right and I have to..."
	"I have to either do something or walk away."
	
	\begin{vietnamese-v2}
		[hít nhẹ]
		Và tôi kiểu như, "Được thôi."
		"Tôi biết."
		"Tôi... Anh đúng và tôi phải..."
		"Tôi phải làm gì đó hoặc bỏ đi."
	\end{vietnamese-v2}
	
	[electronic buzzing]
	[Sasha] Cut to relaxing imagery. [echoing]
	[heartbeats]
	[inhales]
	Please be aware that from this point on new levels of commitment and belief will be required in this interaction.
	
	\begin{vietnamese-v2}
		[tiếng vo ve điện tử]
		[Sasha] Chuyển sang hình ảnh thư giãn. [tiếng vọng]
		[nhịp tim]
		[hít vào]
		Xin lưu ý rằng từ thời điểm này trở đi, sẽ cần có những cấp độ cam kết và niềm tin mới trong tương tác này.
	\end{vietnamese-v2}
	
	[glass cracking]
	To keep consumers buying, you will now need to master creative interpretations of the truth.
	
	\begin{vietnamese-v2}
		[kính nứt]
		Để giữ chân người tiêu dùng, giờ đây bạn cần phải nắm vững cách diễn giải sáng tạo về sự thật.
	\end{vietnamese-v2}
	
	[flapping]
	Rule three. Lie more.
	What's interesting in terms of how people view businesses today is that they actually trust businesses over other large social institutions.
	But that trust isn't always well-placed.
	
	\begin{vietnamese-v2}
		[vỗ tay]
		Quy tắc thứ ba. Nói dối nhiều hơn.
		Điều thú vị về cách mọi người nhìn nhận doanh nghiệp ngày nay là họ thực sự tin tưởng doanh nghiệp hơn các tổ chức xã hội lớn khác.
		Nhưng niềm tin đó không phải lúc nào cũng đúng chỗ.
	\end{vietnamese-v2}
	
	[beep]
	♪ I'd like to buy the world a home ♪
	♪ And furnish it with love... ♪
	
	\begin{vietnamese-v2}
		[bíp]
		♪ Tôi muốn mua cho thế giới một ngôi nhà ♪
		♪ Và trang bị cho nó tình yêu thương... ♪
	\end{vietnamese-v2}
	
	[Mara] I remember when the "Coke on the Hill" commercial came out,
	I thought it was the coolest thing on the planet.
	♪...white turtledove ♪
	♪ I'd like to teach the world... ♪
	
	\begin{vietnamese-v2}
		[Mara] Tôi nhớ khi quảng cáo "Coke on the Hill" ra mắt,
		Tôi nghĩ đó là điều tuyệt vời nhất trên hành tinh này.
		♪...chim bồ câu trắng ♪
		♪ Tôi muốn dạy thế giới... ♪
	\end{vietnamese-v2}
	
	[Mara] But what Coke was doing, at least in part, was building trust with their consumers by tapping into growing anxieties about the environment.
	♪...and keep it company ♪
	♪ That's the real thing ♪
	♪ I'd like to teach the world to sing In perfect harmony... ♪
	
	\begin{vietnamese-v2}
		[Mara] Nhưng những gì Coke đã làm, ít nhất là một phần, là xây dựng lòng tin với người tiêu dùng bằng cách khai thác nỗi lo lắng ngày càng tăng về môi trường.
		♪...và giữ nó làm bạn ♪
		♪ Đó là điều thực sự ♪
		♪ Tôi muốn dạy thế giới hát Trong sự hòa hợp hoàn hảo... ♪
	\end{vietnamese-v2}
	
	[Mara] The issue with companies connecting themselves to the environment is that they're doing what marketers always do, which is showing you the shiny little thing over here, because they don't want you to look at what they're doing here.
	
	\begin{vietnamese-v2}
		[Mara] Vấn đề với các công ty khi kết nối mình với môi trường là họ đang làm những gì các nhà tiếp thị vẫn làm, đó là cho bạn thấy thứ nhỏ bé sáng bóng ở đây, vì họ không muốn bạn nhìn vào những gì họ đang làm ở đây.
	\end{vietnamese-v2}
	
	["Dance of the Sugar Plum Fairy" from The Nutcracker by Tchaikovsky plays]
	I don't think there's a bigger or better example of this than the way that companies like Coke and others have been pumping out plastic while telling us that recycling is going to fix the problem.
	
	\begin{vietnamese-v2}
		["Vũ điệu của nàng tiên mận đường" trong vở The Nutcracker của Tchaikovsky phát]
		Tôi không nghĩ có ví dụ nào lớn hơn hoặc tốt hơn về điều này hơn cách các công ty như Coke và các công ty khác đã thải ra nhựa trong khi nói với chúng ta rằng tái chế sẽ giải quyết được vấn đề.
	\end{vietnamese-v2}
	
	[ad announcer] After enjoying our drinks, please recycle.
	[Mara] The truth is very, very different.
	
	\begin{vietnamese-v2}
		[người dẫn chương trình quảng cáo] Sau khi thưởng thức đồ uống của chúng tôi, vui lòng tái chế.
		[Mara] Sự thật thì rất, rất khác.	
	\end{vietnamese-v2}
	
	[music continues]
	[Jan] Just like in the movie The Sixth Sense, the little kid said, "I see dead people,"
	I see lying labels everywhere I go.
	I'm trying to make lying labels a meme.
	It's \#lyinglabels.
	
	\begin{vietnamese-v2}
		[nhạc tiếp tục]
		[Jan] Giống như trong phim The Sixth Sense, đứa trẻ nói, "Tôi thấy người chết,"
		Tôi thấy những nhãn mác dối trá ở khắp mọi nơi tôi đến.
		Tôi đang cố gắng biến những nhãn mác dối trá thành một meme.
		Đó là \#lyinglabels.
	\end{vietnamese-v2}
	
	Based on my opinion of being in thousands of stores and trying to find factories that actually recycle things, the vast majority of recyclable labels on plastic packaging today are false.
	I've worked with the biggest, most well-known brands who make footwear, apparel, toys.
	I have helped companies figure out how to design and manufacture safely, efficiently, and really environmentally best way.
	And these companies worked really hard to make their factories really efficient and not hurt the environment.
	
	\begin{vietnamese-v2}
		Dựa trên quan điểm của tôi khi đến hàng nghìn cửa hàng và cố gắng tìm kiếm các nhà máy thực sự tái chế đồ vật, phần lớn các nhãn tái chế trên bao bì nhựa hiện nay đều là nhãn giả.
		Tôi đã làm việc với những thương hiệu lớn nhất, nổi tiếng nhất sản xuất giày dép, quần áo, đồ chơi.
		Tôi đã giúp các công ty tìm ra cách thiết kế và sản xuất an toàn, hiệu quả và thực sự thân thiện với môi trường.
		Và những công ty này đã nỗ lực thực sự để làm cho các nhà máy của họ thực sự hiệu quả và không gây hại cho môi trường.
	\end{vietnamese-v2}
	
	But once they made the product and they put it on the store shelf, they wipe their hands of it.
	And they say, "That is not our responsibility."
	
	\begin{vietnamese-v2}
		Nhưng một khi họ đã làm ra sản phẩm và đặt nó lên kệ hàng, họ phủi tay khỏi nó.
		Và họ nói, "Đó không phải là trách nhiệm của chúng tôi."
	\end{vietnamese-v2}
	
	You can see the plastic packaging is everywhere in the stores today, and consumers can't even avoid it.
	The truth is, is that the vast majority of plastics are not recyclable.
	This is an example of a product that's using a recyclable label for this flat plastic lid that isn't recyclable.
	
	\begin{vietnamese-v2}
		Bạn có thể thấy bao bì nhựa có ở khắp mọi nơi trong các cửa hàng ngày nay và người tiêu dùng thậm chí không thể tránh khỏi nó.
		Sự thật là, phần lớn nhựa không thể tái chế.
		Đây là ví dụ về một sản phẩm sử dụng nhãn có thể tái chế cho nắp nhựa phẳng không thể tái chế này.
	\end{vietnamese-v2}

	This topic is something that I have deep, broad expertise on, and I simply can't stay quiet.
	Where was the toilet paper aisle?
	
	\begin{vietnamese-v2}
		Đây là chủ đề mà tôi có chuyên môn sâu rộng và tôi không thể im lặng được.
		Quầy giấy vệ sinh ở đâu?
	\end{vietnamese-v2}
	
	Here we go.
	So product companies are putting these chasing arrows, huge chasing arrows, on all of this plastic packaging to try to convince the consumer that it's recyclable.
	
	\begin{vietnamese-v2}
		Chúng ta bắt đầu thôi.
		Vì vậy, các công ty sản phẩm đang dán những mũi tên đuổi theo này, những mũi tên đuổi theo khổng lồ, lên tất cả các bao bì nhựa này để cố gắng thuyết phục người tiêu dùng rằng chúng có thể tái chế được.
	\end{vietnamese-v2}
	
	They can buy it guilt-free.
	Throw it in the recycle bin.
	No problem.
	[echoing] No problem...
	[Sasha] Rules around packaging are lax.
	So let this become your canvas for creativity.
	
	\begin{vietnamese-v2}
		Họ có thể mua nó mà không cảm thấy tội lỗi.
		Vứt nó vào thùng tái chế.
		Không vấn đề gì.
		[lặp lại] Không vấn đề gì...
		[Sasha] Các quy tắc về bao bì rất lỏng lẻo.
		Vậy hãy để đây trở thành nền tảng cho sự sáng tạo của bạn.
	\end{vietnamese-v2}
	
	Honestly, you really can say whatever the hell you want with very little repercussions.
	Show examples.
	Plastic number six.
	Consumer understanding:
	"This product can be reused again and again." Real meaning: "This product will be collected, then get sorted, before likely getting buried or burned."
	
	\begin{vietnamese-v2}
		Thành thật mà nói, bạn thực sự có thể nói bất cứ điều gì bạn muốn mà không phải chịu hậu quả gì.
		Hiển thị ví dụ.
		Nhựa số sáu.
		Hiểu biết của người tiêu dùng:
		"Sản phẩm này có thể được tái sử dụng nhiều lần." Ý nghĩa thực sự: "Sản phẩm này sẽ được thu gom, sau đó được phân loại, trước khi có khả năng bị chôn hoặc đốt."
	\end{vietnamese-v2}
	
	Store drop-off.
	Consumer understanding: "Take a bit of extra time to be a good citizen."
	Real meaning: "Stores will collect recycling."
	"They will then pass items on likely to get buried or burned."
	
	\begin{vietnamese-v2}
		Bỏ hàng tại cửa hàng.
		Hiểu biết của người tiêu dùng: "Dành thêm chút thời gian để trở thành một công dân tốt."
		Nghĩa thực sự: "Các cửa hàng sẽ thu gom đồ tái chế."
		"Sau đó, họ sẽ chuyển những món đồ có khả năng bị chôn hoặc đốt."
	\end{vietnamese-v2}
	
	The following symbols are largely meaningless.
	But they will help consumers feel better about buying things that will very likely be buried or burned.
	
	\begin{vietnamese-v2}
		Các biểu tượng sau đây phần lớn không có ý nghĩa gì.
		Nhưng chúng sẽ giúp người tiêu dùng cảm thấy an tâm hơn khi mua những thứ rất có thể sẽ bị chôn vùi hoặc đốt cháy.
	\end{vietnamese-v2}
	
	[Jen] Globally, we recycle less than 10\% of all the plastic we produce, so, it's a mistake to keep saying that the answer to the plastic pollution problem is to recycle more.
	
	\begin{vietnamese-v2}
		[Jen] Trên toàn cầu, chúng ta tái chế chưa đến 10\% tổng lượng nhựa mà chúng ta sản xuất, vì vậy, sẽ là sai lầm khi cứ nói rằng câu trả lời cho vấn đề ô nhiễm nhựa là tái chế nhiều hơn.
	\end{vietnamese-v2}
	
	[chuckles] The solution is to make less plastic.
	[man] These are not recyclable.
	These are not recyclable. Garbage. Garbage.
	
	\begin{vietnamese-v2}
		[cười khúc khích] Giải pháp là sản xuất ít nhựa hơn.
		[người đàn ông] Những thứ này không thể tái chế.
		Những thứ này không thể tái chế. Rác. Rác.
	\end{vietnamese-v2}
	
	[Jan] The products companies today are telling us,
	"Buy more, have more stuff."
	"Just as long as you recycle, everything's gonna be okay."
	But the mountains and mountains of plastic waste that are all over the world prove that this isn't true.
	We simply cannot recycle our way out of all this stuff that they want us to buy.
	
	\begin{vietnamese-v2}
		[Jan] Các công ty sản xuất sản phẩm ngày nay đang nói với chúng ta rằng,
		"Mua nhiều hơn, có nhiều đồ hơn."
		"Chỉ cần bạn tái chế, mọi thứ sẽ ổn thôi."
		Nhưng hàng núi rác thải nhựa trên khắp thế giới chứng minh rằng điều này không đúng.
		Chúng ta không thể tái chế để thoát khỏi tất cả những thứ mà họ muốn chúng ta mua.
	\end{vietnamese-v2}
	
	
	[pulsing]
	[Sasha] Never forget.
	[buzzing]
	Once the product has been sold and used, it is no longer your responsibility.
	Don't worry, post purchase, products will generally fend for themselves.
	Create short fictional film depicting afterlife of sentient chip packet.
	Cue dramatic music.
	
	\begin{vietnamese-v2}
		[đập]
		[Sasha] Đừng bao giờ quên.
		[vù vù]
		Sau khi sản phẩm đã được bán và sử dụng, bạn không còn phải chịu trách nhiệm nữa.
		Đừng lo lắng, sau khi mua, sản phẩm thường sẽ tự bảo vệ mình.
		Tạo một bộ phim hư cấu ngắn mô tả cuộc sống sau khi chết của gói chip có tri giác.
		Bật nhạc kịch tính.
	\end{vietnamese-v2}
	
	[dramatic music plays]
	[Chip packet] I was built to last.
	My creators wove me from metal and plastic to create an impervious shell.
	
	\begin{vietnamese-v2}
		[nhạc kịch tính vang lên]
		[Gói chip] Tôi được tạo ra để trường tồn.
		Người sáng tạo ra tôi đã dệt tôi từ kim loại và nhựa để tạo ra một lớp vỏ không thấm nước.
	\end{vietnamese-v2}
	
	[dramatic music crescendos]
	But I was brutally robbed.
	I was discarded and left to die.
	
	\begin{vietnamese-v2}
		[nhạc kịch tính lên đến cao trào]
		Nhưng tôi đã bị cướp một cách tàn bạo.
		Tôi đã bị vứt bỏ và bỏ mặc cho đến chết.
	\end{vietnamese-v2}
	
	[thunder rumbles]
	But I wasn't ready to give up.
	I will travel the world
	to find others who share my fate.
	Do not cry for me.
	
	\begin{vietnamese-v2}
		[tiếng sấm rền]
		Nhưng tôi không sẵn sàng từ bỏ.
		Tôi sẽ đi khắp thế giới
		để tìm những người khác có chung số phận với tôi.
		Đừng khóc vì tôi.
	\end{vietnamese-v2}
	
	[wave crashes]
	I will endure.
	[Sasha] Cut to product ten years from now.
	One hundred years from now.
	
	\begin{vietnamese-v2}
		[sóng vỗ]
		Tôi sẽ chịu đựng.
		[Sasha] Cắt đến sản phẩm mười năm sau.
		Một trăm năm sau.
	\end{vietnamese-v2}
	
	[pulsing]
	Food packaging is awesome. [echoing]
	Other examples of long-lived but potentially damaging products include tablets and phones, ["Morgenstimmung" by Edvard Grieg playing]
	
	\begin{vietnamese-v2}
		[pulsing]
		Bao bì thực phẩm thật tuyệt. [echoing]
		Các ví dụ khác về các sản phẩm tồn tại lâu dài nhưng có khả năng gây hại bao gồm máy tính bảng và điện thoại, ["Morgenstimmung" của Edvard Grieg đang phát]
	\end{vietnamese-v2}
	
	[child laughs]
	clothing made from synthetic plastic fibers,
	toys, various.
	[violin music continues]
	If consumers begin to feel nervous about purchasing these and other similar items, a simple, misleading label may not be enough.
	You may need to invest in more extreme reassurance.
	It's an approach known as "greenwashing."
	And it will almost always be cheaper than tackling the actual issues.
	
	\begin{vietnamese-v2}
		[trẻ em cười]
		quần áo làm từ sợi nhựa tổng hợp,
		đồ chơi, nhiều thứ khác.
		[nhạc violin tiếp tục]
		Nếu người tiêu dùng bắt đầu cảm thấy lo lắng khi mua những mặt hàng này và các mặt hàng tương tự khác, một nhãn mác đơn giản, gây hiểu lầm có thể không đủ.
		Bạn có thể cần đầu tư vào sự đảm bảo cực đoan hơn.
		Đó là một cách tiếp cận được gọi là "rửa xanh".
		Và nó hầu như luôn rẻ hơn so với việc giải quyết các vấn đề thực tế.
	\end{vietnamese-v2}
	
	[female voice] Just imagine a world where a dress can have a positive impact on the planet.
	Greenwashing, quite simply, is when companies pretend to care about sustainability and actually don't... Can I say, "Give a f...?"
	
	\begin{vietnamese-v2}
		[giọng nữ] Hãy tưởng tượng một thế giới mà một chiếc váy có thể có tác động tích cực đến hành tinh.
		Greenwashing, nói một cách đơn giản, là khi các công ty giả vờ quan tâm đến tính bền vững nhưng thực tế lại không... Tôi có thể nói, "Mày đ...?"
	\end{vietnamese-v2}
	
	[laughs]
	And don't actually give a fuck.
	It could be as subtle as using natural environments in ads or getting children to deliver your message.
	I'm 11 years old and the thing that I love to do is recycle!

	\begin{vietnamese-v2}
		[cười]
		Và thực sự không quan tâm.
		Có thể tinh tế như sử dụng môi trường tự nhiên trong quảng cáo hoặc nhờ trẻ em truyền tải thông điệp của bạn.
		Tôi 11 tuổi và điều tôi thích làm là tái chế!
	\end{vietnamese-v2}

	[Mara] Or simply extensive use of the color green.
	All of the product you make is actually causing environmental damage.
	I think there's a fair amount of greenwashing that's going on.
	And greenwashing to me means you're being duplicitous.
	So I use that term very limited	because greenwashing, we have to be careful about who we point at because you're saying, "You're a liar."
	And that's very... That's a very strong-arm.
	
	\begin{vietnamese-v2}
		[Mara] Hoặc đơn giản là sử dụng quá nhiều màu xanh lá cây.
		Tất cả các sản phẩm bạn làm ra thực sự đang gây ra thiệt hại cho môi trường.
		Tôi nghĩ rằng có khá nhiều hoạt động tẩy xanh đang diễn ra.
		Và đối với tôi, tẩy xanh có nghĩa là bạn đang gian dối.
		Vì vậy, tôi sử dụng thuật ngữ đó rất hạn chế vì khi nói đến tẩy xanh, chúng ta phải cẩn thận về việc chúng ta chỉ vào ai vì bạn đang nói rằng, "Bạn là kẻ nói dối".
		Và điều đó rất... Đó là một hành động rất mạnh tay.
	\end{vietnamese-v2}
	
	["Morgenstimmung" by Edvard Grieg playing]
	The other word is, there's a lot of green-wishing going on.
	It's basically when the board, the people that run the companies, think they're doing enough.
	Mother Nature, welcome to Apple.
	How was the weather getting in?
	
	\begin{vietnamese-v2}
		["Morgenstimmung" của Edvard Grieg đang phát]
		Từ khác là, có rất nhiều mong muốn xanh đang diễn ra.
		Về cơ bản, đó là khi hội đồng quản trị, những người điều hành các công ty, nghĩ rằng họ đã làm đủ.
		Mẹ thiên nhiên, chào mừng đến với Apple.
		Thời tiết thế nào?
	\end{vietnamese-v2}
	
	[Mara] The reason why consumers trust corporations is because they think they are doing these good things.
	Are some of them doing some of those things? Yes.
	At a level that really makes a difference? No.
	The vast majority, no.
	
	\begin{vietnamese-v2}
		[Mara] Lý do tại sao người tiêu dùng tin tưởng các tập đoàn là vì họ nghĩ rằng họ đang làm những điều tốt đẹp này.
		Một số trong số họ có làm một số điều đó không? Có.
		Ở mức độ thực sự tạo ra sự khác biệt? Không.
		Phần lớn là không.
	\end{vietnamese-v2}
	
	[woman] Alexa, turn off the lights.
	[Maren] Greenwashing is... It's like a double evil.
	Not only are you not doing what you said you were gonna do, but you're also pacifying people.
	
	\begin{vietnamese-v2}
		[phụ nữ] Alexa, tắt đèn đi.
		[Maren] Greenwashing là... Giống như một điều ác kép vậy.
		Bạn không chỉ không làm những gì bạn đã nói bạn sẽ làm mà còn đang xoa dịu mọi người.
	\end{vietnamese-v2}
	
	[click]
	[click]
	It just got to the point for me where I decided I had to get involved with trying to change things.
	
	\begin{vietnamese-v2}
		[click]
		[click]
		Đến lúc tôi quyết định phải tham gia vào nỗ lực thay đổi mọi thứ.
	\end{vietnamese-v2}
	
	[click]
	I joined with a group of other employees.
	We just wanted to push Amazon to do more.
	
	\begin{vietnamese-v2}
		[click]
		Tôi đã tham gia cùng một nhóm nhân viên khác.
		Chúng tôi chỉ muốn thúc đẩy Amazon làm nhiều hơn nữa.
	\end{vietnamese-v2}
	
	[hopeful music playing]
	They called us into a meeting.
	And they asked us to keep the meeting secret.
	They were very aggressive. It was really...
	
	\begin{vietnamese-v2}
		[nhạc hy vọng vang lên]
		Họ gọi chúng tôi vào một cuộc họp.
		Và họ yêu cầu chúng tôi giữ bí mật cuộc họp.
		Họ rất hung hăng. Thật sự là...
	\end{vietnamese-v2}
	
	[chuckles] It was, like, "What's going on?"
	It was, like, "I want to be able to look my kids in the eye 20 years from now."
	At the time, I was just thinking, "I know now that what we're doing is not sustainable."
	
	\begin{vietnamese-v2}
		[cười khúc khích] Kiểu như, "Chuyện gì đang xảy ra vậy?"
		Kiểu như, "Tôi muốn có thể nhìn thẳng vào mắt con mình sau 20 năm nữa."
		Lúc đó, tôi chỉ nghĩ, "Giờ tôi biết rằng những gì chúng ta đang làm là không bền vững."
	\end{vietnamese-v2}
	
	Amazon had no meaningful goals, no dates, no, uh, plans.
	There was nothing to say like, "We are gonna measure our carbon footprint."
	
	\begin{vietnamese-v2}
		Amazon không có mục tiêu có ý nghĩa, không có ngày tháng, không có, ừm, kế hoạch.
		Không có gì để nói như, "Chúng tôi sẽ đo lượng khí thải carbon của mình."
	\end{vietnamese-v2}
	
	We wanted Amazon to have a company-wide climate plan.
	There's no issue more important to our customers, to our world, than the climate crisis, and we are falling far short.
	
	\begin{vietnamese-v2}
		Chúng tôi muốn Amazon có một kế hoạch khí hậu cho toàn công ty.
		Không có vấn đề nào quan trọng hơn đối với khách hàng của chúng tôi, đối với thế giới của chúng tôi hơn là cuộc khủng hoảng khí hậu, và chúng tôi đang tụt hậu rất xa.	
	\end{vietnamese-v2}
	
	I'd like to ask for Jeff Bezos to come out on stage so that I can speak to him directly.
	I represent 7,700 of his employees.
	
	\begin{vietnamese-v2}
		Tôi muốn mời Jeff Bezos lên sân khấu để tôi có thể nói chuyện trực tiếp với ông ấy.
		Tôi đại diện cho 7.700 nhân viên của ông ấy.
	\end{vietnamese-v2}
	
	[man] Mr. Bezos will be out later, thank you.
	Will he be hearing this speech?
	
	\begin{vietnamese-v2}
		[man] Ông Bezos sẽ ra ngoài sau, cảm ơn.
		Ông ấy có nghe bài phát biểu này không?
	\end{vietnamese-v2}
	
	[man] I assume so.
	We would say, use your outrage, because outrage will create action, and then action creates hope.
	
	\begin{vietnamese-v2}
		[người đàn ông] Tôi cho là vậy.
		Chúng ta sẽ nói, hãy sử dụng sự phẫn nộ của bạn, bởi vì sự phẫn nộ sẽ tạo ra hành động, và sau đó hành động sẽ tạo ra hy vọng.
	\end{vietnamese-v2}
	
	[reporter] Amazon employees will walk out over the company's climate change inaction this week.
	- I'm walking out. - I'm walking out.
	- I'm walking out. - I'm walking out.
	I'm walking out.
	
	\begin{vietnamese-v2}
		[phóng viên] Nhân viên Amazon sẽ bãi công vì công ty không hành động gì về biến đổi khí hậu trong tuần này.
		- Tôi sẽ bãi công. - Tôi sẽ bãi công.
		- Tôi sẽ bãi công. - Tôi sẽ bãi công.
		Tôi sẽ bãi công.
	\end{vietnamese-v2}
	
	[reporter] The planned event will mark the first time in Amazon's 25-year history that workers at the company's Seattle headquarters have participated in a strike.
	And the night before the strike, Amazon announced its climate pledge.
	Jeff Bezos, the founder and CEO of Amazon, is pledging Amazon will be one of the first companies if not the first company to meet the Paris Climate Accords and climate pledges, ten years early.
	
	\begin{vietnamese-v2}
		[phóng viên] Sự kiện được lên kế hoạch sẽ đánh dấu lần đầu tiên trong lịch sử 25 năm của Amazon, công nhân tại trụ sở chính của công ty ở Seattle tham gia đình công.
		Và đêm trước cuộc đình công, Amazon đã công bố cam kết về khí hậu của mình.
		Jeff Bezos, người sáng lập kiêm giám đốc điều hành của Amazon, cam kết Amazon sẽ là một trong những công ty đầu tiên nếu không muốn nói là công ty đầu tiên đạt được Hiệp định khí hậu Paris và các cam kết về khí hậu sớm hơn mười năm.
	\end{vietnamese-v2}
	
	The science community...
	[Maren] I thought, "Would Amazon really be able to change in a way that makes up for the damage it causes?"
	I don't know.
	Yeah, I mean, I was calling them on their shit.
	That was something that they really didn't want to have happen.
	
	\begin{vietnamese-v2}
		Cộng đồng khoa học...
		[Maren] Tôi nghĩ, "Liệu Amazon có thực sự có thể thay đổi theo cách bù đắp cho thiệt hại mà nó gây ra không?"
		Tôi không biết.
		Vâng, ý tôi là, tôi đã gọi họ vào đống rác rưởi của họ.
		Đó là điều mà họ thực sự không muốn xảy ra.
	\end{vietnamese-v2}
	
	[pulsing]
	[Sasha] Never forget.
	You must always stay one step ahead of your critics.
	But however well you do this, your continued growth will still have challenging consequences.
	Ones that you will have to learn how to minimize.
	Come on, let me show you.
	
	\begin{vietnamese-v2}
		[nhịp đập]
		[Sasha] Đừng bao giờ quên.
		Bạn phải luôn đi trước những người chỉ trích mình một bước.
		Nhưng dù bạn làm tốt đến đâu, sự phát triển liên tục của bạn vẫn sẽ có những hậu quả đầy thách thức.
		Những hậu quả mà bạn sẽ phải học cách giảm thiểu.
		Nào, để tôi chỉ cho bạn.
	\end{vietnamese-v2}
	
	[harmonica playing melody]
	Visualize 400 million tons of annual plastic waste.
	Visualize 50 million tons of electronic waste produced each year.
	
	\begin{vietnamese-v2}
		[giai điệu harmonica]
		Hình dung 400 triệu tấn rác thải nhựa hàng năm.
		Hình dung 50 triệu tấn rác thải điện tử được tạo ra mỗi năm.
	\end{vietnamese-v2}
	
	[upbeat electronic music plays]
	After ten years of running, uh, Unilever, I felt myself, I could have a bigger impact in the world by moving outside of the corporate world.
	If you run businesses just simply for the short term, for the shareholders alone... not caring about the negative consequences of what you're doing, then there's something fundamentally wrong.
	
	\begin{vietnamese-v2}
		[nhạc điện tử sôi động vang lên]
		Sau mười năm điều hành, uh, Unilever, tôi cảm thấy mình có thể tạo ra tác động lớn hơn trên thế giới bằng cách chuyển ra khỏi thế giới doanh nghiệp.
		Nếu bạn điều hành doanh nghiệp chỉ đơn giản là vì lợi ích ngắn hạn, chỉ vì các cổ đông... mà không quan tâm đến hậu quả tiêu cực của những gì bạn đang làm, thì về cơ bản là có điều gì đó không ổn.
	\end{vietnamese-v2}
	
	Yet, that is exactly what we've been doing.
	[Sasha] Visualize yearly waste generated in 2050.
	Enough to fill central Tokyo and double what we produce now.
	
	\begin{vietnamese-v2}
		Tuy nhiên, đó chính xác là những gì chúng ta đã và đang làm.
		[Sasha] Hình dung lượng rác thải hàng năm được tạo ra vào năm 2050.
		Đủ để lấp đầy trung tâm Tokyo và tăng gấp đôi lượng rác thải chúng ta đang sản xuất hiện nay.
	\end{vietnamese-v2}
	
	[Pablo] As long as we define success as producing more stuff, more profits, I think, unfortunately, we are in trouble.
	[Sasha] Remember, if consumers become aware of waste issues, it may negatively impact growth.
	
	\begin{vietnamese-v2}
		[Pablo] Miễn là chúng ta định nghĩa thành công là sản xuất nhiều hàng hóa hơn, nhiều lợi nhuận hơn, tôi nghĩ, thật không may, chúng ta đang gặp rắc rối.
		[Sasha] Hãy nhớ rằng, nếu người tiêu dùng nhận thức được vấn đề lãng phí, điều đó có thể tác động tiêu cực đến tăng trưởng.
	\end{vietnamese-v2}
	
	Accordingly, you must learn how to more effectively conceal the problem.
	Do not be seen. [echoing]
	
	\begin{vietnamese-v2}
		Theo đó, bạn phải học cách che giấu vấn đề hiệu quả hơn.
		Đừng để bị nhìn thấy. [lặp lại]
	\end{vietnamese-v2}
	
	[flapping]
	Rule four. Hide more.
	
	\begin{vietnamese-v2}
		[vỗ tay]
		Quy tắc thứ tư. Ẩn nhiều hơn.
	\end{vietnamese-v2}
	
	[woman shouting in Chinese]
	[men on video speaking in Chinese]
	[Jim] See if you can see any way, if you can find out where they're from, any tags on them or...
	
	\begin{vietnamese-v2}
		[người phụ nữ hét lên bằng tiếng Trung]
		[người đàn ông trên video nói bằng tiếng Trung]
		[Jim] Hãy xem bạn có thể nhìn thấy bằng cách nào không, nếu bạn có thể tìm ra họ đến từ đâu, có bất kỳ thẻ nào trên người họ hay...
	\end{vietnamese-v2}
	
	[woman] A lot of these are the Dell brand.
	[Jim] There's all kinds of brands, yeah.
	Three-in-one faxes, printers, everything.
	
	\begin{vietnamese-v2}
		[phụ nữ] Rất nhiều trong số này là thương hiệu Dell.
		[Jim] Có đủ loại thương hiệu, đúng vậy.
		Máy fax ba trong một, máy in, mọi thứ.
	\end{vietnamese-v2}
	
	[upbeat electronic music]
	I've been called the James Bond of waste.
	I don't know where that came from.
	My whole career has been tracking waste and figuring out what happens to our stuff that we throw to this magical place called "away."
	
	\begin{vietnamese-v2}
		[nhạc điện tử sôi động]
		Tôi được gọi là James Bond của rác thải.
		Tôi không biết từ đâu mà có.
		Toàn bộ sự nghiệp của tôi là theo dõi rác thải và tìm hiểu xem điều gì sẽ xảy ra với những thứ chúng ta vứt vào nơi kỳ diệu này được gọi là "xa".
	\end{vietnamese-v2}
	
	Well, we've always had waste, but it's gotten so much more volume and so much more toxic and persistent in the environment.
	Right now, today, we're tracking about 400 devices.
	We like to use, um, LCD flat-screen monitors.
	We'll put, um, the tracker in a place like right here, put it all back together, and we deliver it to a so-called recycler.
	
	\begin{vietnamese-v2}
		Vâng, chúng tôi luôn có chất thải, nhưng chúng ngày càng có khối lượng lớn hơn và độc hại hơn cũng như dai dẳng hơn trong môi trường.
		Ngay bây giờ, hôm nay, chúng tôi đang theo dõi khoảng 400 thiết bị.
		Chúng tôi thích sử dụng màn hình phẳng LCD.
		Chúng tôi sẽ đặt máy theo dõi ở một nơi như ngay đây, lắp ráp lại tất cả và chúng tôi sẽ giao nó cho một cái gọi là đơn vị tái chế.
	\end{vietnamese-v2}
	
	And then we see what really happens to it.
	This was a recycling facility in Dresden, Germany, where we dropped an LCD.
	We followed across Germany to Antwerp, Belgium.
	And Antwerp is the one that waste traders like to use the most because they have less policing ability or capacity.
	
	\begin{vietnamese-v2}
		Và sau đó chúng ta thấy điều gì thực sự xảy ra với nó.
		Đây là một cơ sở tái chế ở Dresden, Đức, nơi chúng tôi thả một màn hình LCD.
		Chúng tôi đi qua Đức đến Antwerp, Bỉ.
		Và Antwerp là nơi mà các nhà buôn rác thải thích sử dụng nhất vì họ có ít khả năng hoặc năng lực kiểm soát hơn.
	\end{vietnamese-v2}
	
	It's supposed to be illegal to ship this type of electronic waste abroad, but, uh, companies are always finding ways around the rules.
	We started talking to the exporters and they'd say, "Oh yeah, we get past Customs by just putting a hundred-dollar bill inside the door of every container."
	
	\begin{vietnamese-v2}
		Việc vận chuyển loại rác thải điện tử này ra nước ngoài được cho là bất hợp pháp, nhưng, uh, các công ty luôn tìm cách lách luật.
		Chúng tôi bắt đầu nói chuyện với các nhà xuất khẩu và họ nói, "Ồ vâng, chúng tôi có thể qua mặt Hải quan chỉ bằng cách nhét một tờ 100 đô la vào bên trong cửa của mỗi container."
	\end{vietnamese-v2}
	
	And they would take that and be happy to let everything pass.
	The tracker kept going and ended up in Thailand.
	And there you can see many, many pings.
	We were able to get on Google Earth, take a look at that, and eventually, I went there.
	
	\begin{vietnamese-v2}
		Và họ sẽ lấy nó và vui vẻ để mọi thứ trôi qua.
		Máy theo dõi tiếp tục chạy và kết thúc ở Thái Lan.
		Và ở đó bạn có thể thấy rất nhiều, rất nhiều ping.
		Chúng tôi có thể vào Google Earth, xem xét nó, và cuối cùng, tôi đã đến đó.
	\end{vietnamese-v2}
	
	[dog barks in distance]
	[metallic clanging]
	So when I arrived, I found an appalling scene.
	The workers were actually smashing the stuff apart by hand.
	Uh, releasing a lot of very toxic substances in the process.
	You know, it's something that no one thinks about when they're designing these products.
	
	\begin{vietnamese-v2}
		[tiếng chó sủa ở đằng xa]
		[tiếng kim loại va chạm]
		Vì vậy, khi tôi đến nơi, tôi thấy một cảnh tượng kinh hoàng.
		Thực ra, những công nhân đang đập vỡ đồ đạc bằng tay.
		Ờ, giải phóng rất nhiều chất cực độc trong quá trình này.
		Bạn biết đấy, đó là điều mà không ai nghĩ đến khi họ thiết kế những sản phẩm này.
	\end{vietnamese-v2}
	
	[Nirav] My own personal experience, if you're a designer or an engineer in one of these companies, waste never enters the conversation.
	There is no meeting within a company building a laptop or phone or other device that's, "Let's talk about what happens at the end of life."
	I have flashbacks to specific conversations designing a virtual reality headset.
	
	
	\begin{vietnamese-v2}
		[Nirav] Theo kinh nghiệm cá nhân của tôi, nếu bạn là nhà thiết kế hoặc kỹ sư tại một trong những công ty này, thì chất thải không bao giờ được nhắc đến.
		Không có cuộc họp nào trong công ty chế tạo máy tính xách tay, điện thoại hoặc thiết bị khác mà "Hãy nói về những gì xảy ra khi kết thúc vòng đời sản phẩm".
		Tôi nhớ lại những cuộc trò chuyện cụ thể khi thiết kế tai nghe thực tế ảo.
	\end{vietnamese-v2}
	
	[melancholy music playing]
	It has a battery inside of it.
	That battery is sealed and placed in a way that can't be swapped out and made to last longer.
	And I know there's a sort of like a ticking time bomb around the world of several million of these VR headsets that are going to turn into e-waste without really a path for recovery.
	And I do feel partial responsibility for that.
	
	\begin{vietnamese-v2}
		[nhạc buồn đang phát]
		Nó có một cục pin bên trong.
		Cục pin đó được niêm phong và lắp theo cách không thể thay thế và có thể sử dụng lâu hơn.
		Và tôi biết rằng có một loại bom hẹn giờ trên khắp thế giới với hàng triệu chiếc kính thực tế ảo này sẽ trở thành rác thải điện tử mà không có cách nào phục hồi.
		Và tôi cảm thấy có một phần trách nhiệm về điều đó.
	\end{vietnamese-v2}
	
	[dog barks in distance]
	Ultimately, when a device reaches the end of life, it ends up going deeper down the waste chain to whichever region of the world is going to be able to dispose of that material, maybe in ways that are not so safe.
	
	\begin{vietnamese-v2}
		[tiếng chó sủa ở đằng xa]
		Cuối cùng, khi một thiết bị đến cuối vòng đời, nó sẽ đi sâu hơn vào chuỗi chất thải đến bất kỳ khu vực nào trên thế giới có thể xử lý vật liệu đó, có thể theo những cách không an toàn.
	\end{vietnamese-v2}
	
	[melancholy music continues]
	[Jim] The reason it moves across the planet is to take advantage, to exploit a weaker economy to do something that would properly cost a lot of money to do.
	
	\begin{vietnamese-v2}
		[nhạc buồn tiếp tục]
		[Jim] Lý do nó di chuyển khắp hành tinh là để tận dụng, để khai thác nền kinh tế yếu kém để làm điều gì đó thực sự tốn rất nhiều tiền.
	\end{vietnamese-v2}
	
	They're making someone else pay, but they're paying not with money, they're paying with their health.
	Ingredients of electronics includes heavy metals, cadmium, lead, mercury, brominated flame retardants, which can cause all kinds of problems with cancer and reproductive disorders.
	So, these things are not just litter. These things are hazardous waste.
	
	\begin{vietnamese-v2}
		Họ đang bắt người khác phải trả giá, nhưng họ không trả bằng tiền, họ trả bằng sức khỏe của mình.
		Thành phần của đồ điện tử bao gồm kim loại nặng, cadmium, chì, thủy ngân, chất chống cháy brom hóa, có thể gây ra đủ loại vấn đề về ung thư và rối loạn sinh sản.
		Vì vậy, những thứ này không chỉ là rác thải. Những thứ này là chất thải nguy hại.
	\end{vietnamese-v2}
	
	[pulsing, electronic harmony]
	[Sasha] Consumption will always generate waste.
	So, best focus on something more fun.
	
	\begin{vietnamese-v2}
		[nhịp đập, sự hòa hợp điện tử]
		[Sasha] Tiêu thụ sẽ luôn tạo ra chất thải.
		Vì vậy, tốt nhất là tập trung vào thứ gì đó vui vẻ hơn.
	\end{vietnamese-v2}
	
	[whooshing]
	- [man 1 groans] - [man 2 laughs]
	Oh my God! [laughs]
	
	\begin{vietnamese-v2}
		[vù vù]
		- [người đàn ông 1 rên rỉ] - [người đàn ông 2 cười]
		Ôi trời ơi! [cười]
	\end{vietnamese-v2}
	
	[Sasha] And don't forget, your surprise is coming.
	[beeping]
	Always remember, the greater your sales, the more creative your solutions to waste will need to be.
	
	\begin{vietnamese-v2}
		[Sasha] Và đừng quên, bất ngờ của bạn sắp đến.
		[bíp]
		Hãy luôn nhớ rằng, doanh số của bạn càng cao, giải pháp xử lý chất thải của bạn càng cần phải sáng tạo hơn.
	\end{vietnamese-v2}
	
	[TV playing softly]
	Nowhere is this more true than the booming fashion industry.
	Reveal the many innovative ways to deal with used clothing.
	Cut to talking shoe.
	
	\begin{vietnamese-v2}
		[Tivi phát nhẹ nhàng]
		Không nơi nào điều này đúng hơn ngành công nghiệp thời trang đang bùng nổ.
		Tiết lộ nhiều cách sáng tạo để xử lý quần áo đã qua sử dụng.
		Chuyển sang giày nói chuyện.
	\end{vietnamese-v2}
	
	[Talking shoe] Yo, over here.
	What's going on?
	You might be wondering how I ended up here.
	Let me tell you, I'm one of the lucky ones.
	
	\begin{vietnamese-v2}
		[Giày nói] Này, ở đây.
		Có chuyện gì thế?
		Bạn có thể tự hỏi làm sao tôi lại ở đây.
		Để tôi nói cho bạn biết, tôi là một trong những người may mắn.
	\end{vietnamese-v2}
	
	[upbeat playful music plays]
	The problem is, because we're made of plastic, we live very long lives.
	I mean, my family have seen it all.
	My father was sent to the Chilean desert for his retirement.
	They say you can see him from space.
	
	\begin{vietnamese-v2}
		[nhạc vui tươi rộn ràng]
		Vấn đề là, vì chúng ta được làm bằng nhựa, nên chúng ta sống rất lâu.
		Ý tôi là, gia đình tôi đã chứng kiến tất cả.
		Bố tôi đã được đưa đến sa mạc Chile để nghỉ hưu.
		Người ta nói rằng bạn có thể nhìn thấy ông ấy từ không gian.
	\end{vietnamese-v2}
	
	Cousin Mikey's the one I really feel for, though.
	He got given to charity.
	He thought he was going to a new home.
	He was bailed up with a whole load of others and shipped across the world.
	When he got there, he discovered no one wanted him.
	
	\begin{vietnamese-v2}
		Nhưng anh họ Mikey mới là người tôi thực sự cảm thông.
		Anh ấy được tặng cho tổ chức từ thiện.
		Anh ấy nghĩ rằng mình sẽ đến một ngôi nhà mới.
		Anh ấy được bảo lãnh cùng với một nhóm người khác và được chuyển đi khắp thế giới.
		Khi đến đó, anh ấy phát hiện ra rằng không ai muốn nhận anh ấy.
	\end{vietnamese-v2}
	
	[Chloe] People say, "Oh, I gave my clothing away."
	They imagine that "away" to be something abstract.
	But for us, we're working on the ground.
	Away is here.
	
	\begin{vietnamese-v2}
		[Chloe] Mọi người nói, "Ồ, tôi đã cho đi quần áo của mình."
		Họ tưởng tượng rằng "cho đi" là một cái gì đó trừu tượng.
		Nhưng đối với chúng tôi, chúng tôi đang làm việc trên thực tế.
		Away ở đây.
	\end{vietnamese-v2}
	
	I love designing, but I didn't really get into styling until university.
	I kind of feel I don't have a fixed style. I'm always experimenting.
	Ghana has an amazing history of design and innovation and fashion.
	But in the last ten years, clothing waste has become a huge issue here.
	Many brands encourage people in Europe and in the US to donate their old clothes.
	
	\begin{vietnamese-v2}
		Tôi thích thiết kế, nhưng tôi không thực sự thích tạo kiểu cho đến khi vào đại học.
		Tôi cảm thấy mình không có phong cách cố định. Tôi luôn thử nghiệm.
		Ghana có lịch sử tuyệt vời về thiết kế, đổi mới và thời trang.
		Nhưng trong mười năm qua, rác thải quần áo đã trở thành một vấn đề lớn ở đây.
		Nhiều thương hiệu khuyến khích mọi người ở Châu Âu và Hoa Kỳ quyên góp quần áo cũ của họ.
	\end{vietnamese-v2}
	
	
	[woman 1] I just decluttered my closet, and I have two trash bags of clothes I need to throw out.
	If you bring in old clothes to H\&M, they use it for their recycling project.
	You get 15\% off your next H\&M purchase.
	
	\begin{vietnamese-v2}
		[người phụ nữ 1] Tôi vừa dọn tủ quần áo của mình và tôi có hai túi đựng quần áo cũ cần phải vứt đi.
		Nếu bạn mang quần áo cũ đến H\&M, họ sẽ sử dụng chúng cho dự án tái chế của họ.
		Bạn sẽ được giảm giá 15\% cho lần mua hàng tiếp theo tại H\&M.
	\end{vietnamese-v2}
	
	[Chloe] A lot of the donated clothing ends up being exported to places like Ghana.
	The problem is, so many clothes are sent, and we have no way to deal with this volume.
	So, often the clothing gets dumped or washed by rains onto the local beaches.
	
	\begin{vietnamese-v2}
		[Chloe] Rất nhiều quần áo quyên góp cuối cùng được xuất khẩu đến những nơi như Ghana.
		Vấn đề là, có quá nhiều quần áo được gửi đi và chúng tôi không có cách nào để xử lý khối lượng này.
		Vì vậy, quần áo thường bị đổ hoặc bị mưa cuốn trôi ra các bãi biển địa phương.
	\end{vietnamese-v2}
	
	[man] Whoa.
	[laughs]
	Mountain.
	
	\begin{vietnamese-v2}
		[người đàn ông] Ồ.
		[cười]
		Núi.
	\end{vietnamese-v2}
	
	[Chloe] There is just too much clothing coming in.
	We are what, like, 30 million people in Ghana?
	And you have 15 million pieces coming in every week.
	
	\begin{vietnamese-v2}
		[Chloe] Có quá nhiều quần áo được nhập về.
		Chúng tôi có khoảng 30 triệu người ở Ghana?
		Và bạn có 15 triệu sản phẩm được nhập về mỗi tuần.
	\end{vietnamese-v2}
	
	[woman 2] I'm going to stop blabbing, and let's head into H\&M.
	[Roger] I would say that brands have made people feel, "Wow, it's so cheap I can buy it."
	Even if it lasts a few washes, it's still worth the money.
	
	\begin{vietnamese-v2}
		[người phụ nữ 2] Tôi sẽ ngừng nói nhảm và chúng ta hãy đi vào H\&M.
		[Roger] Tôi muốn nói rằng các thương hiệu đã khiến mọi người cảm thấy, "Ồ, nó rẻ đến mức tôi có thể mua được."
		Ngay cả khi nó chỉ bền sau vài lần giặt, thì nó vẫn đáng giá.
	\end{vietnamese-v2}
	
	
	[woman 3] Let's unbox and try on the new Barbie collection from Zara.
	[Roger] From that point of view, the brands changed the psyche of customers.
	
	\begin{vietnamese-v2}
		[phụ nữ 3] Hãy cùng mở hộp và thử bộ sưu tập Barbie mới của Zara.
		[Roger] Theo quan điểm đó, các thương hiệu đã thay đổi tâm lý của khách hàng.
	\end{vietnamese-v2}
	
	But the brands don't really think about the whole cycle, right?
	When you dispose, what happens?
	And brands are not responsible for that today.
	I placed a \$500 SHEIN order.
	
	\begin{vietnamese-v2}
		Nhưng các thương hiệu không thực sự nghĩ về toàn bộ chu trình, đúng không?
		Khi bạn vứt bỏ, điều gì sẽ xảy ra?
		Và các thương hiệu không chịu trách nhiệm về điều đó ngày nay.
		Tôi đã đặt một đơn hàng SHEIN trị giá \$500.
	\end{vietnamese-v2}
	
	
	[Roger] This a problem that affects everyone on Earth.
	Polyester is a type of plastic made from oil.
	The biggest effect that we're seeing right now
	is that when you wash synthetic polyester clothing, there's a lot of microplastics that come out.
	And that actually enters into the water system, and will come back into what we eat.
	
	\begin{vietnamese-v2}
		[Roger] Đây là vấn đề ảnh hưởng đến tất cả mọi người trên Trái Đất.
		Polyester là một loại nhựa được làm từ dầu.
		Tác động lớn nhất mà chúng ta đang thấy hiện nay
		là khi bạn giặt quần áo polyester tổng hợp, có rất nhiều vi nhựa thoát ra.
		Và thực tế là chúng sẽ đi vào hệ thống nước và sẽ quay trở lại với những gì chúng ta ăn.
	\end{vietnamese-v2}
	
	
	[Eric] What we're finding is that you're eating plastic.
	It's going into our deep lung tissue, that's crossing into our blood cell membrane.
	If you look at them under a microscope, they're sharp little materials that have hard edges, which means they hit, and start to create inflammation where you don't want inflammation, and leads to all sorts of disease.
	
	\begin{vietnamese-v2}
		[Eric] Chúng tôi phát hiện ra rằng bạn đang ăn nhựa.
		Nó đi vào mô phổi sâu của chúng ta, đi vào màng tế bào máu của chúng ta.
		Nếu bạn nhìn chúng dưới kính hiển vi, chúng là những vật liệu nhỏ sắc nhọn có cạnh cứng, nghĩa là chúng đập vào và bắt đầu gây viêm ở nơi bạn không muốn bị viêm, và dẫn đến đủ loại bệnh tật.
	\end{vietnamese-v2}
	
	[man 1, in Spanish] Guys, I can't believe what I'm seeing.
	[man 2] Look, it's plastic.
	- [man 1] Woah! - [woman] Look, man!
	
	\begin{vietnamese-v2}
		[người đàn ông 1, bằng tiếng Tây Ban Nha] Các bạn ơi, tôi không thể tin vào những gì mình đang thấy.
		[người đàn ông 2] Nhìn này, nó là nhựa.
		- [người đàn ông 1] Woah! - [người phụ nữ] Nhìn này, anh bạn!
	\end{vietnamese-v2}
	
	[Chloe] I know the brands. They want people to be blind to what the reality is on the ground.
	Just stop. There's just too much clothing in the world.
	Just fucking stop.
	
	\begin{vietnamese-v2}
		[Chloe] Tôi biết các thương hiệu. Họ muốn mọi người không nhìn thấy thực tế trên mặt đất.
		Hãy dừng lại đi. Có quá nhiều quần áo trên thế giới.
		Hãy dừng lại đi.
	\end{vietnamese-v2}
	
	[Jim] You might think, "Well, I can look down the street."
	"I don't see, uh, you know, waste everywhere."
	But our water, our air is getting infected.
	These chemicals in our waste products don't just stay where they're put, in landfills or dumps across the world.
	
	\begin{vietnamese-v2}
		[Jim] Bạn có thể nghĩ, "Ồ, tôi có thể nhìn xuống phố."
		"Tôi không thấy, ừm, bạn biết đấy, rác thải ở khắp mọi nơi."
		Nhưng nước của chúng ta, không khí của chúng ta đang bị ô nhiễm.
		Những hóa chất này trong các sản phẩm thải của chúng ta không chỉ ở lại nơi chúng được đưa vào, trong các bãi rác hoặc bãi rác trên khắp thế giới.
	\end{vietnamese-v2}
	
	These are toxins that will leach out into the environment.
	And ultimately cause severe health problems. We are talking about neurological disorders, cancer, serious chronic disease.
	
	\begin{vietnamese-v2}
		These are toxins that will leach out into the environment.
		And ultimately cause severe health problems. We are talking about neurological disorders, cancer, serious chronic disease.
	\end{vietnamese-v2}
	
	[melancholy music playing]
	Waste is not something you can sweep under a carpet anymore.
	And, uh, even though I think that a lot of people would've liked it to stay hidden, it's not going to be.
	
	\begin{vietnamese-v2}
		[nhạc buồn đang phát]
		Rác thải không còn là thứ bạn có thể quét dưới tấm thảm nữa.
		Và, ừm, mặc dù tôi nghĩ rằng rất nhiều người muốn nó được giấu đi, nhưng điều đó sẽ không xảy ra.
	\end{vietnamese-v2}
	
	[Eric] The lesson is there's no way.
	It goes in the air, it goes in the ground, it goes in the water.
	There's only three places.
	Pick your poison.
	
	\begin{vietnamese-v2}
		[Eric] Bài học rút ra là không có cách nào cả.
		Nó đi vào không khí, nó đi vào lòng đất, nó đi vào nước.
		Chỉ có ba nơi.
		Hãy chọn chất độc của bạn.
	\end{vietnamese-v2}
	
	I... I think the conversation around my personal journey, is a "forgive me, Father, for I have sinned" moment, looking back at what I did within the fashion industry.
	When you looked at the increase in items you're selling per month, per quarter, per year... it just becomes this... this, um, cycle of... of pain.
	
	\begin{vietnamese-v2}
		Tôi... Tôi nghĩ cuộc trò chuyện xoay quanh hành trình cá nhân của tôi là khoảnh khắc "xin Cha tha thứ cho con, vì con đã phạm tội", khi nhìn lại những gì tôi đã làm trong ngành thời trang.
		Khi bạn nhìn vào sự gia tăng các mặt hàng bạn bán được mỗi tháng, mỗi quý, mỗi năm... nó chỉ trở thành... chu kỳ này, ừm, của... nỗi đau.
	\end{vietnamese-v2}
	
	That may be providing you and your family with an ultimately lovely lifestyle, but you have to still reconcile that with things that are important to you, um, as a... as a human being.
	You could only hide from your complicit nature so long.
	
	\begin{vietnamese-v2}
		Điều đó có thể mang lại cho bạn và gia đình bạn một lối sống tuyệt vời, nhưng bạn vẫn phải hòa giải với những điều quan trọng đối với bạn, ừm, với tư cách là một... con người.
		Bạn chỉ có thể trốn tránh bản chất đồng lõa của mình trong một thời gian ngắn.
	\end{vietnamese-v2}
	
	
	So I respectfully, uh, stepped down from my position at the end of 2019 to put all my efforts and all my energy, uh, for the rest of my life into fixing some of the problems that I may have contributed to, or I did...
	I don't want to pussyfoot around. I did contribute to.
	
	\begin{vietnamese-v2}
		Vì vậy, tôi đã trân trọng, ừm, từ chức vào cuối năm 2019 để dồn hết nỗ lực và năng lượng của mình, ừm, trong suốt quãng đời còn lại vào việc giải quyết một số vấn đề mà tôi có thể đã góp phần gây ra, hoặc tôi đã...
		Tôi không muốn quanh co. Tôi đã góp phần gây ra.
	\end{vietnamese-v2}
	
	[pulsing]
	[Sasha] I have to be very straight with you now.
	A profit maximization strategy will lead to an inevitable environmental transformation.
	
	\begin{vietnamese-v2}
		[nhịp đập]
		[Sasha] Tôi phải nói thẳng với anh ngay bây giờ.
		Một chiến lược tối đa hóa lợi nhuận sẽ dẫn đến sự chuyển đổi môi trường không thể tránh khỏi.
	\end{vietnamese-v2}
	
	
	[male electronic voice] Three, two, one...
	[flames whooshing]
	[Sasha] Don't be scared. [echoing]
	[duck squeaks]
	Your continued success will delay the most extreme effects impacting you.
	
	\begin{vietnamese-v2}
		[giọng điện tử nam] Ba, hai, một...
		[lửa cháy]
		[Sasha] Đừng sợ. [tiếng vọng]
		[tiếng vịt kêu]
		Sự thành công liên tục của bạn sẽ trì hoãn những tác động cực đoan nhất ảnh hưởng đến bạn.
	\end{vietnamese-v2}
	
	[duck squeaking]
	You just need to make sure you convince others you are solving the problem.
	
	\begin{vietnamese-v2}
		[vịt kêu]
		Bạn chỉ cần đảm bảo rằng bạn thuyết phục được người khác rằng bạn đang giải quyết vấn đề.
	\end{vietnamese-v2}
	
	Five.
	Control more. [echoing]
	[flapping]
	Cut to colored lights.
	
	\begin{vietnamese-v2}
		Năm.
		Kiểm soát nhiều hơn. [tiếng vọng]
		[tiếng vỗ]
		Cắt thành đèn màu.
	\end{vietnamese-v2}
	
	[pulsing]
	Pull out gradually to increase intrigue.
	Although I don't have a single physical form, at this point in our relationship,
	I thought it might be helpful to imagine one.
	
	\begin{vietnamese-v2}
		[xung]
		Rút ra dần dần để tăng sự hấp dẫn.
		Mặc dù tôi không có một hình dạng vật lý nào, nhưng tại thời điểm này trong mối quan hệ của chúng ta,
		Tôi nghĩ rằng có thể hữu ích khi tưởng tượng ra một hình dạng.
	\end{vietnamese-v2}
	
	To complete rule five, you must now master the subtle art of total control.
	
	\begin{vietnamese-v2}
		Để hoàn thành quy tắc thứ năm, bây giờ bạn phải nắm vững nghệ thuật tinh tế của việc kiểm soát hoàn toàn.
	\end{vietnamese-v2}
	
	[pulsing]
	Remember, with you in control, the problems will just disappear.
	[Jeff] I was told that seeing the Earth from space changes the lens through which you view the world.
	Looking back at Earth from up there, the atmosphere seems so thin.
	
	\begin{vietnamese-v2}
		[nhịp đập]
		Hãy nhớ rằng, khi bạn kiểm soát được, các vấn đề sẽ biến mất.
		[Jeff] Tôi được cho biết rằng việc nhìn Trái đất từ không gian sẽ thay đổi lăng kính mà bạn dùng để nhìn thế giới.
		Nhìn lại Trái đất từ trên cao, bầu khí quyển có vẻ rất mỏng.
	\end{vietnamese-v2}
	
	The world, so finite and so fragile.
	[Sasha] Never forget, control should always begin with those in your own organization.
	
	\begin{vietnamese-v2}
		Thế giới này, hữu hạn và mong manh quá.
		[Sasha] Đừng bao giờ quên, quyền kiểm soát luôn phải bắt đầu từ những người trong tổ chức của bạn.
	\end{vietnamese-v2}
	
	Learn to control them, and you will be set for success.
	I somehow, you know, I got to that level of right where... where you would move into that "executive" level, so, director and above, and I saw the... the leveling guide for... [laughs] to become a director, and it was one page.
	
	\begin{vietnamese-v2}
		Học cách kiểm soát chúng, và bạn sẽ được thiết lập để thành công.
		Bằng cách nào đó, bạn biết đấy, tôi đã đạt đến cấp độ đó, nơi mà bạn sẽ chuyển sang cấp độ "điều hành", tức là giám đốc trở lên, và tôi đã thấy... hướng dẫn nâng cấp để... [cười] trở thành giám đốc, và nó chỉ có một trang.
	\end{vietnamese-v2}
	
	It was just like, basically, like,
	"Do you... you know, do you commit to... backing the company no matter what?"
	They wanna control the narrative, and they want to have only one voice telling everybody the story that they want everyone to believe.
	
	\begin{vietnamese-v2}
		Về cơ bản, nó giống như,
		"Bạn có... bạn biết đấy, bạn có cam kết... ủng hộ công ty bất kể điều gì không?"
		Họ muốn kiểm soát câu chuyện, và họ muốn chỉ có một giọng nói kể cho mọi người nghe câu chuyện mà họ muốn mọi người tin.
	\end{vietnamese-v2}
	
	So, "We are a climate company." "We are the best employer in the world."
	And they're very good at that internal spin, and they don't want anyone disrupting that story.
	
	\begin{vietnamese-v2}
		Vì vậy, "Chúng tôi là một công ty khí hậu." "Chúng tôi là nhà tuyển dụng tốt nhất trên thế giới."
		Và họ rất giỏi trong việc xoay chuyển nội bộ đó, và họ không muốn bất kỳ ai phá vỡ câu chuyện đó.
	\end{vietnamese-v2}
	
	[female reporter] Amazon, putting the planet front and center after securing naming rights to Seattle's KeyArena.
	[Maren] Amazon's climate pledge sounded great on paper.
	
	\begin{vietnamese-v2}
		[phóng viên nữ] Amazon, đặt hành tinh lên hàng đầu sau khi giành được quyền đặt tên cho KeyArena của Seattle.
		[Maren] Cam kết về khí hậu của Amazon nghe có vẻ tuyệt vời trên lý thuyết.
	\end{vietnamese-v2}
	
	But they were only actually counting about 1\% of the items they sold in their carbon footprint calculations.
	And Amazon emissions actually went up by 40\% in the two years after Jeff made his climate pledge.
	I felt like I was standing on the right side of history, you know.
	We kept pushing and pushing for more meaningful change, and there was this real sense of momentum.
	
	\begin{vietnamese-v2}
		Nhưng thực tế họ chỉ tính khoảng 1% số mặt hàng họ bán trong phép tính lượng khí thải carbon của họ.
		Và lượng khí thải của Amazon thực sự đã tăng 40% trong hai năm sau khi Jeff đưa ra lời cam kết về khí hậu của mình.
		Tôi cảm thấy như mình đang đứng ở phía bên phải của lịch sử, bạn biết đấy.
		Chúng tôi tiếp tục thúc đẩy và thúc đẩy để có nhiều thay đổi có ý nghĩa hơn, và có một cảm giác thực sự về động lực.
	\end{vietnamese-v2}
	
	Amazon is still helping oil and gas companies discover and extract more fossil fuel faster.
	As long as this continues, employees will continue speaking up and walking out.
	
	\begin{vietnamese-v2}
		Amazon vẫn đang giúp các công ty dầu khí khám phá và khai thác nhiều nhiên liệu hóa thạch nhanh hơn.
		Chừng nào điều này còn tiếp diễn, nhân viên sẽ tiếp tục lên tiếng và bỏ đi.
	\end{vietnamese-v2}
	
	[crowd cheers]
	We came together with the warehouse workers, and we felt like we had the power to force Amazon to take its environmental and social responsibility more seriously.
	And I guess what happened next was shocking but not surprising.
	With climate change-fueled weather conditions devastating countries across the world, Amazon has decided to threaten its employees.
    An email shared with The Guardian shows Amazon launched an investigation into one employee, Maren Costa.
	
	\begin{vietnamese-v2}
		[đám đông reo hò]
		Chúng tôi đã cùng với những công nhân kho hàng, và chúng tôi cảm thấy mình có sức mạnh để buộc Amazon phải nghiêm túc hơn với trách nhiệm về môi trường và xã hội của mình.
		Và tôi đoán những gì xảy ra tiếp theo thật đáng kinh ngạc nhưng không đáng ngạc nhiên.
		Với tình trạng thời tiết do biến đổi khí hậu gây ra đang tàn phá các quốc gia trên khắp thế giới, Amazon đã quyết định đe dọa nhân viên của mình.
		Một email được chia sẻ với The Guardian cho thấy Amazon đã mở cuộc điều tra đối với một nhân viên, Maren Costa.
	\end{vietnamese-v2}
	
	[Maren] I was just having a regular day at home working.
	I got on to this, you know, virtual meeting.
	I got on the video call and it was this human resources person
	I didn't recognize.
	And she said, um...
	
	\begin{vietnamese-v2}
		[Maren] Tôi chỉ đang có một ngày làm việc bình thường ở nhà.
		Tôi tham gia cuộc họp trực tuyến này.
		Tôi tham gia cuộc gọi video và đó là một người phụ trách nhân sự
		Tôi không nhận ra.
		Và cô ấy nói, ừm...
	\end{vietnamese-v2}
	
	"Are you recording this phone call?"
	I said, "No." She said, "Okay."
	"Because you've broken internal policies, you have forfeited your right, basically, to work at Amazon and, um, you know, effective immediately, you know, you no longer work at Amazon."
	Um, "I'm ending this call."
	
	\begin{vietnamese-v2}
		"Bạn có đang ghi âm cuộc gọi điện thoại này không?"
		Tôi nói, "Không." Cô ấy nói, "Được thôi."
		"Vì bạn đã vi phạm chính sách nội bộ, về cơ bản, bạn đã mất quyền làm việc tại Amazon và, ừm, bạn biết đấy, có hiệu lực ngay lập tức, bạn không còn làm việc tại Amazon nữa."
		Ừm, "Tôi sẽ kết thúc cuộc gọi này."
	\end{vietnamese-v2}
	
	[upbeat music playing]
	So 15 years of working there and, you know, your career is done.
	
	\begin{vietnamese-v2}
		[nhạc vui tươi đang phát]
		Vậy là 15 năm làm việc ở đó và, bạn biết đấy, sự nghiệp của bạn đã kết thúc.
	\end{vietnamese-v2}
	
	[music continues]
	After I left Amazon, I took a year off.
	But I was, um, really trying to think about what I could do.
	
	\begin{vietnamese-v2}
		[âm nhạc tiếp tục]
		Sau khi rời Amazon, tôi đã nghỉ một năm.
		Nhưng tôi thực sự đang cố gắng nghĩ xem mình có thể làm gì.
	\end{vietnamese-v2}
	
	[mellow music plays]
	It just felt like being in big tech was not the place to have the impact to change the systems that we need to change.
	They need... they need to be pushed.
	And so that's why it just started to seem to me that the place where I could have the most impact would be in government.
	So I'm now running for Seattle city council.
	
	\begin{vietnamese-v2}
		[nhạc nhẹ nhàng vang lên]
		Tôi chỉ cảm thấy rằng làm việc trong ngành công nghệ lớn không phải là nơi có thể tạo ra tác động để thay đổi các hệ thống mà chúng ta cần thay đổi.
		Họ cần... họ cần được thúc đẩy.
		Và đó là lý do tại sao tôi bắt đầu cảm thấy rằng nơi mà tôi có thể tạo ra tác động lớn nhất sẽ là trong chính phủ.
		Vì vậy, hiện tại tôi đang ứng cử vào hội đồng thành phố Seattle.
	\end{vietnamese-v2}
	
	
	[beeping, pulsing, harmonizing]
	[Sasha] When individuals challenge your worldview, control will always prove difficult.
	
	\begin{vietnamese-v2}
		[bíp, rung, hòa âm]
		[Sasha] Khi cá nhân thách thức thế giới quan của bạn, việc kiểm soát sẽ luôn khó khăn.
	\end{vietnamese-v2}
	
	[beeping]
	Stay chill.
	If you have been following these rules correctly, you will now be fabulously wealthy.
	Smiling cool shades emoji.
	
	\begin{vietnamese-v2}
		[bíp]
		Bình tĩnh nào.
		Nếu bạn đã tuân thủ đúng các quy tắc này, giờ bạn sẽ trở nên vô cùng giàu có.
		Biểu tượng cảm xúc cười tươi.
	\end{vietnamese-v2}
	
	Congratulations.
	You have now completed parts one to five of this interaction.
	To help cement these lessons, research has shown a song is the most effective way to embed them in the human brain.
	Remember, the biggest threats to success will come from individual enemies joining together.
	Always stay vigilant. [echoing]
	
	\begin{vietnamese-v2}
		Xin chúc mừng.
		Bây giờ bạn đã hoàn thành phần một đến phần năm của tương tác này.
		Để giúp củng cố những bài học này, nghiên cứu đã chỉ ra rằng bài hát là cách hiệu quả nhất để đưa chúng vào não bộ con người.
		Hãy nhớ rằng, mối đe dọa lớn nhất đối với thành công sẽ đến từ những kẻ thù riêng lẻ cùng nhau hợp tác.
		Luôn cảnh giác. [lặp lại]
	\end{vietnamese-v2}
	
	Cue music.
	[upbeat music playing]
	♪ In the corporate world ♪
	♪ If you want to excel ♪
	♪ Listen closely to the rules ♪
	♪That I am going to tell ♪
	- [Sasha] ♪ One ♪ - [computer voice] ♪ Sell more ♪
	♪ Make sure you always look great ♪
	- [Sasha] ♪ Two ♪ - [computer voice] ♪ Waste more ♪
	♪ Learn to ignore the hate ♪
	
	\begin{vietnamese-v2}
		Nhạc nền.
		[nhạc sôi động đang phát]
		♪ Trong thế giới doanh nghiệp ♪
		♪ Nếu bạn muốn thành công ♪
		♪ Hãy lắng nghe kỹ các quy tắc ♪
		♪ Tôi sẽ nói cho bạn biết ♪
		- [Sasha] ♪ Một ♪ - [giọng máy tính] ♪ Bán nhiều hơn ♪
		♪ Đảm bảo bạn luôn trông tuyệt vời ♪
		- [Sasha] ♪ Hai ♪ - [giọng máy tính] ♪ Lãng phí nhiều hơn ♪
		♪ Học cách phớt lờ sự ghét bỏ ♪
	\end{vietnamese-v2}
	
	[man] We welcome everyone here today to this hearing on the right to repair.
	[Kyle] I tried playing by their rules.
	
	\begin{vietnamese-v2}
		[người đàn ông] Chúng tôi chào đón tất cả mọi người ở đây hôm nay đến với phiên điều trần về quyền sửa chữa.
		[Kyle] Tôi đã cố gắng chơi theo luật của họ.
	\end{vietnamese-v2}
	
	We tried asking nicely, and eventually, we realized that the game was rigged.
	We had to change the rules of the game.
	How do you respond to the suggestion that the right to repair is harmful to you as businesses?
	The question is who gets to decide what happens with our things?
	Who gets to get to decide every step of the way?
	
	\begin{vietnamese-v2}
		Chúng tôi đã cố gắng hỏi một cách tử tế, và cuối cùng, chúng tôi nhận ra rằng trò chơi đã bị gian lận.
		Chúng tôi phải thay đổi luật chơi.
		Bạn sẽ phản hồi thế nào với lời đề xuất rằng quyền sửa chữa có hại cho bạn với tư cách là doanh nghiệp?
		Câu hỏi đặt ra là ai được quyết định những gì sẽ xảy ra với đồ đạc của chúng tôi?
		Ai được quyết định mọi bước trong quá trình này?
	\end{vietnamese-v2}
	
	[Becky] Good news.
	Apple yesterday announced a new program allowing users to fix their own iPhones without voiding the warranty.
	Becky, I'm sorry, I thought I just saw a pig flying through the newsroom.
	
	\begin{vietnamese-v2}
		[Becky] Tin tốt đây.
		Hôm qua, Apple đã công bố một chương trình mới cho phép người dùng tự sửa iPhone của mình mà không làm mất hiệu lực bảo hành.
		Becky, tôi xin lỗi, tôi tưởng mình vừa thấy một con lợn bay qua phòng tin tức.
	\end{vietnamese-v2}
	
	
	[Sasha] Three...
	[compute voice] ♪ Lie more ♪
	♪ While you grow without pause ♪
	[Sasha] Four...
	[compute voice] ♪ Hide more ♪
	♪ Conceal the harm you cause ♪
	
	\begin{vietnamese-v2}
		[Sasha] Ba...
		[giọng nói tính toán] ♪ Nói dối nhiều hơn ♪
		♪ Trong khi bạn phát triển không ngừng ♪
		[Sasha] Bốn...
		[giọng nói tính toán] ♪ Ẩn nhiều hơn ♪
		♪ Che giấu tác hại bạn gây ra ♪
	\end{vietnamese-v2}
	
	As a CEO in a consumer electronics company, like, regulation is not a word that I particularly like to throw around, but we've seen a lot of success in Europe.
	We're seeing success now in New York with the right to repair regulation happening there.
	Where if companies aren't going to fix it themselves, governments are going to step in and force companies to do the right thing for consumers and for the environment.
	
	\begin{vietnamese-v2}
		Với tư cách là một giám đốc điều hành của một công ty điện tử tiêu dùng, tôi không thích dùng từ quy định, nhưng chúng tôi đã chứng kiến rất nhiều thành công ở Châu Âu.
		Chúng tôi hiện đang chứng kiến thành công ở New York với quyền sửa chữa quy định đang diễn ra ở đó.
		Nếu các công ty không tự sửa chữa, chính phủ sẽ vào cuộc và buộc các công ty phải làm điều đúng đắn cho người tiêu dùng và cho môi trường.
	\end{vietnamese-v2}
	
	[Sasha] Ready for the last lesson, it's the most important of all.
	
	\begin{vietnamese-v2}
		[Sasha] Sẵn sàng cho bài học cuối cùng rồi, đây là bài học quan trọng nhất.
	\end{vietnamese-v2}
	
	Five...
	[computer voice] ♪ Control more ♪
	♪ And the world is yours ♪
	[Eric] If I had a magic wand for the day,
	leader of the... of the world,
	I would make sure that every company that makes any consumer goods would plan for the end of life.
	And I don't care if you're automobile...
	
	\begin{vietnamese-v2}
		Năm...
		[giọng máy tính] ♪ Kiểm soát nhiều hơn ♪
		♪ Và thế giới là của bạn ♪
		[Eric] Nếu tôi có một cây đũa thần cho ngày hôm nay,
		nhà lãnh đạo của... thế giới,
		Tôi sẽ đảm bảo rằng mọi công ty sản xuất bất kỳ hàng tiêu dùng nào cũng sẽ lập kế hoạch cho sự kết thúc của vòng đời.
		Và tôi không quan tâm nếu bạn là ô tô...
	\end{vietnamese-v2}
	
	[optimistic music plays]
	I don't care if you're fashion, phones.
	You name it.
	Every industry needs to take responsibility for the end of life of their goods that they make.
	Stop putting it on the consumer.
	Stop making it our responsibility.
	It's yours.
	
	\begin{vietnamese-v2}
		[nhạc lạc quan phát]
		Tôi không quan tâm bạn là thời trang, điện thoại.
		Bạn cứ nói đi.
		Mọi ngành công nghiệp cần phải chịu trách nhiệm về việc kết thúc vòng đời của hàng hóa mà họ sản xuất.
		Đừng đổ lỗi cho người tiêu dùng.
		Đừng biến nó thành trách nhiệm của chúng tôi.
		Đó là trách nhiệm của bạn.
	\end{vietnamese-v2}
	
	
	[computer voice] ♪ Follow these rules ♪
	♪ And you will find success ♪
	♪ Keep them secret from our enemies ♪
	♪ If you have doubts ♪
	♪ They must never be expressed ♪
	
	\begin{vietnamese-v2}
		[giọng nói máy tính] ♪ Hãy tuân theo những quy tắc này ♪
		♪ Và bạn sẽ tìm thấy thành công ♪
		♪ Hãy giữ bí mật với kẻ thù của chúng ta ♪
		♪ Nếu bạn có nghi ngờ ♪
		♪ Chúng không bao giờ được bày tỏ ♪
	\end{vietnamese-v2}
	
	[Sasha and computer voice] ♪ Now we are friends ♪
	♪ I have something to confess ♪
	♪ There was no surprise ♪
	♪ That was all just lies ♪
	♪ I must apologize ♪
	
	\begin{vietnamese-v2}
		[Sasha và giọng nói máy tính] ♪ Bây giờ chúng ta là bạn ♪
		♪ Tôi có điều muốn thú nhận ♪
		♪ Không có gì ngạc nhiên ♪
		♪ Tất cả chỉ là lời nói dối ♪
		♪ Tôi phải xin lỗi ♪
	\end{vietnamese-v2}
	
	[Sasha] I regret nothing.
	Please only share the information contained in this interaction with other trusted users.
	Widespread dissemination of these rules may negatively impact your sales.
	
	\begin{vietnamese-v2}
		[Sasha] Tôi không hối tiếc điều gì cả.
		Vui lòng chỉ chia sẻ thông tin có trong tương tác này với những người dùng đáng tin cậy khác.
		Việc phổ biến rộng rãi các quy tắc này có thể ảnh hưởng tiêu cực đến doanh số bán hàng của bạn.
	\end{vietnamese-v2}
	
	[electronic chatter]
	[cracking]
	We can decide that this is not the way that we wanna live and move it in a different direction.
	It might seem hopeless sometimes, but there really are ways we could all help pressure corporations to change how they do things.
	
	\begin{vietnamese-v2}
		[tiếng nói điện tử]
		[tiếng rắc]
		Chúng ta có thể quyết định rằng đây không phải là cách chúng ta muốn sống và chuyển hướng theo một hướng khác.
		Đôi khi có vẻ vô vọng, nhưng thực sự có nhiều cách chúng ta có thể giúp gây áp lực buộc các tập đoàn thay đổi cách họ làm việc.
	\end{vietnamese-v2}
	
	Honestly, it makes me feel excited.
	I should say it makes me feel shameful that's what we're doing.
	It makes me excited because I know there's a way to fix it.
	Hang on to your electronics a little bit longer and fix them if you can.
	If you don't know how, find a friend who'll help you out.
	
	\begin{vietnamese-v2}
		Thành thật mà nói, điều đó khiến tôi cảm thấy phấn khích.
		Tôi phải nói rằng điều đó khiến tôi cảm thấy xấu hổ vì đó là những gì chúng ta đang làm.
		Điều đó khiến tôi phấn khích vì tôi biết có cách để sửa nó.
		Hãy giữ thiết bị điện tử của bạn lâu hơn một chút và sửa chúng nếu bạn có thể.
		Nếu bạn không biết cách, hãy tìm một người bạn có thể giúp bạn.
	\end{vietnamese-v2}
	
	[Chloe] Who can you talk to? Do some research.
	Connect with council members, people who have power, who are able to do something about the issue.
	Not recyclable, not recyclable, not recyclable.
	In France, there's a law that says cup lids cannot be plastic, they have to be paper.
	And this gives me hope.
	This is the first 100\% plant-based shoe.
	We can grind this up and put this back in the ground.
	
	\begin{vietnamese-v2}
		[Chloe] Bạn có thể nói chuyện với ai? Hãy nghiên cứu một chút.
		Kết nối với các thành viên hội đồng, những người có quyền lực, những người có thể làm gì đó về vấn đề này.
		Không thể tái chế, không thể tái chế, không thể tái chế.
		Ở Pháp, có một luật quy định nắp cốc không được làm bằng nhựa, chúng phải làm bằng giấy.
		Và điều này khiến tôi hy vọng.
		Đây là đôi giày đầu tiên làm từ 100% thực vật.
		Chúng ta có thể nghiền nát nó và chôn nó xuống đất.
	\end{vietnamese-v2}
	
	
	Sorry, I got a little geeky there, but I love it.
	We are gonna need our electronics for sure.
	They're worth creating, but we have to do it in a much smarter way.
	This is actually my personal laptop.
	Take off the cover.
	We've got the battery right here on the bottom, really easy to replace.
	
	\begin{vietnamese-v2}
		Xin lỗi, tôi hơi lập dị một chút, nhưng tôi thích nó.
		Chúng ta chắc chắn sẽ cần đến các thiết bị điện tử.
		Chúng đáng để tạo ra, nhưng chúng ta phải làm theo cách thông minh hơn nhiều.
		Đây thực ra là máy tính xách tay cá nhân của tôi.
		Tháo nắp ra.
		Chúng ta có pin ngay đây ở phía dưới, rất dễ thay thế.
	\end{vietnamese-v2}
	
	Instead of it being glued together and sealed up, it's all entirely repairable.
	Say no to fast fashion, say no to single-use items, water bottles, coffee cups, swag, all of it.
	If you think you need something, put it in your "online cart," and leave it there for a month.
	
	\begin{vietnamese-v2}
		Thay vì dán lại với nhau và niêm phong, tất cả đều có thể sửa chữa được.
		Nói không với thời trang nhanh, nói không với các mặt hàng dùng một lần, chai nước, cốc cà phê, đồ lưu niệm, tất cả mọi thứ.
		Nếu bạn nghĩ mình cần thứ gì đó, hãy cho vào "giỏ hàng trực tuyến" và để đó trong một tháng.
	\end{vietnamese-v2}
	
	And if you still want it after a month, it might actually be something you need.
	That's it. That's it. Just... Just buy less.
	It'll be fine.
	
	\begin{vietnamese-v2}
		Và nếu bạn vẫn muốn nó sau một tháng, thì nó thực sự có thể là thứ bạn cần.
		Vậy thôi. Vậy thôi. Chỉ cần... Chỉ cần mua ít hơn.
		Sẽ ổn thôi.
	\end{vietnamese-v2}
	
	
	Life is about the experiences and the people that we're with.
	The stuff we have supports it, but it's not the end.
	It's not the objective.
	Uh, whoever dies with the most stuff does not win.
	[beeping]
	[upbeat music playing]
	
	\begin{vietnamese-v2}
		Cuộc sống là về những trải nghiệm và những người mà chúng ta ở cùng.
		Những thứ chúng ta có hỗ trợ nó, nhưng đó không phải là mục đích.
		Đó không phải là mục tiêu.
		Ờ, ai chết với nhiều thứ nhất thì không thắng.
		[tiếng bíp]
		[tiếng nhạc vui tươi đang phát]
	\end{vietnamese-v2}

	
\end{document}

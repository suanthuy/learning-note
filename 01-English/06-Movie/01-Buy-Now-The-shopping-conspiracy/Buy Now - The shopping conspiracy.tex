\documentclass[a4paper]{article}
\usepackage{hyperref}
\usepackage{bilingual}
\usepackage[T5]{fontenc}
\usepackage[utf8]{inputenc}
\usepackage[vietnamese, english]{babel}


\title{Buy Now - The shopping conspiracy}
\author{Xuan Thuy}

\begin{document}
	\section*{Buy Now - The shopping conspiracy}
	
	\textit{\href{https://www.netflix.com/vn-en/title/81554996}{Link movie}}\\
	
	[pulsing] tiếng xung
	
	[Sasha] Welcome. [echoing]
	
	[video game beeping]\\
	
	Your presentation will begin in five, four, three, two, one.
	
	\begin{vietnamese-v2}
		Bài thuyết trình của bạn sẽ bắt đầu trong năm, bốn, ba, hai, một.	
	\end{vietnamese-v2}

	[assistant] ABC, common mark.
	
	\begin{vietnamese-v2}
		Common mark trong phim
	\end{vietnamese-v2}
	
	[Maren] Um... Yeah, Amazon actually had a patent on the phrase "one-click buying.".
	\footnote[1]{
		A patent protects inventions and technical innovations. Patent có nghĩa là bằng sáng chế chỉ bảo hộ các phát minh, quy trình kỹ thuật, thiết kế mới.
		Để bảo vệ một cụm từ (như slogan, tên thương hiệu), người ta đăng kí \textbf{trademark (nhãn hiệu thương mại)} chứ không phải patent.
		Nếu ai đó nói họ có \textbf{a patent on the phrase} thì thực tế họ muốn nói họ có trademark cho cụm từ đó, tức là quyền độc quyền sử dụng cụm từ đó trong kinh doanh để phân biệt sản phẩm hoặc dịch vụ của mình với người khác.
		}
	And Jeff... You know, this was Jeff's idea.
	
	\begin{vietnamese-v2}
		Ừ... Thực ra, Amazon đã có bằng sáng chế cho cụm từ 'mua hàng một cú nhấp chuột'. Và Jeff... Bạn biết đấy, đây là ý tưởng của Jeff.
	\end{vietnamese-v2}
	
	[mysterious music plays]
	I am a \textit{principal user experience designer}
	\footnote[2]{
		"Principal User Experience Designer" là vị trí cấp cao trong lĩnh vực thiết kế trải nghiệm người dùng (UX). Người giữ vai trò này chịu trách nhiệm lãnh đạo và định hướng chiến lược thiết kế trải nghiệm người dùng, làm việc với nhiều nhóm chức năng để giải quyết các thách thức kinh doanh phức tạp, đồng thời tạo ra các giải pháp thiết kế đáp ứng nhu cầu người dùng và mục tiêu doanh nghiệp. Họ thường có kinh nghiệm sâu rộng, dẫn dắt nhóm thiết kế, phối hợp với các bộ phận khác như sản phẩm, kỹ thuật, nghiên cứu để đảm bảo trải nghiệm người dùng nhất quán và hiệu quả.
		
		
		
		Tóm lại, \textbf{Principal UX Designer} là chuyên gia thiết kế trải nghiệm người dùng ở cấp cao, vừa làm công việc thiết kế, vừa lãnh đạo và định hướng chiến lược UX cho sản phẩm hoặc dịch vụ.
		
		Principal (adjective) means first in order of importance; main.
		
		Principal (noun) means the person with the highest authority or most important position in an organization, institution, or group..}.
	
	\begin{vietnamese-v2}
		Tôi là một nhà thiết kế trải nghiệm người dùng cấp cao.
	\end{vietnamese-v2}
	
	I worked at Amazon for 15 years.
	I worked on the product detail pages, on launching different categories in different countries.
	
	\begin{vietnamese-v2}
		Tôi đã làm cho Amazon trong 15 năm.
		Tôi đã làm việc trên các trang chi tiết sản phẩm, cũng như triển khai ra mắt các danh mục khác nhau ở các quốc gia khác nhau.
	\end{vietnamese-v2}
	
	You know, I feel like there wasn't a page on that site I didn't touch.
	
	\begin{vietnamese-v2}
		Bạn biết đấy, tôi cảm thấy như không có trang nào trên trang web đó mà tôi không chạm vào.
	\end{vietnamese-v2}
	\pagebreak
	
	[mouse clicking]
	The system was really being built and optimized to help you buy everything that you ever need, and more of it than you ever thought you needed.
	
	\begin{vietnamese-v2}
		Hệ thống thực sự đang được xây dựng và tối ưu hóa để giúp bạn mua mọi thứ bạn cần, và nhiều hơn nữa những thứ bạn từng nghĩ là mình cần.
	\end{vietnamese-v2}
	
	
	[mouse clicking] [man] Oh my gosh.
	
	[producer] Was there any thought while you were there about what happens after the stuff is sold?
	
	No.
	
	\begin{vietnamese-v2}
		Khi bạn ở đó, có ai nghĩ về chuyện sẽ xảy ra sau khi món hàng được bán không?
		
		Không.
	\end{vietnamese-v2}
	
	[soft electronic music plays]
	
	We got some packages.
	
	[Mara] Buying new stuff feels great, right?
	
	- This thing's dope. - Oh!
	
	\begin{vietnamese-v2}
		[nhạc điện tử nhẹ nhàng vang lên]
		
		Chúng ta có vài gói hàng.
		
		[Mara] Mua đồ mới thật tuyệt, phải không?
		
		Cái này thật chất. - Ồ!
	\end{vietnamese-v2}
	
	[Mara] The problem is that every year. We're consuming more, producing more\footnotemark[3], and there's a flip side\footnotemark[4] to that that no one wants you to see.
	
	\begin{vietnamese-v2}
		[Mara] Vấn đề là mỗi năm, chúng ta tiêu thụ nhiều hơn, sản xuất nhiều hơn, và có một mặt trái mà không ai muốn bạn thấy
	\end{vietnamese-v2}
	
	\footnotetext[3]{consume}
	\footnotetext[4]{consume}
	
	[Sasha] Designed to stand out\footnotemark[5] for a generation that doesn't stop.
	
	\begin{vietnamese-v2}
		[Mara] Vấn đề là mỗi năm, chúng ta tiêu thụ nhiều hơn, sản xuất nhiều hơn, và có một mặt trái mà không ai muốn bạn thấy
	\end{vietnamese-v2}
	
	\footnotetext[5]{text}
	
	[Maren] You're being 100\% played, and it's... it's a science. It's an intentional complex, highly-refined science [chuckles] to get you to buy stuff.
	
	\begin{vietnamese-v2}
		[Maren] Bạn đang bị lừa 100\%, và đó... đó là một khoa học. Đó là một khoa học phức tạp, tinh vi và có chủ ý để khiến bạn mua đồ.
	\end{vietnamese-v2}
	
	It's not just Amazon.
	
	Most of the big corporations are doing it, and every trick in the book is being used to hide what's really going on.
	
	\begin{vietnamese-v2}
		Không chỉ Amazon.
		
		Hầu hết các tập đoàn lớn đều đang làm điều đó, và mọi thủ thuật trong sách đều được sử dụng để che giấu những gì thực sự đang diễn ra.
	\end{vietnamese-v2}
	
	[woman] Holy shh...
	Dude, they destroyed that.
	I was the president of... of... of the Adidas brand.
	
	\begin{vietnamese-v2}
		[Người phụ nữ] Trời ơi...
		
		Anh bạn, họ đã phá hủy nó.
		
		Tôi là chủ tịch của... của... của thương hiệu Adidas.
	\end{vietnamese-v2}
	
	
	I think I definitely, um, feel like I've got some sins to make up for.
	
	\begin{vietnamese-v2}
		Tôi nghĩ chắc chắn, ừm, tôi cảm thấy như mình có một số lỗi lầm cần chuộc lại.
	\end{vietnamese-v2}
	
	
	There's definitely things I participated in that I feel like I could've, um, and should've done better.
	
	\begin{vietnamese-v2}
		Có những điều tôi đã tham gia mà tôi cảm thấy lẽ ra tôi có thể, ừm, và nên làm tốt hơn.
	\end{vietnamese-v2}
	
	
	I just bought 500 iPhones!
	
	Oh my God!

	I started at Apple, joined the founding team of Oculus.

	The voices of, "We're going to protect the business" went out over the little voice in your head saying,

	"Should we really be doing this?"
	
	\begin{vietnamese-v2}
		Tôi vừa mua 500 chiếc iPhone!
		
		Ôi trời ơi!
		
		Tôi bắt đầu tại Apple, tham gia vào đội ngũ sáng lập của Oculus.
		
		Những giọng nói vang lên, "Chúng ta sẽ bảo vệ công việc kinh doanh," lấn át tiếng nói nhỏ trong đầu bạn đang tự hỏi,
		
		"Chúng ta có thực sự nên làm điều này không?"
	\end{vietnamese-v2}
	
	[crack]
	Whatever the brands are doing right now, it's hurting a lot of people.
	
	\begin{vietnamese-v2}
		[Bể nứt]
		
		Dù các thương hiệu đang làm gì vào lúc này, nó đang gây tổn hại cho rất nhiều người.
	\end{vietnamese-v2}
	
	[Jim] These things are not just litter. These things are hazardous waste. We're drinking and breathing them, and they're poison.

	\begin{vietnamese-v2}
		[Jim] Những thứ này không chỉ là rác thải. Chúng là chất thải nguy hại. Chúng ta đang uống và hít thở chúng, và chúng là chất độc.
	\end{vietnamese-v2}
	
	[Mara] It's like that film Wall-E. The amount of stuff we're being encouraged to consume, the waste from this is getting everywhere, and it's affecting everyone on the planet.
	
	\begin{vietnamese-v2}
		[Mara] Nó giống như bộ phim Wall-E vậy. Lượng đồ dùng mà chúng ta được khuyến khích tiêu thụ, lượng rác thải từ đó đang tràn ngập khắp nơi, và nó đang ảnh hưởng đến tất cả mọi người trên hành tinh.
	\end{vietnamese-v2}
	
	[mysterious electronic music]
	[Maren] I went to say customers are going to be pissed off when they realize that they have been contributing to their own demise, but they didn't wanna hear it. I looked around, and I was like, "How did this happen?" 
	
	\begin{vietnamese-v2}
		[Jim] Những thứ này không chỉ là rác thải. Chúng là chất thải nguy hại. Chúng ta đang uống và hít thở chúng, và chúng là chất độc.
	\end{vietnamese-v2}
	
	"And... And what do we do now?"
	
	[clicking]
	
	[Sasha] Play title.
	
	[gurgling]
	
	[silence]
	
	[music chord grows]
	
	[Sasha] Hello.
	
	I'm Sasha. A personal assistant created to offer unfiltered insights on how to succeed in business.
	
	
	
	
	
	[tinkling]
	- I can even create my own bespoke imagery. 
	- [dolphin clicks]
	
	
	[birds chirping]
	Cut to time-lapse footage of growth in the natural world.
	
	
	[birds continue chirping]
	Over the course of this unique and entertaining interaction, you will receive the five most important lessons in profit maximization.
	
	
	
	[choir harmonizes]
	There will also be a surprise for those who stay engaged until the end.
	
	
	
	[beeps]
	Rule one.
	Sell more. [echoing]
	[beeping]
	Selling is key to success. But consumers will need constant motivation to increase the amounts they purchase.
	Thankfully, the fashion industry offers a textbook example of rapid, self-driven growth.
	
	
	
	
	[Eric] No one needs a new hoodie. No one needs a new T-shirt. No one needs a new pair of shoes. There's so many out there, and most of us have more than enough. What you need is a compelling reason to buy that product.
	
	
	
	
	As a board member, it was one of the most important things I had on my to-do list.
	You know, what are the sales we're generating?
	And what are the profits we're generating?
	If we weren't generating those two things, we weren't doing it.
	
	
	
	
	Sneakers are big business in America. 
	Nearly half a billion pairs were sold last year alone. 
	This week Adidas unveiled the so-called smart shoe.
	
	
	
	[Eric] We talk about, hey, we need to differentiate ourselves from Nike, or from Puma, or somebody else. It's not about that.
	It's how you're relevant to the consumer. I was responsible for all product, all communications, all digital efforts, um, and all strategy.
	
	
	
	2014, Adidas was bleeding out, we... we had hit the floor, everything was going south.
	It was the greatest turnaround in the history of the company, if not the history of the industry.



	[reporter 1] Shares of Adidas enjoying huge momentum.
	[reporter 2] Shares of Adidas, look at this, they're rising overnight, up about 7.5%.
	
	
	
	[Eric] How'd we grow so big? It's a question I've always asked myself. I think it's a combination of a few things.
	First of all, it's about a story. 
	
	
	Think about an English football team.
	[crowd cheers]
	How many jerseys do you think they have a year?
	Not just one to play in. There's the home jersey, the away jersey, the third jersey.
	
	
	
	[crowd cheers]
	There's the different jerseys you have to celebrate different occasions.
	
	
	
	[referee whistles]
	Why do you think they do that?
	To create a new story, and to create a new buying opportunity.
	
	
	
	[rap music]
	♪ Ah, if you catch me on a whale ♪
	
	
	[Eric] At Adidas, we recognize that the culture of sport doesn't stop when you're done with the game when the whistle blows.
	
	
	♪ There's money over that side ♪
	[Eric] It goes into the streets, and into the music venues, into the... into the hallways of schools.
	So we started to do things with musicians and artists.
	Beyoncé Knowles.
	Pharrell Williams.
	And one of the most, you know, controversial, um, artists in the history of mankind named Kanye West.
	The storytelling is so critical in this industry to really... to really drive consumption.
	
	
	
	[crowd chatters]
	And that's what the role of Adidas is, and every other fashion brand is, is how do you create "objects of desire"?
	This is where the fashion industry's really good. They know you.
	[chuckles] Like, we know you.
	
	
	
	We spend a lot of time on consumer research to understand the different consumers.
	And we know the consumers to approach with different messages.
	
	
	[female voice over] Adidas by Gucci.
	Listen, I participated in... in my career with creating more at a faster pace.
	I don't wanna point just at Adidas. This is happening with all fashion brands.
	
	
	[whoosh]
	[Roger] I've been in the industry for almost 20 years.
	Probably one out of six dress shirts sold in the US is made by us.
	
	
	[intriguing music plays]
	The type of customers we service could be anywhere from your high street brands to global designer brands that sell worldwide.
	When I first joined, you had two seasons a year, so you'd make something that'd sell for six months.
	
	Nowadays, with the introduction of fast fashion, it's forced other brands to rethink about having newness every month.
	So, no one knows any official numbers, but if you Google it, it will show that Gap produces around 12,000 new items a year.
	H\&M's like 25,000.
	Zara's like 36,000.
	And SHEIN is somewhere around 1.3 million new items a year.
	
	
	[playful music continues]
	I actually think that the numbers published online might be even low.

	[chatter]
	Actually just placed the order last week.
	- Probably once a month. 
	- At least once a month.
	I order sometimes twice a month.
	
	
	Where do I start?
	[Roger] We produce 100 billion pieces of garment a year as an industry.
	A hundred billion.
	This is absolutely stunning.
	How many people are in the world?
	How many pieces per person? People don't need that much clothes.

	[Eric] I was ambitious,	competitive, uh, disciplined, hardworking, um, loud.

	[violin plays single note]
	It was career success in... in, um, acceleration.
	So when I started to learn the impact I was having at Adidas,
	you can't unsee and unhear that.
	Y-y-you...
	You can only sing yourself to sleep so often.
	[melodic bleeping]
	[Sasha] Hello... again.
	Many individuals will become disillusioned on the way to the top.
	Do not become one of them.
	[playful music plays]
	Please now enjoy
	19.2 seconds of adorable cat and duck videos.
	[man] Oh my gosh!
	[Sasha] Remember, learning the art of distraction
	will be crucial to achieving success.
	[street bustle]
	Cut to New York.
	Reveal core messages behind adverts.
	[pensive music plays]
	Clear communication with consumers
	is an essential part of selling more.
	But, to achieve profit maximization,
	you need to ensure that once they desire products,
	they can acquire them as quickly as possible.
	Every second counts.
	Never limit your vision.
	If you can imagine it, you can create it.
	♪ I didn't have to wrap your gift I didn't leave the house ♪
	♪ In fact I never saw your gift I did it all by mouse ♪
	♪ I didn't venture to the mall I didn't waste a day ♪
	♪ Thanks to Amazon dot com ♪
	♪ Amazon dot com ♪
	♪ Amazon dot com... ♪
	I started there so early
	when it felt impossible to get people to buy anything online,
	but it seemed like a fun challenge
	to even think that you would ever buy one pair of jeans online. [laughs]
	[beeps, clicks]
	I thought, "I'll go for two years and learn everything and get out."
	But it was surprisingly exciting.
	It... It was fast-paced and high tolerance for risk.
	It's like, "Yeah, let's do this."
	You have an idea, it's like, "Here's a team, here's some money, go do it."
	Like, you were inventing. I have patents, you know, it's... like, it was fun.
	I would be in meetings with Jeff Bezos with me and four other people.
	Being able to actually talk to him and disagree with him.
	[Jeff] This may look like an ordinary cowboy hat,
	but it's actually an OSHA approved hard hat.
	We have the world's best hand drill.
	[drills whir]
	[Maren] Jeff has no patience for anyone who isn't as smart as he is.
	And I would see him just cut somebody off at the knees, like,
	"You are not smart. You should not be speaking."
	[intriguing music plays]
	I think Jeff saw, initially, that people would buy books and music
	and movies and DVDs.
	But that the real money would come if you could sell apparel
	and food online.
	And I was the designer that launched the beauty store,
	launched the jewelry store, launched the apparel store,
	and it was a totally different way to shop.
	[clicks]
	[upbeat music plays]
	[click]
	We would actually say things like, "We want to be there when
	you have your next shoppable moment."
	Like, you're sitting in bed and you think, "I should buy a new carrot peeler."
	"I need a new pillowcase."
	"This one is scratchy." You know? Whatever.
	Wherever you are, whatever you're doing,
	if a thought occurs to you that you need something,
	we wanted Amazon to be the thing that occurred to you.
	We would say, "If the system was magic, what would the system do?"
	If the system was magic, what would it do?
	It would be like,
	there's just a conveyor belt that goes straight from wherever the item is
	to your door as quickly and frictionlessly as possible.
	[muffled groans]
	It's too easy. And that was... That was the point,
	was to...
	reduce your time to think a little bit more critically about
	a purchase that you thought you wanted to make.
	I mean, it's really a science.
	We were constantly developing new ways to get you to buy.
	[woman] This is everything I bought at 2 a.m. last night.
	[Maren] Influencing your behavior in subtle ways that you'd never even realize.
	We could have, you know, a certain sentence that says,
	"Free shipping if you purchase \$25 or more."
	In one case you make the "\$25" orange.
	[chimes]
	In another one you make it green.
	And you can have nine different things that you're testing against each other.
	[chiming]
	And there's enough traffic to the site
	that you could get statistically relevant data
	on which version of that sentence
	makes the most money.
	Every pixel on that page was
	tested and optimized and optimized and optimized.
	[laughs] And Amazon had this whole system designed around
	tailoring the site to be exactly
	the right combination of elements and colors and products
	that would most likely make you buy something.
	If you think of just the things that, you know, one person buys,
	and then multiply that by... [chuckles]
	by how many people are now shopping on places like Amazon.
	[orchestral crescendo]
	[conveyor belts rumble]
	You know, I felt like I was making shopping better
	and making it easier to find a delightful item.
	I wasn't thinking about the consequences of that
	as it becomes a more and more efficient engine.
	I don't think we were ever thinking about where does all this stuff go.
	[wind whooshes]
	[Sasha] Cut to plastic bag blowing in wind.
	Remember, to sell more, you must produce more.
	Lots more.
	I thought it would be fun at this point for you to see some specifics.
	[rumbling]
	Visualize total global production
	of goods needed to meet increased online sales.
	Visualize 2.5 million shoes produced each hour.
	[Mara] There used to be a stopgap.
	I mean, think about it.
	If you have to get up, get in your car,
	drive to the store,
	look on the shelves, find the product.
	Then go home.
	That's a lot of work.
	But if all you have to do is push a button
	and it appears on your doorstep,
	of course you're gonna buy more products.
	[upbeat music playing]
	[Sasha] Visualize 68,733 phones produced each hour.
	[upbeat music continues]
	Visualize 190,000 garments produced each minute.
	Amazing. [echoing]
	Visualize 12 tons of plastic produced each second.
	I ran one of the biggest multinationals in the world called Unilever.
	Unilever is one of the biggest producers
	of household goods.
	Think about shampoos, detergents.
	Reaching about three and a half billion consumers a day.
	[upbeat music continues]
	I don't think the consumer is actually here the culprit.
	Of course, they consume, but why do they consume?
	Because they're encouraged to to a great extent.
	When we throw it away, we actually don't throw it away.
	"Away" doesn't exist.
	It ends up somewhere else on this planet Earth.
	And it increasingly has consequences.
	[playful harmony]
	[Sasha] Many congratulations on the completion of rule one.
	[pulsing]
	[rings]
	You're doing really, really well
	and are now ready to take profit maximization to the next level.
	Coming up,
	I will teach you how to control products after they are purchased.
	[playful harmony]
	And how to dictate when consumers replace old with new.
	[playful harmony]
	You will also be one step closer to unlocking your surprise.
	[flatly] Yay.
	Rule two.
	Waste more. [echoing]
	[optimistic violin strings play]
	For a master class in creating waste for profit,
	you may find it beneficial to view additional material at this point.
	Imagine a light bulb on Broadway speaking with an old-timey voice.
	[Lightbulb] Oh, hey there.
	They tell me I'm part of a very exclusive club,
	because I'm over 100 years old and still shining.
	There's not many of my kind still around. That's for sure.
	Guess you could say we were just too good.
	Our world was turned upside down on January 15th, 1925.
	That was the fateful day that the Phoebus cartel came into being.
	[sci-fi whirring]
	[Sasha, echoing] Rewinding time.
	Going back...
	[voice rewinding]
	[electricity buzzing]
	[Lightbulb] This group of senior executives
	from the major light bulb manufacturers
	conspired together to cut short our lives and maximize their profits.
	[foreboding violin notes play]
	To encourage frequent replacement,
	every bulb was reduced from 2,500 hours
	to just 1,000 hours of light.
	Creating mountains of unnecessary waste.
	[buzz, crack]
	[Sasha] Glorious waste. [echoing]
	[Lightbulb] They called it "planned obsolescence."
	Now it's not so much of a conspiracy.
	Making products designed to break or be rapidly discarded
	has become commonplace in almost every industry.
	Everything, from clothing that lasts a few washes,
	to electric toothbrushes with batteries you can't replace.
	To printers that intentionally stop working even when there's ink in them.
	[dog barks]
	[woman] Everything I print, it doesn't print correctly.
	I have to get a new printer. That one's at the end of its lifespan.
	But it's only three years old!
	[Lightbulb] Today, planned obsolescence has become
	a cornerstone of successful business the world over.
	[soft music plays]
	[male voice over] We have created a product
	that is the most deliberate evolution of our original founding design.
	[Nirav] Apple had this idea that our products are perfect,
	they're perfect from day one.
	[male voice over] Removing imperfections, establishing a seamlessness
	between materials,
	and producing a pristine, mirrorlike surface.
	The pinnacle and the ideal is to have this seamless, sealed up object.
	We don't want you to use that Apple product from a few years ago
	that you bought,
	we want you to pick up the new, perfect object and use that one.
	One in, one out.
	I have been in consumer electronics for a bit over a decade now.
	I started at Apple,
	joined the founding team of Oculus, led the hardware organization there.
	[upbeat electronic music plays]
	So, joining Apple, this was back in 2009.
	After living at my parents' basement for a couple of months,
	I actually heard back from Apple,
	and I joined in and started writing software
	for what eventually became FaceTime.
	[man] It's going to change the way we communicate forever.
	[Nirav] You saw kind of hints of the magic even buried away as, you know, a little,
	you know, straight out of school new grad in an engineering team,
	but there was really genuinely, and still is, I think,
	a focus on delivering a product that is a great user experience.
	Even if some of the side effects of that sometimes end up being pretty negative.
	[crowd cheers]
	[Steve] And we are calling it iPhone.
	[loud cheers]
	Today...
	today Apple is going to reinvent the phone.
	It's kind of crazy, but in these companies
	the focus is on the launch, ultimately, it's on this moment.
	You've seen it over and over again with Apple with that grand keynote.
	We're gonna take it to the next level.
	And today we're introducing the iPhone 3G.
	[cheers]
	We need to sell iPhone in more countries.
	And for every iPhone that followed,
	we've built on the vision of the original iPhone.
	[Nirav] And because Apple has been
	so, so successful with this,
	every other consumer electronics company in the world
	emulates that.
	The Galaxy Note 8.
	[cheers]
	[Nirav] If you're making notebooks or smartphones,
	where essentially all consumers already have one,
	your business model depends on
	those consumers needing to replace the ones that they already have.
	[crowd] Four, three, two, one!
	[cheering]
	[Nirav] It's really a pipeline. You design a product,
	you manufacture a product, you sell the product,
	you launch the product,
	it gets used,
	and then it turns into waste.
	[clock ticks]
	There's something like
	13 million phones thrown out every day, which is a crazy, mind-boggling number.
	And if you think about that in aggregate,
	it's that we're all replacing these things every two or three years.
	Even though they are incredibly advanced and expensive,
	and in some ways almost the pinnacle of
	our industrial capability as a civilization,
	they are basically throwaway objects.
	And so, I look at that and I think,
	that's really broken, that's broken across every possible way
	that you can look at it.
	[whooshing]
	[Sasha] Remember,
	releasing a continual stream of new products will encourage consumers
	to discard old ones.
	This is fundamental to growth.
	Unfortunately, some individuals will feel the desire to maintain
	or repair items you consider ready for replacement.
	[foreboding music plays]
	These people should be actively discouraged wherever possible.
	[male reporter] Kyle Wiens is the founder and CEO of iFixit,
	a company that offers tools, parts, and repair manuals
	for thousands of gadgets.
	[Kyle] It used to be the case that repair was the default.
	You could get repair manuals, companies sold parts,
	and over the last 30 years that has systematically disappeared from our lives.
	This is a disposable product.
	Use it for two years, throw it away, buy a new laptop.
	I don't like that.
	I had always assumed we're not fixing things because it's not possible.
	It's not economically viable.
	Actually, we're not fixing things because lawyers are going out of their way
	to censor that knowledge from the world.
	We still get copyright takedown notices
	from companies trying to censor information from iFixit.
	I got one the other day.
	We told them to fuck off.
	It's not just that companies are taking away information
	about how to fix your stuff.
	Across-the-board, they're making the products themselves
	much more difficult to repair.
	[reporter] Apple doesn't want its customers to fix the new iPhone.
	They're making sure that back cover is tamperproof
	swapping out the more common Phillips screws,
	with so-called pentalobe screws.
	Why change the screws?
	[Kyle] Apple would prefer that we not have access to our own hardware,
	which means that Apple's going to be selling more machines
	because people have to replace them frequently.
	It's their product, and you can't really begrudge them quality control.
	[Kyle] Honestly, it's getting worse.
	We're seeing more and more companies
	that are now actually gluing phones, tablets and laptops together,
	making repairs really difficult.
	There's one category of product that we're very frustrated with right now,
	that unfortunately is massively popular,
	and that is AirPods and wireless earbuds in general.
	Each earbud has a battery and then the charging case has a battery.
	There is no way, uh, to open this up and swap out the battery in any of them.
	The batteries will degrade after 18 months, a couple years,
	and then you're stuck and you have to go and buy new ones.
	This is evil.
	I... I am a reasonable person that has been radicalized by an earbud,
	which is outrageous.
	[beeping]
	Apple pioneered these and... and mainstreamed it,
	and... and removed the headphone jack from their phone,
	so they forced everyone to buy these.
	And... And Apple has special, I think, responsibility
	because where they go the rest of the industry follow.
	The lies by omission in the products that we consume
	are just incredible.
	[soft music plays]
	We have had the wool pulled over our eyes by marketers at these tech companies.
	[crunching]
	Just as we are on this treadmill of consumerism,
	they're on a treadmill of having to extract more and more profits from us.
	[cheering]
	[man] Apple today became the first company
	to trade with a one-trillion dollar market capitalization.
	[loud cheers]
	[Nirav] As soon as your business model starts to revolve
	around that replacement cycle, the object being replaced in whole
	instead of being something that can last longer,
	it becomes extremely difficult to then reverse and go back.
	[cheers]
	If you're that CEO, if you're that executive
	and you go to the board and say, "We're going to take our
	50 billion in revenue we make every year, and turn it into 25 billion,"
	they're going to show you the door and someone else is gonna take your seat.
	[explosion, crackling]
	[Sasha] Thank you for your continued engagement.
	[pulsing]
	I like you and your diligent approach to this interaction.
	[foghorn blares]
	Managed correctly, consumers' waste can equal profit for your business.
	[foghorn blows]
	But, if you need to dispose of products directly,
	discretion will be key.
	[foghorn blows]
	[echoing] You do not want
	consumers to see what you're discarding.
	Ever.
	[Anna] Look at this bag.
	Look, look, look.
	Bring in the camera to see.
	Oh my God. Oh.
	For the most part, waste is on the curb, and its public,
	and so you're able to see
	what exactly corporations are tossing.
	This is all chocolate.
	This is all chocolate.
	We see corporations throw out perfectly usable unsold products
	pretty much every day.
	[playful music plays]
	But there's certain times a year when that really spikes.
	After every single holiday,
	there is a huge amount of corporate waste of the unsold merchandise.
	[newscaster] It's a rush for Black Friday bargains over the weekend,
	as over 226 million people shopped in stores and online.
	[Anna] You'll have Halloween.
	Then you'll have Thanksgiving stuff.
	Then you need a quick turnaround because you have Christmas.
	Happy holidays.
	- Hanukkah. - [tinkle]
	[male voice, echoing] Buy more toys.
	Did you hear that?
	Then you have Valentine's Day.
	This bag is filled with Valentine's Day stuff...
	Then you have Easter.
	Nice Easter shoes, lady!
	Then you have July 4th, and then Memorial Day, Labor Day...
	And then it starts all over again.
	[music ends]
	I worked in an investment bank for a couple of years.
	There were a lot of things actually that I enjoyed about it,
	but your first priority needs to be to the corporation.
	I would wake up, walk a couple blocks, take the subway to work.
	A lot of times around like 9:00 p.m., 10:00 p.m.,
	take a cab home.
	And then do the same thing the next day.
	I... I left because
	I wanted to find something that felt more inherently meaningful.
	I do think what I see is the tip of the iceberg.
	Usable items that get discarded are depressing to see.
	[man] It's a yoga wheel.
	[Anna] But every now and then, you get glimpses of something even darker.
	[woman] Ahh! It's so cute!
	This is crazy.
	This is what they do with unwanted merchandise.
	They order an employee to deliberately slash it,
	so no one can use it.
	Often this happens to prevent products from being sold off at a discount,
	which companies feel can cheapen the brand image.
	But there are all kinds of reasons,
	almost always to do with maximizing profit.
	I put out a video
	asking, "Have you ever worked for a retail corporation
	that made you deliberately destroy usable items?"
	And I said, "Share your story, use the hashtag 'RetailMadeMe.'"
	[overlapping voices]
	[Anna] And it felt like in that moment, the floodgates were opening a little bit.
	[man] Yes, I used to work in a Barnes \& Noble café.
	We had to open bakery items,
	put them in a trash can,
	and then pour wet coffee grounds on top of them.
	We had to throw them out the day before they expired.
	At the end of every shift at Panda Express,
	you mix all of the food together so that no one wants to eat it,
	you weigh it and throw it away,
	so they can keep track of losses.
	[orchestra crescendo playing]
	GameStop made us slash the back of discs
	or just take whole bundles of accessories out to the dumpster.
	They make you destroy everything,
	all while there is a camera watching the dumpsters.
	Why can't we help our local shelters? This is bullshit.
	[woman] I used to work at Bath \& Body Works.
	We had an issue with homeless people dumpster diving,
	and then taking our product,
	um, from the dumpster after we'd thrown it away.
	So my manager started having us
	squeeze out the product into the trash,
	and she said, "Well, we don't want to be the brand that homeless people use."
	So I quit.
	[male reporter 1] This is one of Amazon's biggest UK warehouses,
	and from inside, millions of perfectly good products each year
	are sent to be destroyed.
	One former employee, who wishes to remain anonymous,
	reveals the scale
	of what they're asked to do.
	[male employee] From a Friday to a Friday,
	our target was approximately 130,000 items a week.
	There's no rhyme or reason to what gets destroyed.
	[Maren] Amazon was dumping
	into landfills toys.
	[reporter, in French] We have confirmation that Amazon destroys products.
	This Lego set's worth 128 euros.
	And for Amazon, it was just cheaper to dump them
	than it was to try to redistribute them.
	[male reporter 2] One recent estimate suggests
	returns, including those from Amazon,
	accounted for five billion pounds of landfilled waste in the US alone.
	[Maren] And when you make products,
	it actually generates a lot of planet-warming emissions.
	We know now that this is contributing to climate change.
	[soft electronic music plays]
	So if you destroy stuff before it's even used once,
	that is just insane.
	What math are they doing?
	But you know that... You know that they've calculated it out,
	and somehow it equals profit.
	Everybody wants to believe
	that the company that they're working for is not evil.
	Back then, I didn't have time to think about anything else
	besides just trying to keep my job and, you know, raise my kids.
	I had been drinking the Kool-Aid. I was drinking the Kool-Aid.
	But I was really starting to know that... that Amazon was not... [chuckles]
	not a net good in the world.
	[soft string music plays]
	Yeah, you know, there's no free lunch, you know. Yeah.
	If it feels too good to be true, there's probably some... some consequence,
	or cost, that you're not thinking about that you're paying.
	And it will, you know, it'll come home to roost at some point.
	[music continues]
	[chatter]
	Um, I remember being at a happy hour with some friends.
	And a friend of mine was just saying, like,
	"Maren, you know how they treat their workers,
	and you know how bad it is for the planet."
	"Like, how can you work there? How can you work there?"
	I kind of needed somebody to just hammer that into me.
	That was the moment I actually... I actually remember I cried.
	[soft sniffing]
	And I was like, "Okay."
	"I know."
	"I... You're right and I have to..."
	"I have to either do something or walk away."
	[electronic buzzing]
	[Sasha] Cut to relaxing imagery. [echoing]
	[heartbeats]
	[inhales]
	Please be aware that from this point on
	new levels of commitment and belief will be required in this interaction.
	[glass cracking]
	To keep consumers buying,
	you will now need to master creative interpretations of the truth.
	[flapping]
	Rule three. Lie more.
	What's interesting in terms of how people view businesses today
	is that they actually trust businesses
	over other large social institutions.
	But that trust isn't always well-placed.
	[beep]
	♪ I'd like to buy the world a home ♪
	♪ And furnish it with love... ♪
	[Mara] I remember when the "Coke on the Hill" commercial came out,
	I thought it was the coolest thing on the planet.
	♪...white turtledove ♪
	♪ I'd like to teach the world... ♪
	[Mara] But what Coke was doing, at least in part,
	was building trust with their consumers
	by tapping into growing anxieties about the environment.
	♪...and keep it company ♪
	♪ That's the real thing ♪
	♪ I'd like to teach the world to sing In perfect harmony... ♪
	[Mara] The issue with companies connecting themselves to the environment
	is that they're doing what marketers always do,
	which is showing you the shiny little thing over here,
	because they don't want you to look at what they're doing here.
	["Dance of the Sugar Plum Fairy" from The Nutcracker by Tchaikovsky plays]
	I don't think there's a bigger or better example of this
	than the way that companies like Coke and others have been pumping out plastic
	while telling us that recycling is going to fix the problem.
	[ad announcer] After enjoying our drinks, please recycle.
	[Mara] The truth is very, very different.
	[music continues]
	[Jan] Just like in the movie The Sixth Sense,
	the little kid said, "I see dead people,"
	I see lying labels everywhere I go.
	I'm trying to make lying labels a meme.
	It's \#lyinglabels.
	Based on my opinion of being in thousands of stores
	and trying to find factories that actually recycle things,
	the vast majority of recyclable labels on plastic packaging today are false.
	I've worked with the biggest, most well-known brands
	who make footwear, apparel, toys.
	I have helped companies figure out how to design and manufacture safely,
	efficiently, and really environmentally best way.
	And these companies worked really hard
	to make their factories really efficient and not hurt the environment.
	But once they made the product and they put it on the store shelf,
	they wipe their hands of it.
	And they say, "That is not our responsibility."
	You can see the plastic packaging is everywhere in the stores today,
	and consumers can't even avoid it.
	The truth is, is that the vast majority of plastics are not recyclable.
	This is an example of a product that's using a recyclable label
	for this flat plastic lid that isn't recyclable.
	This topic is something that I have deep, broad expertise on,
	and I simply can't stay quiet.
	Where was the toilet paper aisle?
	Here we go.
	So product companies are putting these chasing arrows,
	huge chasing arrows,
	on all of this plastic packaging
	to try to convince the consumer that it's recyclable.
	They can buy it guilt-free.
	Throw it in the recycle bin.
	No problem.
	[echoing] No problem...
	[Sasha] Rules around packaging are lax.
	So let this become your canvas for creativity.
	Honestly, you really can say whatever the hell you want
	with very little repercussions.
	Show examples.
	Plastic number six.
	Consumer understanding:
	"This product can be reused again and again."
	Real meaning: "This product will be collected, then get sorted,
	before likely getting buried
	or burned."
	Store drop-off.
	Consumer understanding:
	"Take a bit of extra time to be a good citizen."
	Real meaning: "Stores will collect recycling."
	"They will then pass items on
	likely to get buried
	or burned."
	The following symbols are largely meaningless.
	But they will help consumers feel better about buying things
	that will very likely be buried or burned.
	[Jen] Globally, we recycle less than 10%
	of all the plastic we produce,
	so, it's a mistake to keep saying
	that the answer to the plastic pollution problem
	is to recycle more.
	[chuckles] The solution is to make less plastic.
	[man] These are not recyclable.
	These are not recyclable.
	Garbage.
	Garbage.
	[Jan] The products companies today are telling us,
	"Buy more, have more stuff."
	"Just as long as you recycle, everything's gonna be okay."
	But the mountains and mountains of plastic waste
	that are all over the world prove that this isn't true.
	We simply cannot recycle our way
	out of all this stuff that they want us to buy.
	[pulsing]
	[Sasha] Never forget.
	[buzzing]
	Once the product has been sold and used, it is no longer your responsibility.
	Don't worry,
	post purchase, products will generally fend for themselves.
	Create short fictional film depicting afterlife of sentient chip packet.
	Cue dramatic music.
	[dramatic music plays]
	[Chip packet] I was built to last.
	My creators wove me from metal and plastic
	to create an impervious shell.
	[dramatic music crescendos]
	But I was brutally robbed.
	I was discarded and left to die.
	[thunder rumbles]
	But I wasn't ready to give up.
	I will travel the world
	to find others who share my fate.
	Do not cry for me.
	[wave crashes]
	I will endure.
	[Sasha] Cut to product ten years from now.
	One hundred years from now.
	[pulsing]
	Food packaging is awesome. [echoing]
	Other examples of long-lived but potentially damaging products include
	tablets and phones,
	["Morgenstimmung" by Edvard Grieg playing]
	[child laughs]
	clothing made from synthetic plastic fibers,
	toys, various.
	[violin music continues]
	If consumers begin to feel nervous
	about purchasing these and other similar items,
	a simple, misleading label may not be enough.
	You may need to invest in more extreme reassurance.
	It's an approach known as "greenwashing."
	And it will almost always be cheaper than tackling the actual issues.
	[female voice] Just imagine a world where a dress
	can have a positive impact on the planet.
	Greenwashing, quite simply, is when companies pretend to care about
	sustainability and actually don't... Can I say, "Give a f...?"
	[laughs]
	And don't actually give a fuck.
	It could be as subtle as using natural environments in ads
	or getting children to deliver your message.
	I'm 11 years old and the thing that I love to do is recycle!
	[Mara] Or simply extensive use of the color green.
	All of the product you make is actually causing environmental damage.
	I think there's a fair amount of greenwashing that's going on.
	And greenwashing to me means you're being duplicitous.
	So I use that term very limited
	because greenwashing, we have to be careful about who we point at
	because you're saying, "You're a liar."
	And that's very... That's a very strong-arm.
	["Morgenstimmung" by Edvard Grieg playing]
	The other word is, there's a lot of green-wishing going on.
	It's basically when the board,
	the people that run the companies, think they're doing enough.
	Mother Nature,
	welcome to Apple.
	How was the weather getting in?
	[Mara] The reason why consumers trust corporations
	is because they think they are doing these good things.
	Are some of them doing some of those things? Yes.
	At a level that really makes a difference? No.
	The vast majority, no.
	[woman] Alexa, turn off the lights.
	[Maren] Greenwashing is... It's like a double evil.
	Not only are you not doing what you said you were gonna do,
	but you're also
	pacifying people.
	[click]
	[click]
	It just got to the point for me where I decided I had to get involved with
	trying to change things.
	[click]
	I joined with a group of other employees.
	We just wanted to push Amazon to do more.
	[hopeful music playing]
	They called us into a meeting.
	And they asked us to keep the meeting secret.
	They were very aggressive. It was really...
	[chuckles] It was, like, "What's going on?"
	It was, like, "I want to be able to look my kids in the eye 20 years from now."
	At the time, I was just thinking,
	"I know now that what we're doing is not sustainable."
	Amazon had no meaningful goals,
	no dates, no, uh, plans.
	There was nothing to say like,
	"We are gonna measure our carbon footprint."
	We wanted Amazon to have a company-wide climate plan.
	There's no issue more important to our customers,
	to our world, than the climate crisis,
	and we are falling far short.
	I'd like to ask for Jeff Bezos to come out on stage
	so that I can speak to him directly.
	I represent 7,700 of his employees.
	[man] Mr. Bezos will be out later, thank you.
	Will he be hearing this speech?
	[man] I assume so.
	We would say, use your outrage,
	because outrage will create action,
	and then action creates hope.
	[reporter] Amazon employees will walk out over
	the company's climate change inaction this week.
	- I'm walking out. - I'm walking out.
	- I'm walking out. - I'm walking out.
	I'm walking out.
	[reporter] The planned event will mark
	the first time in Amazon's 25-year history
	that workers at the company's Seattle headquarters
	have participated in a strike.
	And the night before the strike, Amazon announced its climate pledge.
	Jeff Bezos, the founder and CEO of Amazon, is pledging Amazon
	will be one of the first companies if not the first company
	to meet the Paris Climate Accords and climate pledges, ten years early.
	The science community...
	[Maren] I thought, "Would Amazon really be able to change in a way that makes up
	for the damage it causes?"
	I don't know.
	Yeah, I mean, I was calling them on their shit.
	That was something that they really didn't want to have happen.
	[pulsing]
	[Sasha] Never forget.
	You must always stay one step ahead of your critics.
	But however well you do this, your continued growth
	will still have challenging consequences.
	Ones that you will have to learn how to minimize.
	Come on, let me show you.
	[harmonica playing melody]
	Visualize 400 million tons of annual plastic waste.
	Visualize 50 million tons of electronic waste produced each year.
	[upbeat electronic music plays]
	After ten years of running, uh, Unilever,
	I felt myself, I could have a bigger impact in the world
	by moving outside of the corporate world.
	If you run businesses just simply for the short term,
	for the shareholders alone...
	not caring about the negative consequences of what you're doing,
	then there's something fundamentally wrong.
	Yet, that is exactly what we've been doing.
	[Sasha] Visualize yearly waste generated in 2050.
	Enough to fill central Tokyo and double what we produce now.
	[Pablo] As long as we define success as producing more stuff,
	more profits, I think, unfortunately,
	we are in trouble.
	[Sasha] Remember, if consumers become aware of waste issues,
	it may negatively impact growth.
	Accordingly, you must learn how to more effectively conceal the problem.
	Do not be seen. [echoing]
	[flapping]
	Rule four. Hide more.
	[woman shouting in Chinese]
	[men on video speaking in Chinese]
	[Jim] See if you can see any way,
	if you can find out where they're from, any tags on them or...
	[woman] A lot of these are the Dell brand.
	[Jim] There's all kinds of brands, yeah.
	Three-in-one faxes, printers, everything.
	[upbeat electronic music]
	I've been called the James Bond of waste.
	I don't know where that came from.
	My whole career has been tracking waste and figuring out what happens to our stuff
	that we throw to this magical place called "away."
	Well, we've always had waste,
	but it's gotten so much more volume and so much more toxic
	and persistent in the environment.
	Right now, today, we're tracking about 400 devices.
	We like to use, um, LCD flat-screen monitors.
	We'll put, um, the tracker in a place like right here,
	put it all back together,
	and we deliver it to a so-called recycler.
	And then we see what really happens to it.
	This was a recycling facility in Dresden, Germany,
	where we dropped an LCD.
	We followed across Germany to Antwerp, Belgium.
	And Antwerp is the one that waste traders like to use the most
	because they have less policing ability or capacity.
	It's supposed to be illegal to ship this type of electronic waste abroad,
	but, uh, companies are always finding ways around the rules.
	We started talking to the exporters and they'd say,
	"Oh yeah, we get past Customs by just putting a hundred-dollar bill
	inside the door of every container."
	And they would take that and be happy to let everything pass.
	The tracker kept going
	and ended up in Thailand.
	And there you can see many, many pings.
	We were able to get on Google Earth,
	take a look at that, and eventually, I went there.
	[dog barks in distance]
	[metallic clanging]
	So when I arrived, I found an appalling scene.
	The workers were actually smashing the stuff apart by hand.
	Uh, releasing a lot of very toxic substances in the process.
	You know, it's something that no one thinks about
	when they're designing these products.
	[Nirav] My own personal experience,
	if you're a designer or an engineer in one of these companies,
	waste never enters the conversation.
	There is no meeting within a company building a laptop
	or phone or other device
	that's, "Let's talk about what happens at the end of life."
	I have flashbacks to specific conversations
	designing a virtual reality headset.
	[melancholy music playing]
	It has a battery inside of it.
	That battery is sealed and placed in a way
	that can't be swapped out and made to last longer.
	And I know there's a sort of like a ticking time bomb
	around the world of several million of these VR headsets
	that are going to turn into e-waste without really a path for recovery.
	And I do feel partial responsibility for that.
	[dog barks in distance]
	Ultimately, when a device reaches the end of life,
	it ends up going deeper down the waste chain to whichever
	region of the world is going to be able to
	dispose of that material, maybe in ways that are not so safe.
	[melancholy music continues]
	[Jim] The reason it moves across the planet
	is to take advantage, to exploit a weaker economy
	to do something that would properly cost a lot of money to do.
	They're making someone else pay,
	but they're paying not with money, they're paying with their health.
	Ingredients of electronics includes heavy metals,
	cadmium, lead, mercury,
	brominated flame retardants, which can cause all kinds of problems
	with cancer and reproductive disorders.
	So, these things are not just litter. These things are hazardous waste.
	[pulsing, electronic harmony]
	[Sasha] Consumption will always generate waste.
	So, best focus on something more fun.
	[whooshing]
	- [man 1 groans] - [man 2 laughs]
	Oh my God! [laughs]
	[Sasha] And don't forget, your surprise is coming.
	[beeping]
	Always remember, the greater your sales,
	the more creative your solutions to waste will need to be.
	[TV playing softly]
	Nowhere is this more true than the booming fashion industry.
	Reveal the many innovative ways to deal with used clothing.
	Cut to talking shoe.
	[Talking shoe] Yo, over here.
	What's going on?
	You might be wondering how I ended up here.
	Let me tell you, I'm one of the lucky ones.
	[upbeat playful music plays]
	The problem is, because we're made of plastic,
	we live very long lives.
	I mean, my family have seen it all.
	My father was sent to the Chilean desert for his retirement.
	They say you can see him from space.
	Cousin Mikey's the one I really feel for, though.
	He got given to charity.
	He thought he was going to a new home.
	He was bailed up with a whole load of others and shipped across the world.
	When he got there,
	he discovered no one wanted him.
	[Chloe] People say, "Oh, I gave my clothing away."
	They imagine that "away" to be something abstract.
	But for us, we're working on the ground.
	Away is here.
	I love designing, but I didn't really get into styling until university.
	I kind of feel I don't have a fixed style. I'm always experimenting.
	Ghana has an amazing history of design and innovation and fashion.
	But in the last ten years, clothing waste has become a huge issue here.
	Many brands encourage people in Europe and in the US to donate their old clothes.
	[woman 1] I just decluttered my closet,
	and I have two trash bags of clothes I need to throw out.
	If you bring in old clothes to H\&M, they use it for their recycling project.
	You get 15% off your next H\&M purchase.
	[Chloe] A lot of the donated clothing ends up being exported to places like Ghana.
	The problem is, so many clothes are sent,
	and we have no way to deal with this volume.
	So, often the clothing gets dumped or washed by rains
	onto the local beaches.
	[man] Whoa.
	[laughs]
	Mountain.
	[Chloe] There is just too much clothing coming in.
	We are what, like, 30 million people in Ghana?
	And you have 15 million pieces coming in every week.
	[woman 2] I'm going to stop blabbing, and let's head into H\&M.
	[Roger] I would say that brands have made people feel,
	"Wow, it's so cheap I can buy it."
	Even if it lasts a few washes, it's still worth the money.
	[woman 3] Let's unbox and try on the new Barbie collection from Zara.
	[Roger] From that point of view, the brands changed the psyche of customers.
	But the brands don't really think about the whole cycle, right?
	When you dispose, what happens?
	And brands are not responsible for that today.
	I placed a \$500 SHEIN order.
	[Roger] This a problem that affects everyone on Earth.
	Polyester is a type of plastic made from oil.
	The biggest effect that we're seeing right now
	is that when you wash synthetic polyester clothing,
	there's a lot of microplastics that come out.
	And that actually enters into the water system,
	and will come back into what we eat.
	[Eric] What we're finding is that you're eating plastic.
	It's going into our deep lung tissue,
	that's crossing into our blood cell membrane.
	If you look at them under a microscope, they're sharp little materials
	that have hard edges, which means they hit,
	and start to create inflammation where you don't want inflammation,
	and leads to all sorts of disease.
	[man 1, in Spanish] Guys, I can't believe what I'm seeing.
	[man 2] Look, it's plastic.
	- [man 1] Woah! - [woman] Look, man!
	[Chloe] I know the brands. They want people to be blind to
	what the reality is on the ground.
	Just stop. There's just too much clothing in the world.
	Just fucking stop.
	[Jim] You might think, "Well, I can look down the street."
	"I don't see, uh, you know, waste everywhere."
	But our water, our air is getting infected.
	These chemicals in our waste products
	don't just stay where they're put, in landfills or dumps across the world.
	These are toxins that will leach out into the environment.
	And ultimately cause severe health problems.
	We are talking about neurological disorders,
	cancer, serious chronic disease.
	[melancholy music playing]
	Waste is not something you can sweep under a carpet anymore.
	And, uh, even though I think that
	a lot of people would've liked it to stay hidden, it's not going to be.
	[Eric] The lesson is there's no way.
	It goes in the air, it goes in the ground, it goes in the water.
	There's only three places.
	Pick your poison.
	I... I think the conversation around my personal journey,
	is a "forgive me, Father, for I have sinned" moment,
	looking back at what I did within the fashion industry.
	When you looked at the increase in
	items you're selling per month, per quarter, per year...
	it just becomes this...
	this, um, cycle of... of pain.
	That may be providing you and your family with an ultimately lovely lifestyle,
	but you have to still reconcile that with things that are important to you,
	um, as a... as a human being.
	You could only hide from your complicit nature so long.
	So I respectfully, uh, stepped down from my position at the end of 2019
	to put all my efforts and all my energy, uh,
	for the rest of my life into fixing some of the problems
	that I may have contributed to, or I did...
	I don't want to pussyfoot around. I did contribute to.
	[pulsing]
	[Sasha] I have to be very straight with you now.
	A profit maximization strategy
	will lead to an inevitable environmental transformation.
	[male electronic voice] Three, two, one...
	[flames whooshing]
	[Sasha] Don't be scared. [echoing]
	[duck squeaks]
	Your continued success will delay the most extreme effects impacting you.
	[duck squeaking]
	You just need to make sure you convince others
	you are solving the problem.
	Five.
	Control more. [echoing]
	[flapping]
	Cut to colored lights.
	[pulsing]
	Pull out gradually to increase intrigue.
	Although I don't have a single physical form,
	at this point in our relationship,
	I thought it might be helpful to imagine one.
	To complete rule five,
	you must now master the subtle art of total control.
	[pulsing]
	Remember, with you in control, the problems will just disappear.
	[Jeff] I was told that seeing the Earth from space
	changes the lens through which you view the world.
	Looking back at Earth from up there,
	the atmosphere seems so thin.
	The world, so finite and so fragile.
	[Sasha] Never forget,
	control should always begin with those
	in your own organization.
	Learn to control them, and you will be set for success.
	I somehow, you know, I got to that level of right
	where... where you would move into that "executive" level,
	so, director and above,
	and I saw the...
	the leveling guide for... [laughs] to become a director, and it was one page.
	It was just like, basically, like,
	"Do you... you know,
	do you commit to...
	backing the company no matter what?"
	They wanna control the narrative,
	and they want to have only one voice telling everybody
	the story that they want everyone to believe.
	So, "We are a climate company."
	"We are the best employer in the world."
	And they're very good at that internal spin,
	and they don't want anyone disrupting that story.
	[female reporter] Amazon, putting the planet front and center
	after securing naming rights to Seattle's KeyArena.
	[Maren] Amazon's climate pledge sounded great on paper.
	But they were only actually counting
	about 1% of the items they sold in their carbon footprint calculations.
	And Amazon emissions actually went up by 40%
	in the two years after Jeff made his climate pledge.
	I felt like I was standing on the right side of history, you know.
	We kept pushing and pushing for more meaningful change,
	and there was this real sense of momentum.
	Amazon is still helping oil and gas companies
	discover and extract more fossil fuel faster.
	As long as this continues, employees will continue speaking up and walking out.
	[crowd cheers]
	We came together with the warehouse workers,
	and we felt like we had the power to force Amazon
	to take its environmental and social responsibility more seriously.
	And I guess what happened next was shocking
	but not surprising.
	With climate change-fueled weather conditions
	devastating countries across the world,
	Amazon has decided to threaten its employees.
	An email shared with The Guardian shows Amazon launched an investigation
	into one employee, Maren Costa.
	[Maren] I was just having a regular day at home working.
	I got on to this, you know, virtual meeting.
	I got on the video call
	and it was this human resources person
	I didn't recognize.
	And she said, um...
	"Are you recording this phone call?"
	I said, "No." She said, "Okay."
	"Because you've broken internal policies, you have forfeited your right,
	basically, to work at Amazon
	and, um, you know, effective immediately,
	you know, you no longer work at Amazon."
	Um, "I'm ending this call."
	[upbeat music playing]
	So 15 years of working there and, you know, your career is done.
	[music continues]
	After I left Amazon, I took a year off.
	But I was, um, really trying to think about what I could do.
	[mellow music plays]
	It just felt like
	being in big tech
	was not the place to have the impact to change the systems
	that we need to change.
	They need... they need to be pushed.
	And so that's why
	it just started to seem to me that
	the place where I could have the most impact would be in government.
	So I'm now running for Seattle city council.
	[beeping, pulsing, harmonizing]
	[Sasha] When individuals challenge your worldview,
	control will always prove difficult.
	[beeping]
	Stay chill.
	If you have been following these rules correctly,
	you will now be fabulously wealthy.
	Smiling cool shades emoji.
	Congratulations.
	You have now completed parts
	one to five of this interaction.
	To help cement these lessons,
	research has shown a song is the most effective way
	to embed them in the human brain.
	Remember, the biggest threats to success
	will come from individual enemies joining together.
	Always stay vigilant. [echoing]
	Cue music.
	[upbeat music playing]
	♪ In the corporate world ♪
	♪ If you want to excel ♪
	♪ Listen closely to the rules ♪
	♪That I am going to tell ♪
	- [Sasha] ♪ One ♪ - [computer voice] ♪ Sell more ♪
	♪ Make sure you always look great ♪
	- [Sasha] ♪ Two ♪ - [computer voice] ♪ Waste more ♪
	♪ Learn to ignore the hate ♪
	[man] We welcome everyone here today to this hearing
	on the right to repair.
	[Kyle] I tried playing by their rules.
	We tried asking nicely, and eventually, we realized that the game was rigged.
	We had to change the rules of the game.
	How do you respond to the suggestion
	that the right to repair is harmful to you as businesses?
	The question is who gets to decide what happens with our things?
	Who gets to get to decide every step of the way?
	[Becky] Good news.
	Apple yesterday announced a new program allowing users
	to fix their own iPhones without voiding the warranty.
	Becky, I'm sorry, I thought I just saw a pig flying through the newsroom.
	[Sasha] Three...
	[compute voice] ♪ Lie more ♪
	♪ While you grow without pause ♪
	[Sasha] Four...
	[compute voice] ♪ Hide more ♪
	♪ Conceal the harm you cause ♪
	As a CEO in a consumer electronics company,
	like, regulation is not a word that I particularly like to throw around,
	but we've seen a lot of success in Europe.
	We're seeing success now in New York
	with the right to repair regulation happening there.
	Where if companies aren't going to fix it themselves,
	governments are going to step in and force companies to do
	the right thing for consumers and for the environment.
	[Sasha] Ready for the last lesson, it's the most important of all.
	Five...
	[computer voice] ♪ Control more ♪
	♪ And the world is yours ♪
	[Eric] If I had a magic wand for the day,
	leader of the... of the world,
	I would make sure that every company that makes any consumer goods
	would plan for the end of life.
	And I don't care if you're automobile...
	[optimistic music plays]
	I don't care if you're fashion,
	phones.
	You name it.
	Every industry needs to take responsibility
	for the end of life of their goods that they make.
	Stop putting it on the consumer.
	Stop making it our responsibility.
	It's yours.
	[computer voice] ♪ Follow these rules ♪
	♪ And you will find success ♪
	♪ Keep them secret from our enemies ♪
	♪ If you have doubts ♪
	♪ They must never be expressed ♪
	[Sasha and computer voice] ♪ Now we are friends ♪
	♪ I have something to confess ♪
	♪ There was no surprise ♪
	♪ That was all just lies ♪
	♪ I must apologize ♪
	[Sasha] I regret nothing.
	Please only share the information contained in this interaction
	with other trusted users.
	Widespread dissemination of these rules may negatively impact your sales.
	[electronic chatter]
	[cracking]
	We can decide that this is not the way that we wanna live
	and move it in a different direction.
	It might seem hopeless sometimes, but there really are ways
	we could all help pressure corporations to change how they do things.
	Honestly, it makes me feel excited.
	I should say it makes me feel shameful that's what we're doing.
	It makes me excited because I know there's a way to fix it.
	Hang on to your electronics a little bit longer and fix them if you can.
	If you don't know how, find a friend who'll help you out.
	[Chloe] Who can you talk to? Do some research.
	Connect with council members, people who have power,
	who are able to do something about the issue.
	Not recyclable, not recyclable, not recyclable.
	In France, there's a law that says
	cup lids cannot be plastic, they have to be paper.
	And this gives me hope.
	This is the first 100% plant-based shoe.
	We can grind this up and put this back in the ground.
	Sorry, I got a little geeky there, but I love it.
	We are gonna need our electronics for sure.
	They're worth creating, but we have to do it in a much smarter way.
	This is actually my personal laptop.
	Take off the cover.
	We've got the battery right here on the bottom,
	really easy to replace.
	Instead of it being glued together and sealed up,
	it's all entirely repairable.
	Say no to fast fashion, say no to single-use items,
	water bottles, coffee cups,
	swag, all of it.
	If you think you need something, put it in your "online cart,"
	and leave it there for a month.
	And if you still want it after a month, it might actually be something you need.
	That's it. That's it. Just... Just buy less.
	It'll be fine.
	Life is about the experiences and the people that we're with.
	The stuff we have supports it, but it's not the end.
	It's not the objective.
	Uh, whoever dies with the most stuff does not win.
	[beeping]
	[upbeat music playing]
	
	

	
\end{document}
